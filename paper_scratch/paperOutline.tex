\subsection{Outline of Sections}

% The remainder of the paper is organized as follows. In
% \secref{problemDefinition} we define more rigorously the problem of
% gait learning. \secref{experimentalSetup} describes our experimental
% setup, including the hardware we used and our methods for evaluating
% fitness of a gait. \secref{gaitGenLearn} discusses the different gait
% generation and learning methods we tested, and \secref{results}
% presents and discusses performance results.  Finally,
% \secref{conclusion} concludes and offers possible future extensions to this
% work.

The remainder of the paper is organized as follows. We begin by giving
a more rigorous \emph{Problem Definition}, then describe our
\emph{Experimental Setup}, including the hardware we used and our
methods for evaluating fitness of a gait. We then discuss methods used
for \emph{Gait Generation and Learning} and then present and discuss
\emph{Results}.  Finally, we offer several \emph{Conclusions and
  Future Work}.
