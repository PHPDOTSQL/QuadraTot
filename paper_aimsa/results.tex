\section{Results}

RL\_POWER Gaits

The results for the gaits evolved by RL\_POWER are shown in Figure 8
and Table 2. A total of 900 evaluations were performed for HyperNEAT
(300 in each of three runs). Overall the RL\_POWER gaits were the
faster and more stable by far, beating the HyperNEAT when comparing
either average or best gaits

(pic)RL\_POWER AND HyperNEAT performance chart


or best gaits. We believe that this is because RL\_POWER uses evolvable splines to represent gaits As explained earlier,  
The single best gait found during this study had a speed of

XXXX body lengths/minute, XX\%

better than the best HyperNEAT gait. Figure 9 shows a typical
RL\_POWER gait that had high fitness. The pattern of motion is less
complex and more regular than the HyperNEAT, in that patterns of
multiple motors are coordinated.

A corresponding observation of this property is that the noisiness of
RL\_POWER is significantly lower than that of HyperNEAT. One reason of
this is that the when Expectation Maximization methodology is used,
explorations happened in the vicinity space of the highest ranked
patterns. Even some individual HyperNEAT gaits turned out very
effective, the average noisiness of HyperNEAT is very large. For
example, as found in \cite{yosinski2011evolving-robot-gaits}, the mean of HyperNEAT generation-champion’s
performance was XX body lengths/minute (+-XX SD), but it had a max of
38 and a min of 3. RL\_POWER, on the other hand, proved to be stable
and much less noisy throughout the entire experiment. The maximum SD
over all three runs is +- XX SD. This may also be due to a much larger
space of possibilities of HyperNEAT than the more heuristically guided
reinforcement learning.

[pic][speed chart of two algos]


