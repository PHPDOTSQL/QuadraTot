\section{Methods}

how we tried to solve it...

Parameterized Gaits by RL\_PoWER: Here we used an RL approach to change the complexity of the policy representation dynamically while the trial is running. In Petar et al’ s studies on reducing energy consumption for bipedal robots [xx], a mechanism that can evolve the policy parameterization was used. The method starts from a very simple parameterization and gradually increases its representational capability. The method tested to be able to generate an adaptive policy parameterization that can accommodate increasingly more complex policies. In the work described in [xx], the policy generated by this approach can reach the global optimum at a faster rate. Also found is that the chance of converging to a suboptimal solution is reduced, because in the lower-dimensional representation the effect is less exhibited. Spline policy representationThe simplest model with back-compatibility is geometric splines. For a given model f(x) with K knots, we can preserve the exact shape of the generated curve while adding extra knots to the original spline. Say, if we put one additional knot between every two consecutive knots of the original spline, we end up with a 2K - 1 knots and a spline that has the same shape as the original one. In order to do this, we need to define an algorithm for evolving the parameterization from K to L knots (L > K ), which is formulated in Algorithm 1.  Without loss of generality, the policy parameters are normalized into [0, 1], and appropriately scaled/shifted as necessary later upon use.[pic]Fig. 2 illustrates the process of using spline representation for the evolving policy parameterization. Fig. 1 shows an example for a reinforcement learning process using evolving policy parameterization to approximate an unknown function. Petar et alIntegrating evolving policy parameterization with Hill-climbing(RL?)Kober et al  proposed (Kober et al) a RL algorithm called Policy learning by Weighting Exploration with the Returns(PoWER), which is based on Expectation-Maximization  algorithm (EM). Our proposed technique for evolving the policy parameterization are combined with this EM-based RL algorithm PoWER [xx], due to its relatively fewer parameters that need tuning. Also because PoWER demonstrated high performance in tasks learned directly on real robots, such as pendulum swing-up and pancake flipping task [9].We evolved the policy parameterization only on those past trials ranked the highest by the importance sampling technique used by the PoWER algorithm(). The intuition behind is that high-ranked parameterizations have more potential to perform even better in the future. Besides, evolving all the parameterizations requires significantly more efforts. (Bongard et al.,2006) Since our experiment is done on physical robot, explore all the variations of every parameterization is not practical. A promising extension can be, using physical simulator for the robot, to further explore the parameterization space by evolving more parameterizations. Gaits of RL\_POWER:For the experiment, we have 3 knots for each spline and 8 splines in total for the servos. The servo in the hip is not used in our experiment. Previous work has verified that quadruped gaits perform better when they are coordinated (Clune et al., 2009a, 2011; Valsalam and Miikkulainen, 2008). So the gaits For each set of splines, we calculate its corresponding parameterized gait within a time cycle. Given the pattern of one unit cycle, then apply the same pattern to every cycle throughout the whole period of trial. Specifically, each spline is stretched to servo positions as following:   rescale for one cycle: 	Unit Cycle :s(si).cyclePos = s(si).YY .* posRange + posMin;    		s(si).cyclePos = s(si).cyclePos(:);    	Repeat cycle,      	Cycle[i] = UnitCycle% the last pos in a cycle is the same as the first pos in the next cycle, from [pic for one gait]
