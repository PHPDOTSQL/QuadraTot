%A reinforcement learning approach which can evolve the policy
%parameterization dynamically during the learning process for quadruped
%robot gait \edit{???}. By gradually increasing the representational
%power of the policy parameterization, it manages to find better
%policies faster than HyperNEAT, a state-of-the-art evolutionary ANN
%algorithm that achieved the best performance in previous studies.
%
%Expectation-Maximization-based reinforcement learning algorithm is
%used. The test was done on a physical quadrupedal robot
%(QuadraTot). The results show that the evolving policy
%parameterization combined with expectation-maximization (RL PoWER)
%outperforms substantially the HyperNEAT, by having faster convergence
%and higher final reward. According to the statistics collected, the RL
%PoWER achieves 16.3\% advantage over the HyperNEAT, the best method
%tested on QuadraTot in previous studies.

Legged robots are uniquely privileged over their wheeled counterparts
in their potential to access rugged terrain. However, designing
walking gaits by hand for legged robots is a difficult and
time-consuming process, so we seek algorithms for learning such gaits
automatically using real world experimentation. Numerous previous
studies have examined a variety algorithms for learning gaits, using
an assortment of different robots. It is often difficult to compare
the algorithmic results from one study to the next, because the
conditions and robots used vary. With this in mind, we have used an
open-source, 3D printed quadruped robot called QuadraTot, so the results may be
verified, and hopefully improved upon, by any group so
desiring. Because many robots do not have accurate simulators, we test
gait-learning algorithms entirely on the physical robot. Previous
studies using the QuadraTot have compared parameterized splines, the
HyperNEAT generative encoding, and (Kyrre's algorithm). Here we
compare these results to an algorithm called Policy learning by
Weighting Exploration with the Returns, or RL PoWER. We report that
this algorithm has learned the fastest gait yet reported in the
literature, 16.3\% faster \edit{different number than in results?} than reported for HyperNEAT and \edit{???
compared to Kyrre's?}. In addition, the learned gaits are less taxing
on the robot and more repeatable than previous record-breaking gaits.
