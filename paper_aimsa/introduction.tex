\section{Introduction}

\figp{quadratotWhiteBooties}{.75}{Example figure...}

Various learning algorithms have been tested to be effective for
legged robots. Their studies tested the viability of applying RL
method to legged robot gaits learning. Competitive performance has
been verified on methods such as HyperNEAT (Yosinski et al, 2011) and
GA(Chernova et al, 2004) and other methods(Hornby et al., 2005; Zykov
et al., 2004; T´ellez et al., 2006; Valsalam and Miikkulainen,
2008). Despite the excellent performance of these algorithms, a major
part of achieving good performance relies in tuning the parameters for
the evolutionary algorithms(Kormushev et al, 2006). Here we present a
different way for learning gaits using RL(Hill climbing). In our
experiment, it turned out this method achieved competitive performance
with Problem Definition:The gait-learning problem is defined to find a
gait that maximizes some specific metric. Mathematically, we define a
gait as a function that specifies a vector of commanded motor
positions for a robot over time. Gaits without feedback --- also
called open-loop gaits --- can be defined as

\be\vec{x} = g(t)\ee

According to the definition, open-loop gaits are deterministic. One
particular gait should behave exactly the same when it is run from
trial to trial. However, the actual robot motion and fitness measured
will vary due to the noisiness of the real world physics.In our
trials, one gait pattern was generated and sent to the robot and
executed in an open loop manner, as defined. In particular, in this
paper, we chose to compare open-loop gaits generated specifically by
the RL\_POWER and HyperNEAT. The particular metric used will be
described later.
