HyperNEAT has proven to be an effective encoding in gait evolution and is able to find gait regularities without human intervention \cite{clune1}, \cite{clune2}, \cite{yos:clune}. 
Glette et al. \cite{glette} has shown that evolving gaits in simulation using a simple algorithm with a large number of evaluations could outperform gait evolution using a more sophisticated algorithm in hardware.  
This was possible because the simulator afford a much larger number of evaluations to be performed during the evolution process. 
In this study, we took the two -- a sophisticated algorithm (HyperNEAT) and a physical simulator -- to see how HyperNEAT would perform with a many evaluations in the evolution process. 
We showed that HyperNEAT, when used with a simulator, could evolve very impressive gaits. 
In simulation, HyperNEAT gaits outperformed Genetic Algorithm gaits \cite{glette} by 54.3\%. 
In the real life trials, HyperNEAT's best gait performed 5.4\% better than the GA gait. 
This study also showed that HyperNEAT will exploit the characteristics of the problem in order to maximize its fitness.
Without frequency limiters, it would make the legs move really fast in order to increase fitness. 
When that was no longer allowed, it pushed the servo positions to their boundaries by increasing the stride lengths as wide as possible. 
However, because of the physical limitations of the robot, these exploitations could not be transferred well to reality. 
Moreover, the inevitable faults with the simulator (it cannot model the robot or the real world perfectly) added to the difficulty in the transfer from simulation to reality. 
Thus, many of the gaits performed much worse than it did in simulation and had difficulty in achieving consistent results. 


Next steps for this research heads towards improving the transferabilities of the gaits from simulation to reality. 
The performance gap between simulation and reality is called the \emph{reality gap} \cite{jakobi}. 
Many different approaches to reduce the reality gap has been proposed \cite{koos1}, \cite{bongard}, \cite{zagal}.
To improve upon the results from this study, one or more of these approaches could be used to aid in the transfer to reality. 
The one easiest and most promising to implement is the Transferability Approach \cite{koos2} which has shown to be successful in quadrapedal gait evolution. 
HyperNEAT, used with a simulator and a transfer approach, shows much hope in producing yet a better, more coordinated, and higher performing gait on the QuadraTot platform. 
