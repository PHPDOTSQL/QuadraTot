% !TEX root =main.tex

\edit{Jeff: use me} Overall, these results provide important confirmation that the HyperNEAT generative encoding can evolve state-of-the-art results for challenging engineering problems, such as evolving gaits for real, legged robots. The results also reaffirm the benefits of using simulators when solving real-world challenges with evolutionary algorithms. 


\edit{Last: check and edit this section.}

HyperNEAT has proven to be an effective encoding in gait evolution with an ability to find gait regularities without human intervention~\cite{clune2009evolving,clune2011performance,yos:clune}. Glette et al. 2012~\cite{glette} showed the benefits of a physical simulator by evolving, with a simple genetic algorithm, a gait that outperformed those HyperNEAT evolved in hardware.
In this study, we took the two -- HyperNEAT and a physical simulator -- and showed that HyperNEAT's performance significantly increases with longer evolution periods. The gaits produced in this study outperformed the previous best gait on the QuadraTot by 54.3\% in simulation and by 5.4\% in real life trials. This study also showed that HyperNEAT will exploit the characteristics of the problem, such as frequency and regularity, in order to maximize its fitness.
%In simulation, HyperNEAT gaits outperformed Genetic Algorithm gaits~\cite{glette} by 54.3\% and by 5.4\% in the real life trials. 


%Without frequency limiters, it would make the legs move really fast in order to increase fitness. 
%When that was no longer allowed, it pushed the servo positions to their boundaries by increasing the stride lengths as wide as possible. 
However, we faced a problem in the transfer from simulation to reality. %because of the physical limitations of the robot, these exploitations did not transfer well to reality. %could not be transferred well to reality.
One of the reasons was that the robot was greatly underpowered and could not perform the transferred gaits. The other reasons was the inevitable faults with the simulator (as it cannot model the robot or the real world perfectly). Because of these factors, many gaits performed much worse than they did in simulation and had difficulty in achieving consistent results. 


Next steps for this research heads towards improving the transfer process of the gaits from simulation to reality by reducing or managing the \emph{reality gap}~\cite{jakobi}. Many different approaches to reduce the reality gap have been proposed~\cite{koos2010crossing,bongard,zagal}. One of the easiest and most promising of these methods is the Transferability Approach~\cite{koos2011transferability}, which has already proven to be successful in quadrapedal gait evolution. Another method that we could take is to evolve the simulator itself to adapt itself to the real world. HyperNEAT, used with a simulator and a transfer approach, shows much hope in producing yet a better, more coordinated, and higher performing gait for many robotic platforms.
