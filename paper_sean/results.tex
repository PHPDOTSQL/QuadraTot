% !TEX root =main.tex

%We
%compare performance in simulation and hardware
%to available data from previous studies.
%\subsubsection{Simulation:}

The simulator enabled HyperNEAT to evolve
fast gaits that look like those of natural animals. In simulation, HyperNEAT gaits were significantly faster than those from Glette et al. (\figref{FINAL_kyrre_sean_fitness}, $\emph{p} < 6.8\times10^{-8}$, comparing the best gaits in the final generation of each run via MATLAB's Wilcoxon rank sum test).  
Specifically, HyperNEAT gaits were 55\% faster on average in simulation (25.4
cm/s $\pm$ 3.4 SD versus 16.7 cm/s $\pm$ 1.9 SD, \tabref{results}).\edit{Table 1, not 3} 
Plots of servo positions over time reveal that the evolved HyperNEAT gaits are regular, smooth and coordinated~(\figref{servo_plot_111}), confirming previous results with HyperNEAT in simulation~\cite{clune2009evolving,clune2011performance}.

\figp{FINAL_kyrre_sean_fitness}{0.62}
{HyperNEAT outperforms a genetic algorithm with a simpler encoding, when both algorithms are combined with a simulator. Plotted are means over 20 runs in simulation (solid lines) $\pm$ STD (dashed lines). HyperNEAT gaits are 54.3\% faster in
simulation and 5.4\% faster in reality~(\tabref{results}).}

\figgp{servo_plot_111}{.48}{frequencyPlotThreshFinal}{.48}
{\textbf{Left: }
Servo positions over time (for nine servos) for a simulated HyperNEAT gait. HyperNEAT produces gaits that are smooth, symmetrical, and contain other regularities. 
\textbf{Right: }Mean gait frequency averaged over 20 runs. 
Gaits with frequencies above a threshold (horizontal line) receive a fitness penalty. HyperNEAT quickly learns to produce gaits with frequencies low enough to avoid this penalty. }


On the physical robot, HyperNEAT gaits from this study outperformed gaits from all previous Quadratot studies~(\tabref{results})~\cite{yos:clune,haocheng,glette}, including those of Glette et al. 
 However, comparing performance in hardware between studies performed in
different laboratories is difficult, not only because reality is
inherently noisy, but because even two copies of the same robot are
not identical and may produce different gait speeds for the same input
gait. Gaits for this study and two previous studies~\cite{yos:clune,haocheng} were evaluated on a copy of the QuadraTot robot in the Cornell Creative Machines Lab (CCML), while the gaits in Glette et al. were evaluated on a different copy of the same robot in the Robotics and
Intelligent Systems (ROBIN) lab at the University of Oslo.

The previous fastest gait on any copy of a QuadraTot was from Glette
et al., and traveled 17.8 cm/s on the ROBIN QuadraTot. The
fastest gait produced by HyperNEAT with a simulator in the experiments for this paper traveled 14.5
cm/s on the CCML QuadraTot. It was unclear whether this difference in performance was due to
the differences in the gaits themselves or dissimilarities between
hardware. To control for this possibility, we ran the fastest gait
from Glette et al. 10 times on CCML and measured an average speed of
only 12.95 cm/s (vs. 17.8 cm/s measured on ROBIN) and a maximum of
13.76 cm/s (\tabref{results}). It thus appears that the CCML version of the robot is slower. Moreover, the HyperNEAT gait (14.5 cm/s) is faster than the best gait of Glette et al. on the CCML QuadraTot. Unfortunately, it was not possible to test the best HyperNEAT gait on the ROBIN QuadraTot. Because HyperNEAT outperformed the simple encoding from Glette et al. on the same robotic hardware, we tentatively
conclude that HyperNEAT produces faster gaits for the QuadraTot robot. This conclusion is supported by the fact that HyperNEAT outperformed the encoding from Glette et al. in simulation. 

\begin{table}
\begin{center}
\begin{tabular}{|l|r|r|r|r|}
\hline
                                         &              & Simulated  & Real Vel. & Real Vel.  \\
                                         & Evaluations  & Velocity &    (CCML)     & (ROBIN) \\
\hline
Parameterized gaits + optimization \cite{yos:clune}   &153    & --    & 5.8 & --\\
\hline
HyperNEAT in hardware \cite{yos:clune}                 & 180         & --         &   9.7  & --   \\
\hline
RL PoWER Spline \cite{haocheng}                         & 300         & --         &   11.1 & --\\
\hline
GA + simulator \cite{glette}             & 60000       & *16.7       &   13.0   & \textbf{17.8}  \\
\hline
HyperNEAT + simulator [this paper]                     & 40000       & \textbf{25.4}       &   \textbf{14.5} & --\\
\hline
\end{tabular}
\end{center}
\tablabel{results}
\caption{Velocities of learned gaits in simulation and on two different copies of the QuadraTot robot. Subject to availability, data are reported from the experiments for this paper and three previous studies. Reported are the total number of evaluations per run, the average of the fastest gaits produced in each run in simulation, and the fastest gait produced on two different copies of the QuadraTot robot (CCML and ROBIN, see text). *Instead of using the single fitness value reported in~\cite{glette}, we ran 19 additional runs and used the average fitness of those 20 runs. Velocities are in cm/s, and bold indicates the best performance.}  
\end{table}

We now discuss the differences between the gaits evolved by HyperNEAT directly on the physical robot, and those first evolved in a simulator and then transferred to the robot. 
On the physical robot, the gaits produced by HyperNEAT with a simulator were faster, more natural, and more repeatable than those evolved directly on the Quadratot robot~\cite{yos:clune}. The gaits were also more regular, as they were in simulation~(\figref{servo_plot_111}). This result is important because it confirms that HyperNEAT can produce the important property of
regularity in a challenging, real-world domain, which was not previously observed when
evolving directly on the hardware~\cite{yos:clune}. Producing regular solutions is a key to exploiting regularity in difficult engineering problems~\cite{clune2011performance}. 


The simulator likely improved performance because of the number of evaluations it enabled, both in terms of the population (200 vs. 9) and the number of generations (200 vs. 20), leading
to a total difference of 40000 vs 180 per evolutionary run~\cite{yos:clune}. Another
potential cause of improved performance is that with a simulator
HyperNEAT had much less noise in the evaluation process during early
generations and could thus find coordinated, regular, gaits, which are
both higher-performing and more likely to transfer. On the physical robot, the noise in the evaluation was
substantial, preventing effective learning and the discovery of
effective gaits~\cite{yos:clune}. It could also be the case that 
the addition in this study of a punishment for high-frequency gaits
aided performance, especially since in simulation the gaits were
high-frequency in a few early generations before rapidly settling to a
frequency range below the penalized threshold~(\figref{servo_plot_111},
right).



While this study was able to produce the fastest QuadraTot gait to
date, most of the gaits in simulation did not transfer well to
reality. Many gaits that were fast in simulation performed
poorly on the real robot, largely due to servos that were too weak and
shut down, or because of differences between simulation and
reality. Repeated attempts to minimize these problems were
unsuccessful.  A major, unavoidable cause of the servo shutdowns was
the design of the robot: the force required to lift the robot in many
poses was too much for the servos. Future studies will use a
robot that has more mechanical advantage and requires less torque from
each servo, such as the Aracna platform~\cite{lohmann2012aracna},
which was designed in response to the challenges experienced with the
QuadraTot platform.











