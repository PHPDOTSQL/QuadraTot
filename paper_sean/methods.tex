% !TEX root =main.tex
We evolved gaits first in simulation and then transferred them to the physical QuadraTot robot, as in Glette et al. \cite{glette}. 
Each run had a population size of 200 and lasted 200 generations. 20 runs were performed, each of which was identical except for the seed provided to the random number generator, which affected stochastic events such as mutation. 
Each gait was evaluated for 12 seconds, both in simulation and in reality. 
All HyperNEAT parameters were identical to those in Yosinski et al. \cite{yos:clune} except for the frequency of the sine wave input to the ANN, which was lowered to decrease the number of servo shutdowns do to high-frequency gaits. 
With the sine wave frequency used for Yosinski et al. \cite{yos:clune} (0.0265Hz), HyperNEAT often evolved high frequency gaits (over 8Hz per servo) that performed poorly in reality due to servo constraints. 
This led to dramatically different gait performance in real-life compared to simulations. 
To address this, we changed the frequency of sine function to 0.004Hz so that HyperNEAT would lean towards evolving low frequency gaits.

%\begin{equation}
%sin(2\pi t/250)\pi
%\end{equation}

%from
%
%\begin{equation}
%sin(t/6)\pi
%\end{equation}
%

 %Sine frequency of 0.004 was used because we experimentally found that HyperNEAT was able to produce natural and high-performing gaits with low frequencies with this frequency. 
We also added a policy that punished high frequency gaits while evolving in simulation. 
If the average frequency of each leg for the gait was higher than the threshold frequency, the fitness was punished by exponentially reducing the distance travelled by the robot:

\begin{equation}
distance = distance*e^{(freq_{thr}-freq)} \\
\end{equation}

We chose the threshold frequency for punishment to be 1.67 Hz/servo. After implementing the frequency punishment policy, the servos shutdown much less frequently. 


The fitness was evaluated using the following equation:

\begin{equation}
fitness = 2^{d^{2}}
\end{equation}

where \emph{d} is the Euclidean distance the center of mass of the robot travelled during the gait duration. %as was done in Clune et al. \cite{clune1}.

%In simulation, the distance travelled was calculated using the output file generated by the simulator.
%In real-life, the distance travelled was calculated using the log file recorded using the Wiimote system described earlier in the paper. For real-life trials, the actual fitnesses were not calculated and the gaits were evaluated using only the velocity of the robot. 


To evolve gaits using the physical simulator, we wrote a Python script\footnote{All scripts and drivers available at http://quadratot.yosinski.com} that interfaced between the simulator and HyperNEAT. % which allowed us to evolve gaits fast and efficiently with no human intervention.
The runs were performed on three separate machines all running Ubuntu 11.04 using the same HyperNEAT settings and files to speed up the simulation process.


After evolving the gaits in simulation, the champion gait of the last generation of each of the 20 runs was transferred onto the real robot and evaluated for its performance on the real robot.
In real-life trials, the motion of the robot was interpolated linearly between a stationary pose and the commanded gait for two seconds at the beginning and the end. 
The position of the robot was measured at the beginning and the end of each run, starting from the stationary pose and ending with the stationary pose. 
If a gait performed extremely poorly due to servo shutdowns, the gait was tried again until we could conclude that poor performance is because of the gait and not due to the condition of the robot such as overheated servos.

