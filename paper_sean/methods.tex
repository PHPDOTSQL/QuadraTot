% !TEX root =main.tex

\subsubsection{Evolutionary Process and Parameters:}

Each run had a population size of 200 and lasted 200 generations. We performed 20 HyperNEAT runs that differed only in the seed provided to the random number generator, which affected stochastic events such as mutation. To make a statistical comparison to the encoding from Glette et al., we conducted 19 runs using the original code to supplement the one run performed for that paper. 
Each gait was evaluated for fourteen seconds in reality, with interpolation from and to a stationary position in the first and last two seconds, respectively, as in Yosinski et al. effectively resulting in 12 seconds of full-speed
motion. In simulation gaits were evaluated for 12 seconds. 
All HyperNEAT parameters were identical to those in Yosinski et al. except for the frequency of the sine wave input to the ANN, which was lowered from 4.2Hz to 0.64Hz to reduce servo shutdowns. 
To further reduce servo shutdowns, we punished high-frequency gaits during evolution. We calculated the frequency of a gait as the average number of servo direction changes per leg per second. If this frequency was higher than the experimentally-selected threshold of 1.67 Hz, the measure of distance traveled by the center of mass during the gait was reduced exponentially by multiplying it by a discount factor of $e^{(1.67-\mathrm{freq})}$. Following Clune et al.~\cite{clune2009evolving}, the fitness equation was $2^{\mathrm{distance}^{2}}$. After evolving the gaits in simulation, the champion gait of the last generation of each of the 20 runs was transferred onto the real robot and the distance traveled was measured. 

