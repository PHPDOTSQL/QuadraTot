Gaits were first evolved in simulation, as previous done in Glette et al. \cite{glette}. 
During the evolution process, we evolved 200 organisms for 200 generations in 20 runs with different initial conditions specified by randomly seeding the run. 
To keep in line with previous studies that have been performed on the QuadraTot platform, each gait ran for 12 seconds. 
All the HyperNEAT settings used for gait generation was kept the same as in Yosinski et al. \cite{yos:clune} except for the frequency of the sine wave function used. 
Through experimentation, we found that high frequency gaits, gaits that require servos to change their directions often, were much more prone to servo shutdowns than low frequency gaits (at around 1.5Hz per servo). 
HyperNEAT often evolved high frequency gaits (over 8Hz per servo) which performed poorly in reality due to servo constraints. 
This led to gaits performing dramatically different in real-life compared to simulations. 
To address this issue, two changes were made:
First, the frequency of the sine wave function used in HyperNEAT was reduced by a factor of 6.63 by changing the sine function to

\begin{equation}
sin(2\pi t/250)\pi
\end{equation}

from

\begin{equation}
sin(t/6)\pi
\end{equation}


so that HyperNEAT would lean towards evolving low frequency gaits. Sine frequency of 0.004  was used because we experimentally found that HyperNEAT was able to produce natural and high-performing gaits with low frequencies with this frequency. 
Second, a policy that punished high frequency gaits was implemented during the evolution process. 
If the average frequency of each for the gait was over the threshold frequency, the fitness was punished by exponentially reducing the distance travelled by the robot:

\begin{equation}
distance = distance*e^{(freq_{thr}-freq)} \\
\end{equation}

Through experimentation, we chose the threshold frequency to be 1.67 Hz/servo. After implementing the frequency punishment policy, the servos shutdown much less frequently. 


The fitness was evaluated using the following equation:

\begin{equation}
fitness = 2^{d^{2}}
\end{equation}

where \emph{d} is the distance the center of mass of the robot travelled during the given time as was done in Clune et al. \cite{clune1}.
The distance travelled was calculated as the Euclidean distance travelled by the robot in the XY plane during the alotted time, in centimeters. 

In simulation, the distance travelled was calculated using the output file generated by the simulator.
In real-life, the distance travelled was calculated using the log file recorded using the Wiimote system described earlier in the paper. For real-life trials, the actual fitnesses were not calculated and the gaits were evaluated using only the velocity of the robot. 


To evolve gaits using the physical simulator, we wrote a Python script\footnote{All scripts and drivers available at http://quadratot.yosinski.com} that interfaced between the simulator and HyperNEAT which allowed us to evolve gaits fast and efficiently with no human intervention.
The runs were performed on three separate machines all running the same version of Linux (Ubuntu 11.04) using the same HyperNEAT settings and files to speed up the simulation process.


After evolving the gaits in simulation, the champion gait of the last generation of each of the 20 runs was transferred onto the real robot and evaluated for its performance on the real robot.
In real-life trials, the motion of the robot was interpolated linearly between a stationary pose and the commanded gait for two seconds at the beginning and the end. % figure!! 
The position of the robot was measured at the beginning and end of each run, starting from the stationary pose and ending with the stationary pose. 
If a gait performed extremely poorly due to servo shutdowns, the gait was tried again until we could conclude that poor performance is because of the gait and not because of the condition of the robot.

