% !TEX root =main.tex
Creating gaits for physical robots is a longstanding and still open challenge. 
%Evolutionary algorithms can outperform human engineers in generating such gaits, but this approach historically required engineers to manually decide which types of gaits would evolve. Such manual intervention is time consuming, requires expert knowledge, and adds constraints that may hurt performance. 
Recently, the HyperNEAT generative encoding was shown to automatically discover a variety of gait regularities, producing fast, coordinated gaits, but only for simulated robots. A follow-up study found that HyperNEAT did not produce impressive gaits when they were evolved directly on a physical robot. A simpler encoding hand-tuned to produce regular gaits was tried on the same robot, and outperformed HyperNEAT, but these gaits were first evolved in simulation before being transferred to the robot. In this paper, we tested the hypothesis that the beneficial properties of HyperNEAT would outperform the simpler encoding if HyperNEAT gaits are first evolved in simulation before being transferred to reality. 
That hypothesis was confirmed, resulting in the fasted gaits yet observed for this robot. This result is important because it confirms that the early promise shown by generative encodings, specifically HyperNEAT, were not limited to evolving gaits in simulation, but indeed work on real robots. 




%We further test whether gaits created by generative encodings such as HyperNEAT tend to be transferable from simulation to reality, which is an important question because the regularity-producing property of generative encodings should make the gap smaller. could generate gaits that could outperform manually designed, regularity-constrained, directly-encoded gaits. 
%
%For our experiments, gaits were evolved in simulation and the final gait transferred to the real-world to evaluate the gait's actual performance.
%The platform used was the QuadraTot, an open-source, 3D printed quadrupedal robot on which previous studies in gait evolution has already been done several times. 
%In this study, we produced the fastest gait ever tried on this platform, outperforming the fastest published gait by 5.4\% with 20000 fewer evaluations run during gait generation. 
%The gaits produced showed exploitation in regularity in simulation and real-world as was expected.
