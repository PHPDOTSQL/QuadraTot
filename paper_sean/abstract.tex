Evolving gaits for physical robots is a longstanding and still open challenge. 
While previous work has shown that evolutionary algorithms can outperform human engineers in generating gaits, in all of that work the experimenter has had to manually decide the regularities of the gaits. 
Such manual intervention is time consuming, requires expert knowledge, and adds constraints that may hurt performance. 
In this paper, we tested whether HyperNEAT -- a generative encoding that has shown to be able to automatically evolve a variety of high-performing regular gaits -- could generate gaits that could outperform manually designed, regularity-constrained, directly-encoded gaits. 
For our experiments, gaits were evolved in simulation and the final gait transferred to the real-world to evaluate the gait's actual performance.
The platform used was the QuadraTot, an open-source, 3D printed quadrupedal robot on which previous studies in gait evolution has already been done several times. 
In this study, we produced the fastest gait ever tried on this platform, outperforming the fastest published gait by 5.4\% with 20000 fewer evaluations run during gait generation. 
The gaits produced showed exploitation in regularity in simulation and real-world as was expected.
