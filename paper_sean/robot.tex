% !TEX root =main.tex
\subsubsection{Robot Hardware:}
%\figgp{quadratotWhiteBooties}{.45}{quadratot_simulator}{.45}{captioncaption}

We performed experiments on the QuadraTot quadrupedal robot platform~(\figref{quadratotWhiteBooties})~\cite{yos:clune}.
It has 9 degrees of freedom: two joints per leg and one joint that rotates along the robot's midline. %we def want a picture of the cool robot. I propose putting the picture of the robot on the left and the picture of the simulated robot on the right of the same figure. That also does a nice job of highlighting the theme of the paper (robot+sim) in one figure. Try to get this figure to show up as early as possible in the paper (e.g top of the 2nd page). 
The QuadraTot hardware designs and the software for this project are open source\footnote{A parts list, hardware CAD files, software (including the simulator), and gait videos are available at http://creativemachines.cornell.edu/evolved-quadruped-gaits}, and all hardware components are either off-the-shelf or 3D-printed. %These traits enable other labs to perform research using the same platform, which eases comparison of different gait-learning algorithms. 
There are results on the platform for nine different learning algorithms from three previous publications~\cite{yos:clune,glette,haocheng}. 


The joints are powered by Robotis Dynamixel servos; five AX-18A servos for the inner joints of each leg and the single midline joint, and four AX-12A servos for the the outer joints of each leg, which require less power and can thus have less expensive motors. The servos were sent new positions at 40Hz via the Pydynamixel library. Each servo has a built-in safety mechanism that shuts itself off to prevent damage if the servo's current, range, temperature, or torque is too high. During evolution, this safety mechanism frequently activated, and did so inconsistently, adding significant noise to the evaluation process. As pointed out in a previous study~\cite{yos:clune}, gaits generated during evolution on QuadraTot are highly variable and produce many shutdowns because they force the servos to exert too much torque.
To prevent collisions between different pieces of the robot's body, we limited the allowable range of movement for the inner, outer, and hip joints to [-85\degree, +60\degree], [-113\degree, +39\degree], and [-28\degree, +28\degree], respectively. We also implemented the Smart Cropping System from Shen et al.~\cite{haocheng}, which prevents combinations of joint positions for the inner and outer joint of each leg that generate extreme amounts of torque. A final method of reducing torque was to reduce the weight of the robot. Yosinski et al. had the small Linux computer that performed all computation on the robot, but we removed it and sent commands from it to the robot via a cable. We tracked the robot's position using an infrared LED observed by a Wiimote.
