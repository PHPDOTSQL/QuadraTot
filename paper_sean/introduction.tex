% !TEX root =main.tex
Robots that can move around on legs can operate in a much wider range of environments than their wheeled counterparts. A downside is that designing gaits for legged robots is a difficult and time-consuming process~\cite{storm:green}. 
But designing a gait for a robot is often difficult and time consuming because the engineer must take into consideration many factors which must all fall into place for a successful gait pattern. 
Because of setbacks present in hand-designing gaits, evolving gaits with machine learning algorithms is an attractive and flexible approach that can automatically produce effective gaits for many different robot platforms with minimal human intervention. 

Much research has been done in this subject, and previous work has shown that learned gaits can outperform designed gaits \cite{valsalam:mii,kohl:stone,hornby1,hornby2}. %JMC/SL- I have a similar claim in my CEC 2009 paper. Grab those cites for here (especially the Hornby ones). / have added hornby citations
It has also been shown that gaits with regularity (coordination in leg movement, such as left-right symmetry or front-back symmetry) perform better than gaits without regularities \cite{valsalam:mii,clune2,clune1,clune3}. %JMC/SL-note how you can put multiple cites in one bracket...but no spaces after the comma! / gotcha!
However, in most previous works, the experimenter has had to manually decide the regularities of the gaits \cite{valsalam:mii,tellez,beer,raibert} . %JMC/SL - we need cites for this claim. you can get them from Clune1 / done
Such manual intervention is time consuming, requires expert knowledge, and adds constraints that may hurt performance. 
Previous work has shown that the Hypercube-based NeuroEvolution of Augmenting Topologies (HyperNEAT) generative encoding \cite{stanley1} can automatically generate a variety of regular gaits and that it outperforms direct encoding controls on this task \cite{clune1}, \cite{clune2}. 
However, that work only verified these claims in simulation. 
A follow-up paper by Yosinski et al. evolved gaits with HyperNEAT directly in hardware on the QuadraTot robot platform and found that HyperNEAT's gaits outperformed manually designed parameterized learning algorithms, but still did not produce impressive, natural gaits \cite{yos:clune}. 
A further follow-up study built a simulator for the QuadraTot robot platform to test whether the inclusion of a simulator would improve results and found that it did:  an evolutionary algorithm with direct encoding in simulation produced gaits that outperformed those gaits evolved by HyperNEAT in hardware \cite{glette}. %JMC/SL-Should there be a space between the end of the sentence and the cite? Whatever the answer, we should be consistent. If so, I have added here with the ~. The ~ tells Latex not to split that space from the character before and after it. / There should have been and will make changes accordingly
The simulator reduced the time required to evaluate gaits and afforded much larger population sizes and more generations than was possible when evolving in hardware, ultimately performing 333 times as many evaluations as HyperNEAT had in hardware per run (60000 vs. 180) \cite{yos:clune}, \cite{glette}. 
The evolutionary algorithm used in Glette et al. \cite{glette} was manually constrained to evolve specific, hand-chosen regularities. 
%Another study evolved gaits in hardware on the QuadraTot platform using RL PoWER, a spline-based algorithm, and was successful in evolving faster and more repeatable gaits than HyperNEAT evolved in hardware \cite{haocheng}. %JMC- I don't think we need to cite this paper...it is a non-evolutionary method, and it doesn't outperform Glette...and the sentence just sticks out as a detail that isn't necessary to our story. 

In this paper we test whether the benefits of HyperNEAT -- higher performance and the automatic discovery of complex, effective regularities \cite{clune2,yos:clune} -- will outperform a simple genetic algorithm manually designed to produce regularities once HyperNEAT is combined with a simulator. %JMC/SL-I don't think they used a direct encoding. I prefer describing it as simple and manually designed to produce specific regularities. / changed to reflect that
We test this by evolving gaits using HyperNEAT with a physical simulator and then comparing the performance of these gaits in the real-world to those that were evaluated on the QuadraTot robotic platform in previous publications.  %JMC - Try to write without ever saying ``use'' or ``using.'' If you do, you'll find how much it is overused in writing, as evidenced by this last sentence. Often the word is a stand in for a different, clearer verb...and its use is ambiguous. Please search for this word and try to swap it out throughout the rest of the paper. Often times it requires restructuring a sentence, but the sentence usually ends up better. Similarly, make sure to always follow the word ``this'' with referent. This is extremely ambiguous. You know what you refer to when you write the sentence, but readers (often including your future self!) do not know precisely what you mean, and in science precision is critical. 
%Our experiments confirm the hypothesis: gaits evolved with HyperNEAT and then transferred to reality were the highest performing observed to date on the QuadraTot platform. 
%That said, gaits still transferred poorly, revealing that substantial further performance gains are likely via future work that combines a simulator, HyperNEAT, and techniques that reduce the simulator-to-reality gap. 
%More broadly, this work supports previous work that evolving with a simulator and then transferring to reality is a more powerful means of achieving results than evolving directly in hardware. 
%Our work also confirms that generative encodings such as HyperNEAT are a powerful way to evolve gaits and, more generally, automatically exploit regularities in challenging engineering domains.

