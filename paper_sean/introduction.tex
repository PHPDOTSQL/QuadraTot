% !TEX root =main.tex
\figgp{quadratotWhiteBooties}{.45}{quadratot_simulator}{.45}{The QuadraTot robot platform on which gaits were evolved. \textbf{Left: } The physical robot, which is composed of 3-D printable and off-the-shelf components.
\textbf{Right: } The robot in the simulator. Figure adapted from from Glette et al. \cite{glette}.}  

Robots that can move around on legs can operate in a much wider range of environments than their wheeled counterparts. A downside is that designing gaits for legged robots is a difficult and time-consuming process for human engineers~\cite{strom1999legged,wettergreen1992gait}, and must be repeated every time a new robot is created or the design of an existing robot is changed~\cite{hornby2005autonomous}. 
Because of these difficulties in hand-designing gaits, scientists have investigated automatically producing gaits with machine learning or evolutionary algorithms, an approach that often produces better gaits than those created by human engineers~\cite{valsalam:mii,kohl:stone,hornby2005autonomous,hornby2003generative,yos:clune}. It has often been found that both machine-created and human-created gaits perform better if they are \emph{regular}, in that they have coordinated movements, such as left-right symmetry or front-back symmetry~\cite{valsalam:mii,clune2011performance,clune2009evolving,clune2009sensitivity}.
However, in most previous works, the experimenter has had to manually decide the regularities of the evolved gaits \cite{valsalam:mii,tellez,beer,raibert}. 
Such manual intervention is time consuming, requires expert knowledge, and adds constraints that may hurt performance. 

Previous work has shown that the Hypercube-based NeuroEvolution of Augmenting Topologies (HyperNEAT) generative encoding \cite{stanley2009hypercube} can automatically generate a variety of regular gaits and that it outperforms direct encoding controls on this task \cite{clune2009evolving,clune2011performance}.
However, that work only verified these claims in simulation. 
A follow-up paper by Yosinski et al. evolved gaits with HyperNEAT directly in hardware on the QuadraTot robot platform and found that HyperNEAT's gaits outperformed manually designed parameterized learning algorithms, but still did not produce impressive, natural gaits \cite{yos:clune}. 
A further follow-up study built a simulator for the QuadraTot robot platform to test whether the inclusion of a simulator would improve results and found that it did:  when gaits were evolved in this simulator with a simple direct encoding and then transferred to the QuadraTot robot, the resulting gaits were faster than the gaits produced by evolving gaits with HyperNEAT directly on the robot \cite{glette}. 
The simulator helped because it afforded much larger population sizes and more generations than were possible when evolving directly in hardware. Specifically, the simulator allowed 333 times as many evaluations per run as HyperNEAT had in hardware (60000 vs. 180)~\cite{yos:clune,glette}. 

The work with the simulator (Glette et al.~\cite{glette}) evolved gaits with a simple encoding that was manually constrained to produce specific regularities. The success of that work raises the question of whether the performance gains were due to the use of a simulator, or the use of a simple, manually-constrained encoding. We hypothesized that HyperNEAT, which has been previously shown to automatically discover complex regularities to produce high-performing gaits~\cite{clune2011performance,yos:clune}, would outperform the simpler encoding from Glette et al.~\cite{glette} if HyperNEAT is combined with a simulator. 
Here we test that hypothesis by evolving gaits with HyperNEAT in the same simulator from Glette et al. and then transferring those gaits to the QuadraTot robot.  Our experiments confirm the hypothesis: gaits evolved with HyperNEAT and then transferred to reality were the highest performing observed to date on the QuadraTot robot platform. 



%The results supports previous work that evolving with a simulator and then transferring to reality is a more powerful means of achieving results than evolving directly in hardware. %JMC: Can you add a few cites here, hopefully from our current cite list? I think some of the JBM papers show this. %JMC: Right now I have this commented out. No need to add the cites if we leave it commented out.  

%Our work also confirms that generative encodings such as HyperNEAT are a powerful way to evolve gaits and, more generally, automatically exploit regularities in challenging engineering domains.
