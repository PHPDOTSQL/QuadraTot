% by Suchan Sean Lee
% on 7/29/12

\documentclass{llncs}
\usepackage{graphicx}
\usepackage{ownstyles}
\usepackage{textcomp}
\graphicspath{{../figures/}}
%
\newcommand{\degree}{\ensuremath{^\circ}}
%
%
\begin{document}

%% BEGINNING of TITLE Section
%
%\title{Evolving Quadruped Gaits for Physical Robots with a Bio-Inspired Generative Encoding Using a Simulator}
\title{Evolving gaits for physical robots with the HyperNEAT generative encoding: the benefits of simulation}

\author{Authors taken out for double-blind review}
%\author{Suchan Lee\inst{1} \and Jason Yosinski\inst{1} \and Kyrre Glette\inst{2} \and Hod Lipson\inst{1} \and Jeff Clune\inst{1} }
%\institute{Cornell University, USA
\institute{Anonymous}

%\and
%University of Oslo, Norway \\
%\email{\{sl746,jy495,hod.lipson,jeffclune\}@cornell.edu, kyrrehg@ifi.uio.no}
%}
\maketitle
%
%% END of TITLE section
%
%
%% BEGINNING of ABSTRACT section
%
\begin{abstract}
% A brief summary of the report. This is not an introduction -
% it should be complete. Include 1-2 sentences for each of the items
% below, up to and including conclusions.

This paper presents an array of approaches to optimizing a quadrupedal
gait for forward speed.  We implement, test, and compare different
learning strategies including uniform and Gaussian random hill
climbing, policy gradient reinforcement learning,
Nelder-Mead simplex, new predictive methods based on linear
and support vector regression, and an evolved neural network
(HyperNEAT).  We compare results to a baseline random
search method.  Many of the methods resulted in walks significantly
faster than previously hand-tuned gaits.

%Because the fastest learned walk
%was not significantly faster than the fastest randomly generated walk,
%we conjecture that the motion representation for the robot is more
%integral to forward speed than the learning algorithm.

\edit{Write the abstract}

\end{abstract}
%
%% END of ABSTRACT section
%
%%BEGINNING of  paper section... replace the inputs with the actual texts later to make one, big LaTeX file

%% Pic of QuadraTot
%\figgy{robot_whitebg}{.6}{The QuadraTot robotic platform, on which numerous gait evolution papers are based. Our method of evolving gaits in simulation with the HyperNEAT generative encoding and transferring them to the robot produced faster gaits than all previously published techniques. There are 9 degrees of freedom: 2 per leg, and a single `hip joint' that connects  the two symmetrical halves of the robot.}
%%

\section{Introduction}
\section{Introduction}

\figp{quadratotWhiteBooties}{.75}{Example figure...}

Various learning algorithms have been tested to be effective for
legged robots. Their studies tested the viability of applying RL
cite{yosinski2011evolvingmethod to legged robot gaits learning. Competitive performance has
been verified on methods such as HyperNEAT \cite{yosinski2011evolving-robot-gaits} and
GA \cite{chernova2004an-evolutionary-approach-to-gait} and other methods \cite{hornby2005autonomous-evolution-of-dynamic} \cite{zykov2004evolving-dynamic-gaits} \cite{tellez2006evolving-the-walking-behaviour} \cite{valsalam2008modular-neuroevolution-for-multilegged}. Despite the excellent performance of these algorithms, a major
part of achieving good performance relies in tuning the parameters for
the evolutionary algorithms \cite{kormushev2011bipedal-walking-energy}. Here we present a
different way for learning gaits using RL (Hill climbing). In our
experiment, it turned out this method achieved competitive performance
with Problem Definition:The gait-learning problem is defined to find a
gait that maximizes some specific metric. Mathematically, we define a
gait as a function that specifies a vector of commanded motor
positions for a robot over time. Gaits without feedback --- also
called open-loop gaits --- can be defined as

\be\vec{x} = g(t)\ee

According to the definition, open-loop gaits are deterministic. One
particular gait should behave exactly the same when it is run from
trial to trial. However, the actual robot motion and fitness measured
will vary due to the noisiness of the real world physics.In our
trials, one gait pattern was generated and sent to the robot and
executed in an open loop manner, as defined. In particular, in this
paper, we chose to compare open-loop gaits generated specifically by
the RL\_POWER and HyperNEAT. The particular metric used will be
described later.

%
%
\section{Methods}

% !TEX root =main.tex
\subsubsection{Robot Hardware:}
%\figgp{quadratotWhiteBooties}{.45}{quadratot_simulator}{.45}{captioncaption}

We performed the experiments on the QuadraTot quadrupedal robot platform~(\figref{quadratotWhiteBooties})~\cite{yos:clune}.
It has 9 degrees of freedom: two joints per leg and one joint that allows rotation along the robot's midline. %we def want a picture of the cool robot. I propose putting the picture of the robot on the left and the picture of the simulated robot on the right of the same figure. That also does a nice job of highlighting the theme of the paper (robot+sim) in one figure. Try to get this figure to show up as early as possible in the paper (e.g top of the 2nd page). 
The QuadraTot hardware designs and the software for this project are open source\footnote{A parts list, hardware CAD files, software (including the simulator), and gait videos are available at http://creativemachines.cornell.edu/evolved-quadruped-gaits}, and all hardware components are either off-the-shelf or 3D printed parts. %These traits enable other labs to perform research using the same platform, which eases comparison of different gait-learning algorithms. 
To date there are results on the platform for nine different learning algorithms from three previous publications~\cite{yos:clune,glette,haocheng}. 


The joints are powered by Robotis Dynamixel servos; five AX-18A servos for the inner joint of each leg and the single midline joint, and four AX-12A servos for the the outer joint of each leg, which requires less power and can thus have less expensive motors. The servos were sent new positions at 40Hz via the Pydynamixel library. Each servo has a built-in safety mechanism that shuts itself off to prevent damage if the servo's current, range, temperature, or torque is too high. During evolution, this safety mechanism frequently activated, and did so inconsistently, adding significant noise to the evaluation process. As pointed out in a previous study~\cite{yos:clune}, gaits generated during evolution on QuadraTot are highly variable and produce many shutdowns because they force the servos to exert too much torque.
To prevent collisions between different pieces of the robot's body, we limited the allowable range of movement for the inner, outer, and hip joints to [-85\degree, +60\degree], [-113\degree, +39\degree], and [-28\degree, +28\degree], respectively. We also implemented the Smart Cropping System from Shen et al. 2012~\cite{haocheng}, which prevents combinations of joint positions for the inner and outer joint of each leg that generate extreme amounts of torque. A final method of reducing torque was to reduce the weight of the robot. Yosinski et al. 2011 had the small Linux computer that performed all computation on the robot, but we removed it and sent commands to the robot via a cable. We tracked the robot's position using an infrared LED observed by a Wiimote.


% !TEX root =main.tex
The simulator\footnote{Available for Linux and Windows at http://quadratot.yosinski.com} was the same one used in Glette et al. \cite{glette}. C++ code for the the Nvidia PhysX physics engine approximately describes the QuadraTot, including the mass and size of the robot components and its degrees of freedom. In the simulator, each individual joint range was limit as described above, but Smart Cropping was not included because it hindered performance by limiting the types of gaits evolution could explore in early generations. 

%\figp{simulator_outline}{.6}{Physical representation of the QuadraTot in simulation. It captures the important parts of the robot, such as the number and types of the joints, mass, and the rough shape and lengths. The simulator was written by Kyrre Glette in C++ using NVIDIA PhysX software library and visualized using OpenGL. Figure taken from Glette et al. \cite{glette}.}



HyperNEAT is a neuroevolution method which evolves artificial neuron networks (ANNs) using evolutionary algorithms.
It indirectly encodes ANNs using compositional pattern producing networks (CPPNs) which represent genomes encoding ANN phenotypes \cite{stanley1}, \cite{stanley2}. 
HyperNEAT has shown that it could evolve large neural networks that represent the networks and features present in brains such as repetitions and regularities \cite{stanley3}. 
This encoding has previously been applied for gait-learning \cite{yos:clune} and for evolving 2-D and 3-D objects \cite{clune:lipson}. 
It has been shown that objects evolved using HyperNEAT look complex, natural and feature characteristics such as symmetries and repetition of themes which are often found in nature. % figure from jeff paper


The motivation for CPPNs is that complex patterns can be produced by determining attributes of their phenotypic components as a function of their geometric location. 
This is based on the belief that in nature, cells in a body differentiates and specializes in different things with respect to their location. 
A cell in natural organisms cannot determine its location in space by itself and requires chemical gradients are used to signal its location \cite{carroll}. 
In contrast, \emph{in silico}, cells can be given their geometric coordinates and CPPNs use this to their advantage. 


Each CPPN is a directed graph in which every node is a mathematical function, such as sine or Gaussian. % figure!
Moreover, a CPPN genome allows functions to be made of other functions, allowing coordinate frames to be combined and hierarchies to develop.
For example, a sine function could develop a repeating theme which is passed onto a Gaussian function to create a repeating series of symmetrical motifs. 
These functions allow evolution to exploit various properties such as symmetry (e.g. Gaussian) and repetition (e.g. sine function) \cite{stanley1}. 
Because of evolution is able to exploit such properties, things like gaits are able to feature repetition and symmetry on its own without human intervention \cite{clune1}. 


There are links between nodes in CPPNs each of which has a weight used to multiply the signal in each link to determine the magnitude of effect from each node. 
In HyperNEAT, a CPPN genome takes in Cartesian coordinates (X,Y) of the source and the target nodes, and a constant bias value and outputs the weight between the input and hidden layer and a weight between the hidden layer and output (assuming that there is a hidden layer, as in the case for this study). 
All pairwise combinations of source and target nodes are passed into a CPPN to determine the weight of each ANN link. 


HyperNEAT is first neuroevolutionary algorithm that has shown capabilities in exploiting the geometry of the problem \cite{stanley1}, \cite{clune3}.
This is possible because the connection weights are a function of geometric positions of these nodes. 
If positions inputed to CPPNs represent aspects of the problem relevant to the solution, HyperNEAT could use that information to its advantage.  


HyperNEAT evolves the CPPNs according to the principles of the Neuroevolution of Augmenting Topologies (NEAT) algorithm \cite{stanley4}. 
The NEAT algorithm has three major components \cite{stanley4}. 
First, it starts with small genomes that encode simple networks and complexifies them via mutations and add nodes and links to the network which allows the alogithm to evolve the network topology and weight. 
Second, NEAT has a fitness-sharing system that preserves diversity in the system and allows for new innovations to be tuned by evolution before competing them against more adapted rivals. 
Third, the historical information stored in genes helps to perform crossover in a way that is effective, yet avoids the need for expensive topological analysis. A full explanation of NEAT can be found in Stanley and Miikkulainen \cite{stanley4}. 


In this study, the ANN configuration from evolving gaits with HyperNEAT in hardware \cite{yos:clune} was used. 
In this ANN configuration, the ANN has a fixed topology that consists of three 3 X 4 Cartesian grids of nodes for the input, hidden, and output layers. 
The inputs to the substrate were the angles requested in the prevous time step for each of the 9 joints of the robot and a sine and cosine wave to facilitate periodic motion. 
The outputs of the substrate at each time step were nine numbers (for each joint) in the range [-1, 1] which were scaled to the range [0, 1023], the allowable ranges for the servos. 
As in Yosinski et al. \cite{yos:clune}, we requested four times as many commanded positions from HyperNEAT ANNs and averaged over four commands at a time in order to reduce the possibility of switching from extreme negative to extreme positive numbers at a very high frequency. 

%%% FIGURES %%%

% \begin{figure}
% \begin{center}
% \vspace{1cm}
% \epsfig{file=CPPN.eps, width=8cm}
% \caption [ ]{Compositional Pattern Producing Networks (CPPNs). CPPN  genomes use mathematical functions to generate regularities such as symmetry and repetitions. CPPN genomes allow functions to be made of other functions, allowing multiple regularity motifs to be present. Figure taken from Stanley \cite{stanley2}.}
% \end{center}
% \end{figure}
\figp{hyperneat_bug_example}{.6}{Compositional Pattern Producing Networks (CPPNs). CPPN  genomes use mathematical functions to generate regularities such as symmetry and repetitions. CPPN genomes allow functions to be made of other functions, allowing multiple regularity motifs to be present. Figure taken from Stanley \cite{stanley2}.}

% \begin{figure}
% \begin{center}
% \vspace{1cm}
% \epsfig{file=hyperneat.eps, width=8cm}
% \caption [ ]{Producing ANNs from CPPNs. In HyperNEAT, weights are specified as a function of the Cartesian coordinates for source and input nodes and a constant bias for each connection between the source and input. All pairwise combinations of source and target node coordinates are iteratively passed into CPPNs to determine the weight of each ANN link. Figure taken from Clune et al. \cite{clune2}}.
% \end{center}
% \end{figure}
\figp{hyperneatExplanation}{.6}{Producing ANNs from CPPNs. In HyperNEAT, weights are specified as a function of the Cartesian coordinates for source and input nodes and a constant bias for each connection between the source and input. All pairwise combinations of source and target node coordinates are iteratively passed into CPPNs to determine the weight of each ANN link. Figure taken from Clune et al. \cite{clune2}}


\section{Methods}

\subsection{Policy Representation by Splines}
The simplest model with back-compatibility is geometric
splines. \editbox{Too fast... add a sentence or two on what the methods are trying to do, and what back-compatibility is.} For a given model f(x) with K knots, we can preserve the
exact shape of the generated curve while adding extra knots to the
original spline. Say, if we put one additional knot between every two
consecutive knots of the original spline, we end up with a 2K - 1
knots and a spline that has the same shape as the original one. In
order to do this, we need to define an algorithm for evolving the
parameterization from $K$ to $L$ knots ($L > K$), which is formulated as 
Algorithm 1 in \cite{kormushev2011bipedal-walking-energy} \edit{reproduce that algorithm here?}.  Without loss of generality, the policy parameters are
normalized into $[0, 1]$, and appropriately scaled/shifted as necessary
later upon use.

\subsection{Parameterized Gaits by RL PoWER}

Here we used an RL approach to change the complexity of the policy
representation dynamically while the trial is running. In
earlier studies on reducing energy
consumption for bipedal robots \citep{kormushev2011bipedal-walking-energy}, a mechanism that can
evolve the policy parameterization was used. The method starts from a
very simple parameterization and gradually increases its
representational capability. The method was tested to be capable of generating
an adaptive policy parameterization that can accommodate increasingly
more complex policies. Presented in the studies of \cite{kormushev2011bipedal-walking-energy}, the policy
generated by this approach can reach the global optimum at a fast
rate when applied to the energy reduction problem. Another property found about this method is its chance of converging to a suboptimal solution is reduced, because in the lower-dimensional representation this effect is less exhibited.

\figp{powerSplinesExample}{.6}{An example for an evolving policy parameterization based on
spline representation of the policy. The set of spline knots is the policy
parameterization. The spline knots are the actual policy parameter
values. This original parameterization starts from 4 knots and grows up to 32 knots}

\cite{kober2009learning-motor-primitives} proposed a RL algorithm
called Policy learning by Weighting Exploration with the
Returns (RL PoWER), which is based on Expectation-Maximization algorithm
(EM). The proposed technique for evolving the policy parameterization
is a combination with this EM-based RL algorithm, named PoWER \cite{kober2009learning-motor-primitives}. \edit{Check the prev sentence... I don't get it} The reason for using this is its relatively fewer parameters that need tuning. We
evolved the policy parameterization only on those past trials ranked
the highest by the importance sampling technique used by the PoWER
algorithm. The intuition behind is that highly ranked
parameterizations have more potential to evolve even better in the
future. Besides, evolving all the parameterizations increases the exploring space. Since our experiment
is done on a physical robot, exploring all the variations of every
parameterization is not practical. Future work may incorporate simulations into the studies, as illustrated in \cite{bongard2006resilient-machines-through}.

For the experiment, we set the splines to have 3 knots for each servo, and there are 8
servos in total. The servo in the hip is not used in
our experiment. Previous work has verified that quadruped gaits
perform better when they are coordinated \citep{clune2009evolving-coordinated-quadruped, clune2011on-the-performance-of-indirect-encoding, valsalam2008modular-neuroevolution-for-multilegged}. For each spline, we calculate its corresponding parameterized gait for one unit time cycle \edit{how long is this?} and then apply the same
pattern to every cycle throughout the 12 seconds of one
trial. Specifically, each spline (a set of 3 knots) is interpreted to its corresponding servo positions as shown in \eqnref{rl} and \tabref{params}.

\begin{table}[h]
\begin{center}
\begin{tabular}{|c|c|c|}
\hline
Parameters        &                           &       \\
in $\vec{\theta}$ & Description               & Range \\
\hline
\hline
$f(s1,s2,s3)$        & Spline function           & [0,1] \\  %subject to change
\hline
$R$          & Position multiplier                & [256, 768] \\
\hline
\end{tabular}
\caption{The \emph{RL PoWER} motion model parameters.}
\tablabel{parameters}
\tablabel{params}
\end{center}
\end{table}


\be
\vec{g}(t) =
\left[ {\begin{array}{c@{ }c@{ }c@{ }l@{ }l}
R \cdot f(s1, s2, s3) & \ \          & \             & \            & + C \\ % 1
0                              & \             & \             & \            & + C_C \\ % 8
\end{array} } \right]
\eqnlabel{rl}
\ee

\editbox{Add a new equation showing all servos (all nine) with their
  actual values of the parameters $R$ and $C$ plugged in.}


%
%
\section{Results and Discussion}
\section{Results and Discussion}
\seclabel{results}

% Outline:
% - exploration of parameter space
% - parameterized optimization results, discussion
% - HyperNEAT results, discussion



\subsection{Exploration of Parameterized Gait Space}

Before optimizing the chosen family of parameterized gaits
(\secref{motionModel}) with learning methods, we performed an
experiment to explore the five dimensions of the SineModel5 parameter
space. Specifically, we selected a random parameter vector that
resulted in some motion, but not an exceptional gait. We then varied
each of the five parameters individually and measured performance,
repeating each measurement twice to get a rough estimate of the
measurement noise at each point.  The results of this exploration,
shown in \figref{explore_dim_1}, reveal that some dimensions ($\amp$,
$\tau$, $m_F$) are fairly smooth and exhibit global structure across
the allowable parameter range, while others ($m_O$, $m_R$) exhibit
more complex behavior.  In addition, it gives a rough indication that
measurement noise is often significant and is more likely to be large
for gaits that move more.  Of course, this is only a slice in each
dimension through a single point, and slices through a different point
could reveal different behavior.  The common point at the
intersection of all slices is shown as a red triangle in
each plot of \figref{explore_dim_1}.


%\acmFig{explore_dim_1}{1}{Fitness mean and standard deviation
%  vs. dimension 1.  The circle is a common point in
%  \figref{explore_dim_1} through \figref{explore_dim_5}}
%\acmFig{explore_dim_2}{1}{Fitness mean and standard deviation
%  vs. dimension 2.  The circle is a common point in
%  \figref{explore_dim_1} through \figref{explore_dim_5}}
%\acmFig{explore_dim_3}{1}{Fitness mean and standard deviation
%  vs. dimension 3.  The circle is a common point in
%  \figref{explore_dim_1} through \figref{explore_dim_5}}
%\acmFig{explore_dim_4}{1}{Fitness mean and standard deviation
%  vs. dimension 4.  The circle is a common point in
%  \figref{explore_dim_1} through \figref{explore_dim_5}}
%\acmFig{explore_dim_5}{1}{Fitness mean and standard deviation
%  vs. dimension 5.  The circle is a common point in
%  \figref{explore_dim_1} through \figref{explore_dim_5}}

\acmFiggggg{explore_dim_1}{explore_dim_2}{explore_dim_3}{explore_dim_4}{explore_dim_5}{1}{.53}{Fitness
  mean and standard deviation when each parameter dimension is varied
  independently.  The red triangle in each plot represents the same point in the 5-dimensional parameter space.}



\subsection{Learning Methods for Parameterized Gaits}

% Outline
% - beat hand gait
% - no learning strategy outperformed another by that much
% - real world noise yay

The results for the parameterized gaits are shown in
\figref{std_error} and \tabref{results}.  As a group, these strategies
produced results equal to or worse not significantly better than the randomly-generated
gaits, however, all methods managed to explore the space enough to
significantly improve on the previous hand coded gait in at least one
of the three runs.  No single strategy consistently outperformed the
others: for the first trial Linear Regression produced the fastest
gait at 27.58 body lengths/minute, for the second a random gait
actually won with 17.26 body lengths/minute, and for the third trial
the Nelder-Mead simplex method attained the fastest gait at 14.83 body
lengths/minute.



% octave:15> fit = [6.04 17.26 4.90;  11.37 9.44 2.69;  3.10 13.59 13.40;  0.68 14.69 3.60;  8.51 13.62 14.83;  27.58 12.51 1.95;  24.27  36.44  27.07];
% octave:16> mean(fit,2)
% ans =
% 
%     9.4000
%     7.8333
%    10.0300
%     6.3233
%    12.3200
%    14.0133
%    29.2600
% 
% octave:17> std(fit,0,2)
% ans =
% 
%     6.8308
%     4.5576
%     6.0023
%     7.3914
%     3.3546
%    12.8810
%     6.3737


%%%%%% OLD TABLE
% \begin{table*}
% \begin{center}
% \begin{tabular}{|r|c|c|c||c|}
% \hline
%                                          & A       & B      & C      &  Average \\
% \hline                                                               
% \hline                                                               
% Previous hand-coded gait                 & --      & --     & --     &  5.16 \\
% \hline                                                                 
% Random search                            & 6.04    & 17.26  & 4.90   &  9.40 \\
% \hline                                                                 
% Uniform Random Hill Climbing             & 11.37   & 9.44   & 2.69   &  7.83 \\
% \hline                                                                 
% Gaussian Random Hill Climbing            & 3.10    & 13.59  & 13.40  &  10.03 \\
% \hline                                                                 
% Policy Gradient Descent                  & 0.68    & 14.69  & 3.60   &  6.32 \\
% \hline                                                                 
% Nelder-Mead simplex                      & 8.51    & 13.62  & 14.83  &  12.32 \\
% \hline                                                                 
% Linear Regression                        & 27.58   & 12.51  & 1.95   &  14.01 \\
% \hline                                                                
% Evolutionary Neural Network (HyperNEAT)  & 24.27     & 36.44    & 27.07   & 29.26  \\
% \hline
% \end{tabular}
% \caption{The best gaits found for each starting vector and algorithm,
%   in body lengths per minute.}
% \tablabel{results}
% \end{center}
% \end{table*}

%%%%%%% NEW TABLE
\begin{table}
\begin{center}
\begin{tabular}{|r|c|c|c||c|}
\hline
                                         & Average & Std. Dev. \\
\hline                                    
\hline                                    
Previous hand-coded gait                 & 5.16   &   --     \\
\hline
Random search                            & 9.40   &   6.83   \\
\hline
Uniform Random Hill Climbing             & 7.83   &   4.56   \\
\hline
Gaussian Random Hill Climbing            & 10.03  &   6.00   \\
\hline
Policy Gradient Descent                  & 6.32   &   7.39   \\
\hline
Nelder-Mead simplex                      & 12.32  &   3.35   \\
\hline
Linear Regression                        & 14.01  &  12.88   \\
\hline
Evolved Neural Network              &        &          \\
(HyperNEAT)                              & 29.26  &   6.37   \\
\hline
\end{tabular}
\caption{The best gaits found for each starting vector and algorithm,
  in body lengths per minute.}
\tablabel{results}
\end{center}
\end{table}



\acmFig{std_error}{1}{Average results ($\pm $ SE) for each
  of the parameterized learning methods.  Linear regression found the
  fastest overall gait and had the highest average, followed by
  Nelder-Mead simplex. Many methods were beaten by the random
  strategy.}




\subsection{HyperNEAT Gaits}

The results for the gaits evolved by HyperNEAT are shown
in \figref{hnResults} and \tabref{results}.  Overall the HyperNEAT gaits
were the fastest by far, beating all the parameterized models when
comparing either average or best gaits.  We believe
that this is because HyperNEAT was allowed to explore a
much richer space of motions, but did so while still utilizing
symmetries when advantageous.
\figref{neat_110115_211410_00000_002_filt_zoom} shows a typical
HyperNEAT gait that had high fitness.  The pattern of motion is both
complex (containing multiple frequencies and repeating patterns across time)
and regular, in that patterns of multiple motors are coordinated. 

The evaluation of the gaits produced by HyperNEAT was more noisy than for the parameterized gaits, which made learning difficult. For example, we tested an example HyperNEAT generation-champion gait 11 times and found that its mean performance was 26 body lengths per minute ($\pm$ 13 SD), but it had a max of 38 and a min of 3.  Many effective HyperNEAT gaits were not preserved across generations because if performance in one trial was poor, the genome was unlikely to be selected for. We believe the slope of the HyperNEAT learning curve would be steeper if the noise in the evaluations could be reduced.

\acmFig{hnResults}{1}{Average fitness ($\pm $ SE) of
  the highest performing individual in the population for each generation of HyperNEAT runs. The fitness of many high-performing HyperNEAT gaits were halved if the gait overly stressed the motors (see text), meaning that HyperNEAT's true performance without this penalty would be much higher.}

%\acmFig{neat_110115_211410_00000_002_filt}{1}{Caption here...???}

\acmFig{neat_110115_211410_00000_002_filt_zoom}{1}{Example of one
  particular high-performance gait produced by HyperNEAT showing
  commands for each of nine motors.  Note the complexity of the motion
  pattern. Such patterns were not possible with the parameterized
  SineModel5, nor would they likely be contrived by a human designing
  a different low-dimensional parameterized motion model.}

%
%
\section{Conclusion}
\section{Conclusion and Future Work}
\seclabel{conclusion}

We have presented an array of approaches for optimizing a quadrupedal
gaits for speed.  We implemented and tested six learning
strategies for parameterized gaits and compared them to gaits produced by neural networks
evolved with the HyperNEAT generative encoding.

All methods resulted in an improvement over the robot's previous
hand-coded gait.  Building a model of gait
performance with linear regression to predict promising directions for further exploration
worked well, producing a gait of 27.5 body
lengths/minute.  The Nelder-Mead simplex method performed nearly as well, likely due to its robustness to noise.  The other parameterized
methods did not outperform random search.  One reason the randomly-generated SineModel5 gaits performed so well could be because the gait representation was biased towards effective, regular gaits, making the highly exploratory random strategy more effective than more exploitative learning algorithms. 

HyperNEAT produced higher-performing gaits than all of the parameterized
methods. Its best-performing gait traveled 45.7 body lengths per minute, which is nearly 9 times the speed of the hand-coded gait.  This could be because HyperNEAT tends to generate coordinated gaits~\citep{clune2011performance, clune2009evolving}, allowing it to
take advantage of the symmetries of the problem. HyperNEAT can also explore a much larger space of possibilities than the
more restrictive 5-dimensional parameterized space.  HyperNEAT gaits tended to produce more complex sequences of motor commands, with different frequencies and degrees of coordination, whereas the
parameterized gaits were restricted to scaling single-frequency sine waves and could only produce certain types of motor regularities. 

%
%
%% BEGINNING of BIBLIOGRAPHY

%\begin{thebibliography}{1}
%\bibitem {bongard:lipson}
Bongard, J., Zykov, V., and Lipson, H.:
Resilient Machines Through Continuous Self-Modeling.
Science. 314(5802):1118-1121 (2006)
%
\bibitem {valsalam:mii}
Valsalam, V. and Miikkulainen, R.:
Modular Neuroevolution for Multilegged Locomotion.
In: Proceedings of the Genetic and Evolutionary Computation Conference, pp. 265--272 (2008)
%
\bibitem {kohl:stone}
Kohl, N. and Stone, P.:
Machine Learning for Fast Quadrupedal Locomotion.
In: The Ninetheenth National Conference on Artificial Intelligence (AAAI 2004), pp. 611--616. San Francisco (2004)
%
\bibitem {clune1}
Clune, J., Beckmann, B., Ofria, C., and Pennock, R.:
Evolving Coordinated Quadruped Gaits with the HyperNEAT Generative Encoding.
In: Proceedings of the IEEE Congress on Evolutionary Computation, pp. 2764--2771 (2009)
%
\bibitem {clune3}
Clune, J., Ofria, C., and Pennock, R.:
The Sensitivity of HyperNEAT to Different Geometric Representations of a Problem.
In: Proceedings of the Genetic and Evolutionary Commuptation Conference, pp. 675--682 (2009)
%
\bibitem {stanley1}
Stanley, K. O., D'Ambrosio, D. B., and Gauci, J.:
A Hypercube-Based Encoding for Evolving Large-Scale Neural Networks.
Artificial Life. 15(2), 185-212 (2009)
%
\bibitem {clune2}
Clune, J., Stanley, K. O., Pennock, R., and Ofria, C.:
On the Performance of Indirect Encoding Across the Continuum of Regularity.
IEEE Transactions on Evolutionary Computation. 15, 346-367 (2011)
%
\bibitem {yos:clune}
Yosinski, J., Clune, J., Hidalgo, D., Nguyen, S., Zagal, J. C.:
Evolving Robot Gaits in Hardware: the Hyperneat Generative Encoding vs. Parameter Optimization.
In: Proceedings of the 20th European Conference on Artificial Life, pp. 11--18 (2011)
%
\bibitem {glette}
Glette, K., Kalus, G., Zagal, J. C., and Torresen, J.:
Evolution of Locomotion in a Simulated Quadruped Robot and Transferral to Reality.
In: Proceedings of the Seventeenth International Symposium on Aritifical Life and Robotics, (2012)
%
\bibitem {haocheng}
Shen, H., Yosinski, J., Kormushev, P., Caldwell, D. G., Lipson, H.:
Learning Fast Quadruped Robot Gaits with the RL PoWER Spline Parameterization.
In: AIMSA 2012 Workshop on Advances in Robot Learning and Human-Robot Interaction, (2012)
%
\bibitem {robotis}
Robotis:
User's Manual, Dynamixel AX-12.
URL: http://www.trossenrobotics.com/images/productdownloads/AX-12(English).pdf
%
\bibitem {clune:lipson}
Clune, J., Lipson, H.:
Evolving Three-Dimensional Objects with a Generative Encoding Inspired by Developmental Biology.
In: Proceedings of the European Conference on Artificial Life, pp. 144--148 (2011)
%
\bibitem {stanley2}
Stanley, K. O.:
Compositional Pattern Producing Networks: A Novel Abstraction of Development.
Genetic Programming and Evolvable Matter. 8(2), 131--152 (2007)
%
\bibitem {stanley3}
Stanley, K. O.:
Exploiting Regularity Without Development.
In: Proceedings of the AAAI Fall Symposium on Developmental Systems, (2006)
%
\bibitem {stanley4}
Stanley, K. O., Miikkulainen, R.:
Evolving Neural Networks Through Augmenting Topologies.
Evolutionary Computation, 10(2):99-127 (2002)
%
\bibitem {jakobi}
Jakobi, N., Husbands, P., and Harvey, I.:
Noise and the Reality Gap: The Use of Simulation in Evolutionary Robotics.
In: Advances in Artificial Life: Proceedigns of the 3rd European Conference on Artificial Life, pp. 704--720, Springer-Verlang, (1995)
%%
\bibitem {koos1}
Koos, S., Mouret, J.-B., Doncieux, S.:
Crossing the Reality Gap in Evolutionary Robotics by Promoting Transferable Controllers.
In: Proceedings of the 12th Annual Conference on Genetic and Evoutionary Computation, pp. 119--126 (2010)
%
\bibitem {carroll}
Carroll, S.:
Endless Forms Most Beautiful: The New Science of Evo Devo and the Making of the Animal Kingdom.
Norton, New York (2005)
%
\bibitem {bongard}
Bongard, J.:
Synthesizing Physically-Realistic Environmental Models from Robot Explorations.
In: Proceedings of the 9th European Confrence on Advances in Artificial Life, pp. 806--815 (2007)
%
\bibitem {zagal}
Zagal, J., Ruiz-del Solar, J., and Vallejos, P.:
Back to Reality: Crossing the Reality Gap in Evolutionary Robotics.
In: Proceedings of IAV 2004, the 5th IFAC Symposium on Intelligent Autonomous Vehicles, (2004)
%
\bibitem {koos2}
Koos, S., Mouret, J.-B., Doncieux, S.:
The Transferability Approach: Crossing the Reality Gap in Evolutionary Robotics.
IEEE Transactions on Evolutionary Computation, (2011)

%\end{thebibliograpy}
\footnotesize
\bibliographystyle{splncs}
\bibliography{references}
%
%
\end{document}
