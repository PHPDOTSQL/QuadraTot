HyperNEAT is a neuroevolution method which evolves artificial neuron networks (ANNs) using evolutionary algorithms.
It indirectly encodes ANNs using compositional pattern producing networks (CPPNs) which represent genomes encoding ANN phenotypes \cite{stanley1}, \cite{stanley2}. 
HyperNEAT has shown that it could evolve large neural networks that represent the networks and features present in brains such as repetitions and regularities \cite{stanley3}. 
This encoding has previously been applied for gait-learning \cite{yos:clune} and for evolving 2-D and 3-D objects \cite{clune:lipson}. 
It has been shown that objects evolved using HyperNEAT look complex, natural and feature characteristics such as symmetries and repetition of themes which are often found in nature. % figure from jeff paper


The motivation for CPPNs is that complex patterns can be produced by determining attributes of their phenotypic components as a function of their geometric location. 
This is based on the belief that in nature, cells in a body differentiates and specializes in different things with respect to their location. 
A cell in natural organisms cannot determine its location in space by itself and requires chemical gradients are used to signal its location \cite{carroll}. 
In contrast, \emph{in silico}, cells can be given their geometric coordinates and CPPNs use this to their advantage. 


Each CPPN is a directed graph in which every node is a mathematical function, such as sine or Gaussian. % figure!
Moreover, a CPPN genome allows functions to be made of other functions, allowing coordinate frames to be combined and hierarchies to develop.
For example, a sine function could develop a repeating theme which is passed onto a Gaussian function to create a repeating series of symmetrical motifs. 
These functions allow evolution to exploit various properties such as symmetry (e.g. Gaussian) and repetition (e.g. sine function) \cite{stanley1}. 
Because of evolution is able to exploit such properties, things like gaits are able to feature repetition and symmetry on its own without human intervention \cite{clune1}. 


There are links between nodes in CPPNs each of which has a weight used to multiply the signal in each link to determine the magnitude of effect from each node. 
In HyperNEAT, a CPPN genome takes in Cartesian coordinates (X,Y) of the source and the target nodes, and a constant bias value and outputs the weight between the input and hidden layer and a weight between the hidden layer and output (assuming that there is a hidden layer, as in the case for this study). 
All pairwise combinations of source and target nodes are passed into a CPPN to determine the weight of each ANN link. 


HyperNEAT is first neuroevolutionary algorithm that has shown capabilities in exploiting the geometry of the problem \cite{stanley1}, \cite{clune3}.
This is possible because the connection weights are a function of geometric positions of these nodes. 
If positions inputed to CPPNs represent aspects of the problem relevant to the solution, HyperNEAT could use that information to its advantage.  


HyperNEAT evolves the CPPNs according to the principles of the Neuroevolution of Augmenting Topologies (NEAT) algorithm \cite{stanley4}. 
The NEAT algorithm has three major components \cite{stanley4}. 
First, it starts with small genomes that encode simple networks and complexifies them via mutations and add nodes and links to the network which allows the alogithm to evolve the network topology and weight. 
Second, NEAT has a fitness-sharing system that preserves diversity in the system and allows for new innovations to be tuned by evolution before competing them against more adapted rivals. 
Third, the historical information stored in genes helps to perform crossover in a way that is effective, yet avoids the need for expensive topological analysis. A full explanation of NEAT can be found in Stanley and Miikkulainen \cite{stanley4}. 


In this study, the ANN configuration from evolving gaits with HyperNEAT in hardware \cite{yos:clune} was used. 
In this ANN configuration, the ANN has a fixed topology that consists of three 3 X 4 Cartesian grids of nodes for the input, hidden, and output layers. 
The inputs to the substrate were the angles requested in the prevous time step for each of the 9 joints of the robot and a sine and cosine wave to facilitate periodic motion. 
The outputs of the substrate at each time step were nine numbers (for each joint) in the range [-1, 1] which were scaled to the range [0, 1023], the allowable ranges for the servos. 
As in Yosinski et al. \cite{yos:clune}, we requested four times as many commanded positions from HyperNEAT ANNs and averaged over four commands at a time in order to reduce the possibility of switching from extreme negative to extreme positive numbers at a very high frequency. 

%%% FIGURES %%%

% \begin{figure}
% \begin{center}
% \vspace{1cm}
% \epsfig{file=CPPN.eps, width=8cm}
% \caption [ ]{Compositional Pattern Producing Networks (CPPNs). CPPN  genomes use mathematical functions to generate regularities such as symmetry and repetitions. CPPN genomes allow functions to be made of other functions, allowing multiple regularity motifs to be present. Figure taken from Stanley \cite{stanley2}.}
% \end{center}
% \end{figure}
\figp{hyperneat_bug_example}{.6}{Compositional Pattern Producing Networks (CPPNs). CPPN  genomes use mathematical functions to generate regularities such as symmetry and repetitions. CPPN genomes allow functions to be made of other functions, allowing multiple regularity motifs to be present. Figure taken from Stanley \cite{stanley2}.}

% \begin{figure}
% \begin{center}
% \vspace{1cm}
% \epsfig{file=hyperneat.eps, width=8cm}
% \caption [ ]{Producing ANNs from CPPNs. In HyperNEAT, weights are specified as a function of the Cartesian coordinates for source and input nodes and a constant bias for each connection between the source and input. All pairwise combinations of source and target node coordinates are iteratively passed into CPPNs to determine the weight of each ANN link. Figure taken from Clune et al. \cite{clune2}}.
% \end{center}
% \end{figure}
\figp{hyperneatExplanation}{.6}{Producing ANNs from CPPNs. In HyperNEAT, weights are specified as a function of the Cartesian coordinates for source and input nodes and a constant bias for each connection between the source and input. All pairwise combinations of source and target node coordinates are iteratively passed into CPPNs to determine the weight of each ANN link. Figure taken from Clune et al. \cite{clune2}}
