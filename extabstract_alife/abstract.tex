Legged robots are uniquely privileged over their wheeled counterparts
in their ability to access rugged terrain, yet manually designing gaits for
such robots can be a tedious process. Thus, we
investigate methods by which gaits may be learned autonomously by a
robot, either using physical real world trials or virtual trials in a
simulator. Each of these two methods has its advantages and
disadvantages. Physical trials are slow and often require expensive
human supervision. Simulated trials can be run quickly, in parallel,
and without supervision, which in practice allows orders of magnitude
increases in the number of trials per experiment. However, the
unavoidable shortcoming of simulated trials is that the simulation
inevitably fails to match reality exactly, resulting in what is often
called the simulation-reality gap. This gap frequently results in
learning methods overfitting to the idiosyncrasies of the simulated
world and producing gaits that do not transfer well to the physical
robot.  Thus, the chief benefit of learning gaits directly on a
physical system is the ability to avoid this gap. Herein we summarize
recent gait learning results using both types of fitness evaluation
--- physical and simulated --- on the open-source QuadraTot robot.
