\section{Abstract}
% A brief summary of the report. This is not an introduction -
% it should be complete. Include 1-2 sentences for each of the items
% below, up to and including conclusions.
This paper presents several approaches to optimizing a quadrupedal
trot gait for forward speed. Given a parameterized walk designed for
a quadruped robot designed and printed in the Cornell Computational
Synthesis Lab, we implement, test, and compare different learning strategies, 
including uniform and Gaussian random hill climbing, a form of policy 
gradient reinforcement learning\cite{1}, Simplex (Nelder-Mead), 
linear regression, SVM regression, and an evolutionary neural 
network (HyperNEAT)\cite{2}. Because the fastest learned walk was not
significantly faster than the fastest randomly generated walk, we
conjecture that the motion representation for the robot is more
integral to forward speed than the learning algorithm.

Keywords: Learning Control, Walking Robots, Multi Legged Robots

\section{Introduction}
% Motivate and abstractly describe the problem you are addressing and
% how you are addressing it. What is the problem? Why is it important?
% What is your basic approach? A short discussion of how it fits into
% related work in the area is also desirable. Summarize the basic results
% and conclusions that you will present.
We are using machine learning to design a gait for a quadruped
robot. The robot has an on-board computer, drivers for the motors, and
will have camera feedback. The code so far consists of the robot
class, optimization class, parameterized model, sine model, and camera
feedback.  Both the hardware and software are progressing on schedule.
Some promising gaits have been generated, and we anticipate a final
evaluation of the robot’s gait in November.
  
\section{Problem definition}
% Precisely define the problem you are addressing (i.e. formally specify
% the inputs and outputs). Elaborate on why this is an interesting and
% important problem.
We are testing several different learning methods to design a gait for 
a quadraped robot from the Cornell Computational Synthesis Lab. The metric 
for evaluation of the designed gait is speed. A comparison and evaluation 
of the many different methods available for optimizing the gait of legged 
robots will be useful for future work on this challenging multidimensional 
control problem.

\section{Method}
% Describe in reasonable detail the algorithm you are using to address this 
% problem. A pseudo-code description of the algorithm you are using is 
% frequently useful. If it makes sense for your project, trace through a 
% concrete example, showing how your algorithm processes this example. The 
% example should be complex enough to illustrate all of the important aspects 
% of the problem but simple enough to be easily understood. If possible, 
% an intuitively meaningful example is better than one with meaningless 
% symbols.
We implemented and tested 8 different learning strategies:
\begin{itemize}
\item \emph{Uniform random hill climbing}: Creates a random neighbor by
  randomly choosing one parameter to adjust, and changing it completely 
  randomly in UniformStrategy(). Then evaluates this neighbor by running 
  the robot with the newly chosen parameters. If this neighbor results in a
  longer distance walked then the previous best neighbor, we save this
  neighbor as the new best neighbor. We then repeat the whole process
  using the best neighbor as a base.
\item \emph{Gaussian random hill climbing}: Creates a random neighbor by
  changing each parameter randomly based on a normal distribution. Then
  evaluates this neighbor similarly to uniform random hill climbing.
\item \emph{N-dimensional policy gradient descent}: Estimates the policy
  gradient by evaluating \emph{t} randomly generated policies
  near an initial parameter vector. Then computes the average score for
  each parameter and adjusts the base policy according to the estimated
  gradient.
\item \emph{Random}: Randomly generates policies in the range of the motion
representation.
\item \emph{Simplex (Nelder-Mead) Method}
\item \emph{Linear regression}: To initialize, chooses five random points
and trains on them using least-squares. Then in a loop, takes a fixed-size
step in the direction of the gradient, and retrains on all points so far.
\item \emph{SVM regression}
\item \emph{Evolutionary Neural Network (HYPERNEAT)\cite{2}}
\end{itemize}

\section{Related work}
% Briefly explain who else worked on related problems in the past and what
% methods they used. Explain if you are using similar methods, or if your
% approach is different and if so - how (either is ok).

\section{System Architecture and Implementation}
% Describe how you implemented your system and how you structured it. 
% This should give an overview of the system, not a detailed 
% documentation of the code. The documentation of the code is part of 
% the code you hand in. You might want to comment on high-level design 
% decisions that you made. Also explain how you obtained your
% data and any pre-processing you did to it.

The quadraped robot has an on-board computer running Linux. The lower
level drivers are in C and we are implementing the system in
Python. Feedback about distance travelled is provided via a Wii remote. 

\begin{itemize}
\item \code{Robot} class: Class wrapper for commanding
  motion of the robot.  The \code{Robot} class takes care of the robot
  initialization, communication with the servos, and timing of the
  runs.  In addition, it prevents the servos from ever being commanded
  to a point outside their normal range (0 - 1023) as well as beyond
  points where limbs would collide with parts of the robot body.  The
  main class function, \code{run}, accepts a motion model (any
  function that takes a time argument and outputs a 9 dimensional
  position) and will run the robot using this motion model, including,
  if desired, smooth interpolation over time for the beginning and end
  of the run.

\item \code{Strategy} and \code{RunManager} class: The user has a choice between seven
  different learning strategies: uniform random hill climbing, Gaussian random
  hill climbing, N-dimensional policy gradient descent, random, 
  simplex, linear regression/prediction, and SVM regression/prediction.

\item \code{Parameterized Motion Model} abstract base class: Several functions exist, but none are within a class framework.

\item \code{SineModel} classes: Commands motors to positions
  based on a sine wave, creating a periodic pattern. Parameters are:
  amplitude, wavelength, scale inner vs outer motors, scale left vs right
  motors, scale back vs front motors. Currently a function, not a class.

\item Other derived motion model classes: Several versions of a SineModel
  exist, each with different parameters, though solely as functions.

\item \code{WiiTrackClient} and \code{WiiTrackServer} classes
 and hardware: A Wii remote tracks the location of the robot through an infrared LED mounted on top of the robot. We used CWiid library which is able to interface with the remote via bluetooth to determine the location of the infrared LED. This information is accessed through some functions we wrote and passed to the program. The program gets the robot's position at the beginning of each run and then again at the end. The program then calculates the net change in position and uses that as the distance walked.

\end{itemize}

\section{Evaluation status}
% A list of the major steps in your planned evaluation of the system
% and the status of each. For some projects it may be easier to merge
% this list with the component/module status list above of your
% system. Include also who in the team is responsible for completing
% this module.

\begin{itemize}
\item After running our program and finding a good gait, we will then
evaluate our system by the distance walked in a certain amount
of time. This evaluation metric depend on the result of
running our program for many iterations. We have done 50 runs of 
3 different inital parameter vectors for random hill climbing, 
gaussian random hill climbing, and policy gradient descent.
The evaluation for our particular project is an ongoing
process, but we anticipate a final evaluation taking place
early-to-mid November. The evaluation process will be completed by
Jason, Sarah, and Diana.

\item We also coded \code{explore\_dimensions} to collect data for dimension
varying plots. We intend to do further explore this by running
\code{explore\_dimension} around a parameter that produces a medium-good gait.
\end{itemize}



\section{Project schedule timeline}
% Tell us if you are meeting your project timeline or you are
% exceeding it or falling behind. If you are falling behind, explain
% how you will catch up or adjust the goals.

We are meeting our project timeline. We have developed two new algorithms,
gaussian hill climbing and policy gradient descent and tested all three of
our major algorithms on the robot. We have also quantified our current results. 
In the next month we hope to develop one more learning strategy, machine
learning of an objective function.

\section{Schedule}

Below is a rough schedule of fixed external deadlines and anticipated
team milestones.

\definecolor{doneColor}{rgb}{0.0,0.7,0.0}
\newcommand{\done}[1]{{\color{doneColor}\sout{#1}}}
%\newcommand{\done}[1]{#1}

\begin{center}
\begin{tabular}{|l||p{2.5in}|p{1.5in}|}
\hline
                        & {\bf Milestones} & {\bf Deadlines}
\\ \hline

Week 1 (9/13-9/19)    & \done{Read papers, get lab access, talk to relevant other researchers} & 9/17 Final proposals due
\\ \hline

Week 2 (9/20-9/26)    & \done{Continue reading, get robot to move} &
\\ \hline

Week 3 (9/27-10/03)   & \done{Implement parametrized gait and determine proposed coding schedule for more advanced algorithms in time for Code Review \#1.} &
\\ \hline

Week 4 (10/04-10/10)  & \done{Begin main algorithm dev/testing effort} & 10/5 Code Review \#1
\\ \hline

Week 5 (10/11-10/17)  & \done{Algorithm dev/testing} &
\\ \hline

Week 6 (10/18-10/24)  & \done{Algorithm dev/testing} &
\\ \hline

Week 7 (10/25-10/31)  & \done{Algorithm dev/testing, quantify/solidify current results for Code Review \#2} &
\\ \hline

Week 8 (11/1-11/7)    & \done{Finish collecting results, begin writing} & 11/2 Code Review \#2
\\ \hline

Week 9 (11/8-11/14)   & Algorithm dev/testing &
\\ \hline

Week 10 (11/15-11/21) & Finish collecting results, writing, get final demo ready &
\\ \hline

Week 11 (11/22-11/28) & Finish collecting results, writing, get final demo ready  &
\\ \hline

Week 12 (11/29-11/30) & Final demo  & 11/30 Final presentation
\\ \hline
\end{tabular}
\end{center}



\section{Activity log (Optional)}

% an informal log your efforts. If you’ve been keeping notes on your
% progress, feel free to turn these in as well. These might include:
% (1) notes on what aspect of the project you worked on and when, (2)
% what was accomplished in each “session”, (3) design questions, (4)
% implementation questions, (5) descriptions of experiments and
% results, (6) problems that arose, (7) anything else that you think
% is relevant to the project.

\begin{itemize}
\item \emph{9/13-9/19:} Set up. Set up trac, worked on pre-proposal
  and presentation, got lab access, met with Jim Torresen, contacted
  Juan Cristobal Zagal, read papers, got the robot to walk using
  previously coded program. Questions -- How do we move the robot? How
  do we command the motors?
\item \emph{9/20-9/26:} Moving the robot. Installed Linux on the
  robot, figured out how to command the motors, coded a program to
  make the robot stand up, coded a simple motion pattern. Questions --
  For initial program, what parameters and model should be used? How
  do we limit the search space? What algorithm should we start with?
  How will the robot get information about how far it moved?
\item \emph{9/27-10/3:} Initial program. Chose several parameters and
  created a simple parameterized sine model, implemented a random hill
  climbing algorithm to optimize the parameters, ran some trial
  iterations, began setting up the cameras for feedback. Questions --
  Did the initial program work? What parameters are important? Are
  additional parameters necessary? What algorithm should we implement
  for the final program?
\item \emph{10/4-10/10:} Plan for final implementation. Discussed
  possible algorithms to use and new parameters to add. Worked on
  report for code review. Discussed feedback from code review.  Questions -- Advantages for gradient
  descent? Should we start with a completely random parameter vector
  or should we start with one that results in motion and let the
  program optimize it? How does the camera system work?
\item \emph{10/11-10/17:} Discussed how we were going to proceed with the project and set up a more detailed schedule. Got the cameras working and calibrated them. Researched the 3 available streaming protocols. Questions -- Which streaming protocol would work best? Which improvements should we start with?
\item \emph{10/18-10/24:} Created a variation on the random hill climbing algorithm to choose the next neighbor based on a Gaussian distribution. Tried to work with two of the streaming protocols for the camera system, but were unsuccessful. Decided to use a Wii remote for positional information instead. Wrote the integration code for the Wii remote. Refactored our code to separate the different algorithms. Began planning the algorithm for a modified gradient descent.
Questions -- What values should we use in the gradient descent algorithm? How much should the each parameter be changed? How many neighbors should be sampled before choosing the next state? How can we compare the results of our different algorithms? How repeatable is the distance traveled for a given parameter vector?
\item \emph{10/25-11/2}: Created a modified gradient descent algorithm. Ran 150 trials of the three algorithms. Each algorithm started with a specific parameter vector and ran for 50 iterations. Created and ran a program to get a feel for the overall look of the search space. Ran some trials to see how reproducible the distance traveled was. Prepared for code review 2. Questions -- What does the search space look like? Are there a lot of spikes, or is it fairly smooth? Based on the results of the algorithms, what conclusions can we draw?
\end{itemize}



\section{Proposal Modifications}
% Let us know if you are deviating from your original proposal.

We are not deviating from our original proposal.



\section{Appendix}
% Describe the code files you uploaded into CMS

A brief description of the code uploaded to the CMS follows:

\begin{itemize}
\item \code{optimize.py}: Determines a strategy to try and runs the robot
  with that strategy.
\item \code{RunManager.py}: Deals with all the details of running the robot,
  including choosing an initial parameter, running the robot multiple times, 
  tracking distance walked, and writing to the log file. Also includes the
  \code{explore\_dimensions} method.
\item \code{Strategy.py}: Contains all the different possible strategies, which
  will be passed as objects in \code{optimize.py}. 
\item \code{Robot.py}: Implements the \code{Robot} class, described in
  \secref{implement}.
\item \code{SineModel.py}: Implements a sine based motion model,
  described in \secref{implement}.
\item \code{Motion.py}: Motion helper functions.
\item \code{WiiTrackServer.py}: Broadcasts the position of the infrared LED>
\item \code{WiiTrackClient.py}: Connects to the WiiTrackServer to get the current position information.
\end{itemize}
