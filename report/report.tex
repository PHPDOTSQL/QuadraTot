\section{Introduction}

% Motivate and abstractly describe the problem you are addressing and
% how you are addressing it. What is the problem? Why is it important?
% What is your basic approach? A short discussion of how it fits into
% % related work in the area is also desirable. Summarize the basic
% results and conclusions that you will present.

Applications of walking robots often call for the ability to walk as
quickly, efficiently, or with as little power as possible.  Often
these optimizations are done manually by an expert who designs and
tweaks a gait specifically for a given objective.  Other groups have
used learning methods to generate gaits optimized for some metric.
Approaches differ in their starting assumptions, some essentially
tweaking the parameters of a hand-tuned model \cite{chernova}, others
exploring a reasonably compact parameter space \cite{kohl}, and still
others beginning with few assumptions besides periodicity
\cite{zykov}.

We aimed to strike a middle ground between these approaches.  Our
motion generator did not rely on hand-tweaked gaits, but it did use
parameterized gaits which, by their nature, impose some assumptions on
the answers produced.  We then used machine learning to design gaits
for a quadruped robot with these models.  This paper presents a
comparison of the different learning methods implemented.  Most
methods created walks that are several times faster than the original
hand-tuned gait.  We invite readers with short attention spans to view
a video of some of our results online here:

\url{http://www.youtube.com/watch?v=ODoiOj9DdGg}

%The code so far consists of the robot class, optimization class,
%parameterized model, sine model, and camera feedback.



\section{Problem definition}
% Precisely define the problem you are addressing (i.e. formally specify
% the inputs and outputs). Elaborate on why this is an interesting and
% important problem.

We are testing several different learning methods to design a
parametrized gait for a quadruped robot from the Cornell Computational
Synthesis Lab.

The output each of the learning algorithms is a function of time,
$f(t)$, that outputs a vector of commanded motor positions.  This
function is generated using a parametrized motion model, described in
\secref{implement}.

The robot executes these commands and measures its change in location
using the tracking system described in \secref{implement}.  The input
to the learning algorithms is this measured displacement, which the
algorithms attempt to maximize. This displacement is measured for each
gait over a constant length run, usually 12 seconds.

A comparison and evaluation of the many different methods available
for optimizing the gait of legged robots will be useful for future
work on this challenging multidimensional control problem.



\section{Method}
\seclabel{method}

% Describe in reasonable detail the algorithm you are using to address
% this problem. A pseudo-code description of the algorithm you are
% using is frequently useful. If it makes sense for your project,
% trace through a concrete example, showing how your algorithm
% processes this example. The example should be complex enough to
% illustrate all of the important aspects of the problem but simple
% enough to be easily understood. If possible, an intuitively
% meaningful example is better than one with meaningless symbols.

We use several parameterized motion models that command motors to
positions based on a sine wave, creating a periodic pattern.  While we
investigated several models, for the bulk of our experiments, we used
a model whose five parameters are: amplitude, wavelength, scale inner
vs outer motors, scale left vs right motors, scale back vs front
motors. Each strategy below attempts to choose the best possible
parameters for this motion model.  

We implemented and tested 8 different learning strategies.  All
strategies except for the HyperNEAT method\cite{clune} were
constrained to pick parameters from within predetermined ranges.

\begin{itemize}

\item \emph{Random}: This method randomly generates parameter vectors
  in the allowable range. This strategy is used only as baseline.

\item \emph{Uniform random hill climbing}: This method begins by
  selecting a single random parameter vector.  Subsequent iterations
  generate a neighbor by randomly choosing one parameter to adjust and
  replacing it with a new value chosen with uniform probability in the
  allowable range for that parameter. The neighbor is evaluated by
  running the robot with the newly chosen parameters. If this neighbor
  results in a longer distance walked than the previous best gait, it
  is saved as the new best gait. The process is then repeated, always
  starting with the best gait.

\item \emph{Gaussian random hill climbing}: This method works
  similarly to Uniform random hill climbing, except neighbors are
  generated by adding random Gaussian noise to the current best gait.
  This results in all parameters being changed at once, but the
  resulting vector is always fairly close to the previous best gait.
  We used independently selected noise in each dimension, scaled such
  that the standard deviation of the noise was 5\% of the range of
  that dimension.

\item \emph{N-dimensional policy gradient descent}: As opposed to the
  previous methods, this method explicitly estimates the gradient for
  the objective function. It does this by first evaluating \emph{t}
  randomly generated parameter vectors near the initial vector, each
  dimension of these vectors being perturbed by either $-\epsilon$,
  $0$, or $\epsilon$. Then, for each dimension, it groups vectors into
  three groups: $-\epsilon$, $0$, and $\epsilon$.  The gradient along
  this dimension is then estimated as the average score for the
  $\epsilon$ group minus the average score for the $-\epsilon$
  group. Finally, the method creates a new vector by changing all
  parameters by a fixed-size step in the direction of the gradient.

\item \emph{Nelder-Mead simplex method}\cite{nm}: The Nelder-Mead
  simplex method creates an initial simplex with 6 vertices. The
  initial parameter vector is stored as the first vertex and the other
  five vertices are created by adding to one dimension at a time one
  tenth of the allowable range for that parameter. It then tests the
  fitness of each vertex and based on these fitnesses, it reflects the
  worst point over the centroid in an attempt to improve it.  However,
  to prevent cycles and becoming stuck in local minima, several other
  rules are used.  In general, the worst vertex is reflected over the
  centroid. If the reflected point is better than the second worst
  point and worse than the best point, then the reflected point
  replaces the worst. If the reflected point is better than the best
  point, the simplex is expanded in the direction of the reflected
  point. The better of the reflected and the expanded point replaces
  the worst point. If the reflected point is worse than the second
  worst point, then the simplex is contracted away from the reflected
  point. If the contracted point is better than the reflected point,
  the contracted point replaces the worst point. If the contracted
  point is worse than the reflected point, the entire simplex is
  shrunk \cite{nm}.

\item \emph{Linear regression}: To initialize, this method chooses and
  evaluates five random parameter vectors. It then fits a linear model
  from parameter vector to fitness. In a loop, the method chooses and
  evaluates a new parameter vector generated by taking a fixed-size
  step in the direction of the gradient for each parameter, and fits a
  new linear model to all vectors evaluated so far, choosing the model
  to minimize the sum of squared errors.

\item \emph{SVM regression}: Similarly to linear regression, this
  model starts with several random vectors, but this time they are
  chosen in a small neighborhood about some initial random vector.
  These vectors (generally 8) are evaluated, and a support vector
  regression model is fit to the observed fitnesses.  To choose the
  next vector for evaluation, we randomly generate some number
  (typically 100) of vectors in the neighborhood of the best observed
  gait, and select for evaluation the vector with the best predicted
  performance.  We suspected that if we always chose the best
  predicted point out of 100, we may end up progressing along a narrow
  subspace, prohibiting learning of the true local fitness function.
  Put another way, we would always choose exploitation of knowledge
  vs. exploration of the space.  To address this concern, we added a
  parameter dubbed \code{bumpBy} that added noise to the final
  selected point before it was evaluated.

  Such a method naturally has many tunable parameters, and we
  endeavored to select these parameters by tuning the method in
  simulation.  To estimate the performance of the algorithm, we ran it
  against a simulation with a known optimum.  The simulated function
  was in the same five dimensional parameter space, and simply
  returned a fitness determined as the height of a Gaussian with a
  random mean.  The width of the Gaussian in each dimension was 20\%
  of the range of each dimension, and the maximum value at the peak
  was 100.  \figref{svm_sim_results} shows the learning results on
  this simulated model using the ultimately selected SVM parameters.
  Interestingly, a non-zero value of \code{bumpBy} resulted in better
  learning than noise free (exploration free) learning.

  Ultimately, however, the version of SVM tuned for simulation still
  did not show competitive performance on the real robot.  We tried
  tuning some parameters on the real robot, but after some amount of
  tuning, the method still exhibited too little exploration and easily
  became stuck in local minima.

\item \emph{Evolutionary Neural Network (HyperNEAT)\cite{clune}}: Near
  the end of term, we hacked together an interface between HyperNEAT
  -- an implementation of a method for evolving neural networks -- and
  the robot, requiring a slightly modified strategy interface.
  Preliminary HyperNEAT runs were promising and resulted in several
  interesting gaits.  Unfortunately, the gaits generated by HyperNEAT
  also tended to stress the robot more than typical gaits had before,
  and the servos would often overheat and malfunction, requiring
  restarts.  We think these issues may be addressed by adding a small
  layer between the HyperNEAT strategy and the robot that disallows
  quickly shifting commanded positions, and we hope to be able to test
  these methods further once this filter is in place.

\end{itemize}

\figvarp{svm_sim_results}{.6}{Results for the SVM regression strategy
  in simulation.  This simulation was used to tune the SVM strategy's
  parameters before trying it on the physical robot.  The strategy
  quickly approaches the maximum simulated fitness of 100.}{}



\section{Related work}
% Briefly explain who else worked on related problems in the past and what
% methods they used. Explain if you are using similar methods, or if your
% approach is different and if so - how (either is ok).

Various maching learning techniques have proven to be useful in
finding control policies for a wide variety of robots. Kohl and
Stone\cite{kohl} presented a policy gradient reinforcement learning
approach for generating a fast walk on legged robots. We experimented
with this method to create a walk for our robot (Policy Gradient
Descent). Chernova and Velosa\cite{chernova} took an evolutionary
approach to this problem which we did not implement.  Zykov, Bongard,
and Lipson\cite{zykov} describe the evolution of dynamic gaits on a
physical robot requiring no prior assumptions about the locomotion
pattern beyond the fact that it should be rhythmic.


\section{System Architecture and Implementation}
\seclabel{implement}
% Describe how you implemented your system and how you structured it. 
% This should give an overview of the system, not a detailed 
% documentation of the code. The documentation of the code is part of 
% the code you hand in. You might want to comment on high-level design 
% decisions that you made. Also explain how you obtained your
% data and any pre-processing you did to it.

The quadruped robot has an on-board computer running Linux. The lower
level drivers are in C and the system is implemented in
Python. Feedback about distance travelled is provided via an infrared
LED mounted on the robot and a Wii remote fixed to the ceiling.

An overview of the code follows.

\begin{itemize}

\item \code{Robot} class: Class wrapper for commanding
  motion of the robot.  The \code{Robot} class takes care of the robot
  initialization, communication with the servos, and timing of the
  runs.  In addition, it prevents the servos from ever being commanded
  to a point outside their normal range (0 - 1023) as well as beyond
  points where limbs would collide with parts of the robot body.  The
  main class function, \code{run}, accepts a motion model (any
  function that takes a time argument and outputs a 9 dimensional
  position) and will run the robot using this motion model, including,
  if desired, smooth interpolation over time for the beginning and end
  of the run.

\item \code{RunManager} class: Deals with all the details of running
  the robot, including running the robot multiple times, tracking
  distance walked via a \code{WiiTrackClient} member object, and
  writing results to the log file. Also includes the
  \code{explore\_dimensions} method to generate plots by varying each
  parameter independently.

\item \code{Strategy} class: The user has a choice between eight
  different learning strategies: Random search, uniform random hill
  climbing, Gaussian random hill climbing, N-dimensional policy
  gradient descent, Nelder-Mead simplex, linear regression/prediction,
  SVM regression/prediction, and HyperNEAT evolution.  Each strategy
  must derive from the base \code{Strategy} class and must implement
  two methods: \code{getNext}, for getting the next parameter vector
  to try, and \code{updateResults}, for communicating results of a run
  back to the strategy.

\item \code{MotionModel} class: We implemented several motion models,
  the main being the \code{SineModel5} class.  All motion models take
  as input a parameter vector and produce as output a motion model,
  which is simply a function mapping from time to nine motor
  positions.  The \code{SineModel5} model commands positions based on
  a sine wave, creating a periodic pattern. The parameters are:
  amplitude, wavelength, a multiplier for the inner vs outer motors,
  multiplier for left vs right motors, and multiplier for back vs
  front motors.  Other similar models were tested, including a seven
  parameter model which allowed sine waves shifted in time, but these
  were not used as extensively in our experiments as
  \code{SineModel5}.

\item \code{WiiTrackClient} and \code{WiiTrackServer} classes and
  hardware: A Wii remote tracks the location of the robot through an
  infrared LED mounted on top of the robot. A \code{WiiTrackServer} is
  run on the robot and continuously tracks its position using the
  CWiid library\cite{cwiid} to interface with the remote via
  bluetooth.  The \code{RunManager} then makes a
  \code{WiiTrackClient}, which connects via socket to the tracking
  server and requests position updates periodically.
  \code{RunManager} currently gets the robot's position at the
  beginning of each run and then again at the end and uses this to
  calculate the net change in position.

\end{itemize}



\section{Experimental Evaluation}

\subsection{Methodology}
% What are the criteria you are using to evaluate your method? What
% specific hypotheses does your experiment test? Describe the 
% experimental methodology that you used. What are the dependent and 
% independent variables? For projects in machine learning, what is 
% the training/test data that was used, and why is it realistic or
% interesting? Exactly what performance data did you collect and how 
% are you presenting and analyzing it? Comparisons to competing methods 
% that address the same problem or to variations of your own algorithm 
% are particularly useful.

The metric for evaluation of the designed gait is speed. We stop each 
run after plateauing results (no improvement for one third of the 
policies seen so far). The standard length of a run designates that it
should be stopped after there is no improvement for one half of the policies
seen so far, but since all runs took place on the actual robot, without use
of a simulator, certain time limitations were imposed on the learning process.

We controlled our experiments from a computer that was connected via a
wireless ethernet to the robot. The robot collected data about
distance walked automatically on its own. If it walked outside of the
Wii remote's viewable area, it informed the user, so the only human
intervention required during an experiment was to move the robot back
inside the viewable area and to resume the run, which did not
interrupt the learning process or result in the loss of data.

We evaluated the efficacy of a set of parameters by sending these
parameters to the robot and instructing it to walk for a certain
length of time. The robot always began from the same position and
returned to the starting position at the end of the run in order to
measure true displacement without giving credit for ending in a leaned
position. More efficient parameters resulted in a faster gait, which
translated into a longer distance walked and a better score. After completing
an evaluation, the robot sent the resulting distance walked back to the
host computer and prepared itself for a new set of parameters to evaluate.

Each algorithm was run on 3 different initial parameter vectors on the
physical robot. We decided to evaluate all methods starting at the
same three vector in order to allow for the fair comparison of each
algorithm.  We evaluate each method based on the amount of improvement
seen from the initial parameter vectors, and on the fastest speeds
achieved during runs.

The resulting gaits from our algorithms quickly outperformed the original
hand-coded walk designed for this robot. The fastest walk, for example, was
4 times faster.

\subsection{Results}
% Present the quantitative results of your experiments. Graphical data
% presentation such as graphs and histograms are frequently better 
% than tables. Explain what are the basic differences revealed in the 
% data. If your method is stochastic or you analyze multiple datasets, 
% calculate mean and standard error for every quantitative
% performance result you show based on multiple runs.

We have done complete runs of 3 different initial parameter vectors
for random search, uniform random hill climbing, Gaussian random hill
climbing, policy gradient descent, Nelder-Mead simplex, and linear
regression.  We also evaluated the SVM regression and HyperNEAT
methods, but these methods were more experimental, and thus we do not
yet have the same volume of data for these runs.  We developed several
gaits that were about 4 times faster than the original hand-coded
gait.  Results are shown in \tabref{results_table}.



\begin{table}
\begin{center}
\begin{tabular}{|r|c|c|c||c|}
\hline
                                         & A       & B      & C      &  Average \\
\hline                                                               
\hline                                                               
Previous hand-coded gait                 & --      & --     & --     &  5.16 \\
\hline                                                                 
Random search                            & 6.04    & 17.26  & 4.90   &  9.40 \\
\hline                                                                 
Uniform Random Hill Climbing             & 11.37   & 9.44   & 2.69   &  7.83 \\
\hline                                                                 
Gaussian Random Hill Climbing            & 3.10    & 13.59  & 13.40  &  10.03 \\
\hline                                                                 
Policy Gradient Descent                  & 0.68    & 14.69  & 3.60   &  6.32 \\
\hline                                                                 
Nelder-Mead simplex                      & 8.51    & 13.62  & 14.83  &  12.32 \\
\hline                                                                 
Linear Regression                        & 27.58   & 12.51  & 1.95   &  14.01 \\
%\hline                                                                
%SVM Regression                           & ???     & ???    & ???   &   \\
%\hline                                                                
%Evolutionary Neural Network (HyperNEAT)  & ???     & ???    & ???   &   \\
\hline
\end{tabular}
\caption{The best gaits found for each starting vector and algorithm,
  in body lengths per minute.}
\tablabel{results_table}
\end{center}
\end{table}


\begin{itemize}

\item For vector A, shown in \figref{vectorA}, linear regression
  worked significantly better than the other algorithms, resulting in
  a gait (27.58 body lengths/minute) that walked over twice as fast as
  the next best gait -- uniform random hill climbing at 11.37 body
  lengths/minute -- and 5.3 times better than the previous hand-coded
  gait.

\item Vector B, depicted in \figref{vectorB}, resulted in similarly
  performing gaits with all the algorithms. The random method got
  lucky and produced the best gait (17.26 body lengths/minute) and
  uniform random hill climbing produced the worst (9.44 body
  lengths/minute). The remaining algorithms each produced a gait
  around 13 body lengths/minute.

\item For vector C, shown in \figref{vectorC}, simplex and Gaussian
  random hill climbing each produced a gait that substantially
  outperformed the other algorithms. Simplex resulted in a gait of
  nearly 15 body lengths/minute and Gaussian random hill climbing
  produced a gait of just over 13 body lengths/minute, whereas the
  other algorithms returned gaits of less than 5 body lengths/minute.

\end{itemize}

\figvarp{vectorA}{1}{Results for runs beginning with vector A.
  Linear regression worked significantly better than the other
  algorithms, resulting in a gait (27.58 body lengths/minute) that
  walked over twice as fast as the next best gait -- uniform random
  hill climbing at 11.37 body lengths/minute -- and 5.3 times better
  than the previous hand-coded gait.}{}

\figvarp{vectorB}{1}{Results for run starting with vector B.  All
  algorithms performed similarly.  The random method actually got
  lucky this time and produced the best gait (17.26 body
  lengths/minute) and uniform random hill climbing produced the worst
  (9.44 body lengths/minute). The remaining algorithms each produced a
  gait around 13 body lengths/minute.}{}

\figvarp{vectorC}{1}{Results for runs starting with vector C.
  Simplex and Gaussian random hill climbing each produced a gait that
  substantially outperformed the other algorithms, with around 15 body
  lengths/minute and 13 body lengths/minute, respectively, whereas the
  other algorithms returned gaits of less than 5 body
  lengths/minute.}{}

Based on the three trials, as shown in \figref{std_error}, linear
regression worked the best, followed by simplex. Gaussian random hill
climbing performed about as well as random, but it is unclear whether
Gaussian would continue to improve and become much better than random
search if allowed to run for more iterations. Uniform random hill
climbing and policy gradient descent performed similarly to each other
on average, but differed greatly on individual runs.

\figvarp{std_error}{1}{Average results for each method starting with
  parameter vectors A, B, and C.  Error bars are plotted showing the
  standard error.  Linear regression outperformed other methods,
  followed by simplex. Gaussian random hill climbing performed about
  as well as random, but it is unclear whether Gaussian would continue
  to improve and become much better than random search if allowed to
  run for more iterations. Uniform random hill climbing and policy
  gradient descent performed similarly to each other on average, but
  differed greatly on individual runs.}{}


In addition to running strategies to optimize parameterized gaits, we
also wanted to investigate the space of possible gaits.  To accomplish
this, we selected a parameter vector that resulted in motion, but not
an exceptional gait, and plotted performance along each dimension
individually.  In addition, we duplicated each measurement to be able
to estimate the measurement noise at each point.  Results for this
exploration are shown in \figref{explore_dim_1} through
\figref{explore_dim_5}.  As is shown in the figures, some dimensions
are smoother than others, and some measurements are fairly noisy.

\figvarp{explore_dim_1}{.65}{Fitness mean and standard deviation
  vs. dimension 1.  The circle is a common point in
  \figref{explore_dim_1} through \figref{explore_dim_5}}{}
\figvarp{explore_dim_2}{.65}{Fitness mean and standard deviation
  vs. dimension 2.  The circle is a common point in
  \figref{explore_dim_1} through \figref{explore_dim_5}}{}
\figvarp{explore_dim_3}{.65}{Fitness mean and standard deviation
  vs. dimension 3.  The circle is a common point in
  \figref{explore_dim_1} through \figref{explore_dim_5}}{}
\figvarp{explore_dim_4}{.65}{Fitness mean and standard deviation
  vs. dimension 4.  The circle is a common point in
  \figref{explore_dim_1} through \figref{explore_dim_5}}{}
\figvarp{explore_dim_5}{.65}{Fitness mean and standard deviation
  vs. dimension 5.  The circle is a common point in
  \figref{explore_dim_1} through \figref{explore_dim_5}}{}




\subsection{Discussion}
% Is your hypothesis supported? What conclusions do the results support
% about the strengths and weaknesses of your method compared to other 
% methods? How can the results be explained in terms of the underlying 
% properties of the algorithm and/or the data.

Because only three trials were tested with each algorithm and no
algorithm consistently outperformed the others, there is a large
standard error for each algorithm, as show in \figref{std_error}. For
this reason, it is unclear if any algorithm is superior to another in
this application. More trials with each algorithm would be necessary
to reach a definitive ranking of the performance of the algorithms.

However, each algorithm discovered at least one gait of over 10 body
lengths/minute, including random. For this reason, we speculate that
the motion model is at least as critical as the learning algorithm
used.



\section{Future work}
% What are the major shortcomings of your current method? For each 
% shortcoming, propose additions or enhancements that would help overcome it.

There are several directions in which we could continue our
work. First, the error margins for our runs were large, so reducing
them by running more trials would lead to more conclusive data. It
would also be insightful to run our algorithms on different motion
models.  We suspect that our choice of motion model influenced the
results greatly, as even random choices in the space produced gaits
that moved a significant fraction of the time (> 5\%).  It would be
interesting to see how learning methods would perform using a model
that included a much higher percentage of unproductive gaits. We also
intend to experiment further with SVM regression and evolutionary
algorithms/HyperNEAT.  Some parameter vectors resulted in the robot
turning, as opposed to it walking long distances. This could be an
interesting learning goal in future projects. To these ends, we
propose the following additions and enhancements:

\begin{itemize}
\item More runs and/or longer runs
\item Different motion representations
\item Better tuning of SVM regression
\item Evolutionary algorithms/HyperNEAT\cite{clune}
\item Learning how to turn
\end{itemize}



\section{Conclusion}
% Briefly summarize the important results and conclusions presented in 
% the paper. What are the most important points illustrated by your 
% work? How will your results improve future research and applications 
% in the area?


We have presented an array of approaches to optimizing a quadrupedal
gait for forward speed.  We have implemented and tested different
learning strategies, including uniform and Gaussian random hill
climbing, policy gradient reinforcement learning, Nelder-Mead simplex,
several new predictive methods based on linear and support vector
regression, and an evolved neural network (HyperNEAT).  We have also
compared these approaches to random search as a baseline. Many of the
methods resulted in walks significantly faster than previously
hand-tuned gaits.

Because only three trials were tested with each algorithm and no
algorithm consistently outperformed the others, there was a large
standard error for each method, as shown in \figref{std_error}. Thus
it was unclear if any algorithm was superior to another in this
application. More trials with each algorithm would be necessary to
reach a definitive ranking. Because each algorithm discovered at least one gait of
over 10 body lengths/minute, including random search, we also
conjectured that the motion representation for the robot is more
integral to forward speed than the learning algorithm.  How to learn
the motion representation, in addition to its parameters, remains an
open problem.



\section{Acknowledgments}
% List any people not on the team who helped you with your project: Your
% TA, other people you consulted with or had useful discussions (say in 
% a word or two what they did), people who proofread your report, and 
% any external code you used (libraries etc).
\begin{itemize}
\item Hod Lipson, Cornell Computational Synthesis Lab: adviser.
\item Jim T\o rreson, University of Oslo: adviser.
\item Juan Zagal, University of Chile: designed and printed robot and provided code for hand tuned gait.
\item Jeff Clune, CCSL Red Couch: collaborated on HyperNEAT implementation/testing.
\item Cooper Bills, Cornell University: assisted with Wii tracker development.
\item Anshumali Srivastava, Cornell University: Teaching Assistant.
\end{itemize}

\section{The Team}
% A table listing the names of the people who worked on the project, and who
% did what.

\begin{itemize}

\item Diana Hidalgo: 
\begin{itemize}
\item Simplex (Nelder-Mead)
\item \code{WiiTrackClient} and \code{WiiTrackServer} classes and hardware
\item Plotting and evaluating results
\item Investigation into using Optitrack IR system for tracking robot
\end{itemize}

\item Sarah Nguyen
\begin{itemize}
\item Random search
\item Uniform random hill climbing
\item Gaussian random hill climbing
\item N-dimensional policy gradient descent
\item Linear regression and prediction
\item \code{RunManager} and \code{Strategy} classes
\item Video editing
\end{itemize}

\item Jason Yosinski
\begin{itemize}
\item SVM regression and prediction
\item Evolutionary Neural Network (HyperNEAT)
\item \code{WiiTrackClient} and \code{WiiTrackServer} classes and hardware
\item \code{Robot}, \code{RunManager}, and \code{Strategy} classes
\item \code{Parameterized Motion Model} abstract base class
\item \code{SineModel} classes
\item \code{explore\_dimensions}
\item Plotting and evaluating results
\end{itemize}

\end{itemize}

\section{Appendix}
% Describe the code files you uploaded into CMS

A brief description of the code uploaded to the CMS follows:

\begin{itemize}
\item \code{optimize.py}: Main program, determines a strategy to try and runs the robot
  with that strategy.
\item \code{RunManager.py}: Deals with all the details of running the robot,
  including choosing an initial parameter, running the robot multiple times, 
  tracking distance walked, and writing to the log file. Also includes the
  \code{explore\_dimensions} method.
\item \code{Strategy.py}: Contains all the different possible strategies, which
  will be passed as objects in \code{optimize.py}. 
\item \code{Robot.py}: Implements the \code{Robot} class, described in
  \secref{implement}.
\item \code{SineModel.py}: Implements a sine based motion model,
  described in \secref{implement}.
\item \code{Motion.py}: Motion helper functions.
\item \code{WiiTrackServer.py}: Broadcasts the position of the infrared LED.
\item \code{WiiTrackClient.py}: Connects to the WiiTrackServer to get the current position information.
\end{itemize}

% LocalWords:  vertices
