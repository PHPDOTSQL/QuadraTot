\section{Executive Summary}

We are using machine learning to design a gait for a quadruped
robot. The robot has an on-board computer, drivers for the motors, and
will have camera feedback. The code so far consists of the robot
class, optimization class, parameterized model, sine model, and camera
feedback.  Both the hardware and software are progressing on schedule.
Some promising gaits have been generated, and we anticipate a final
evaluation of the robot’s gait in November.



\section{Proposal Summary}
% A short (one or two sentence) summary of the project
% goal/topic. This is just to remind us what you’re working on.

We are using machine learning to design a gait for a quadraped robot
from Hod Lipson’s lab. We anticipate using speed and number of
failures as metrics for evaluation of the designed gait.



\section{System Architecture Summary}
% A summary of the main components of the system, and how they
% interact with each other to achieve the goal. This is just to remind
% us what you are planning to develop.

The quadraped robot has an on-board computer running Linux. The lower
level drivers are in C and we are implementing the system in
Python. Feedback about distance travelled is currently input manually,
but we eventually plan to provide this via an Optitrack positioning
system.



\section{Implementation Status}
\seclabel{implement}
% A list of the major components of the system as well as the current
% status of each (e.g. designed, implemented, tested, finished, almost
% finished with coding). For the pieces not yet finished, indicate
% tentative completion dates. The schedule doesn’t have to be
% incredibly detailed, but it should include entries for all
% substantial modules/submodules of your system. Include also who in
% the team is responsible for completing this module.



\subsection*{\code{Robot} class}

\begin{itemize}
\item \emph{Team member:} Jason.
\item \emph{Current implementation:} Class wrapper for commanding
  motion of the robot.  The \code{Robot} class takes care of the robot
  initialization, communication with the servos, and timing of the
  runs.  In addition, it prevents the servos from ever being commanded
  to a point outside their normal range (0 - 1023) as well as beyond
  points where limbs would collide with parts of the robot body.  The
  main class function, \code{run}, accepts a motion model (any
  function that takes a time argument and outputs a 9 dimensional
  position) and will run the robot using this motion model, including,
  if desired, smooth interpolation over time for the beginning and end
  of the run.
\item \emph{Future improvements:} None planned, though some will
  probably arise.
\item \emph{Status:} 90\% complete.
\end{itemize}



\subsection*{\code{Optimization} class}

\begin{itemize}
\item \emph{Team member:} Sarah.
\item \emph{Current implementation:} Creates a random neighbor by
  randomly choosing one parameter to adjust, and changing either
  completely randomly in neighbor(), or slightly in
  slightNeighbor(). Then evaluates this neighbor by running the robot
  with the newly chosen parameters. If this neighbor results in a
  longer distance walked then the previous best neighbor, we save this
  neighbor as the new best neighbor. We then repeat the whole process
  using the best neighbor as a base.
\item \emph{Future improvements:} Gradient descent.
\item \emph{Status:} Initial algorithm coded. Improved algorithm
  slightNeighbor() coded. Gradient descent algorithm to be completed
  by Sarah.
\end{itemize}



\subsection*{\code{Parameterized Motion Model} abstract base class}

\begin{itemize}
\item \emph{Team member:} Jason.
\item \emph{Current implementation:} Several functions exist, but none are within a class framework.
\item \emph{Future improvements:} Make into class.
\item \emph{Status:} 40\% complete.
\end{itemize}



\subsection*{\code{SineModel} classes}

\begin{itemize}
\item \emph{Team member:} Jason.
\item \emph{Current implementation:} Commands motors to positions
  based on a sine wave, creating a periodic pattern. Parameters are:
  amplitude, wavelength, scale inner vs outer motors, multiply
  left-right, multiply back-front, shift front-back, shift right-left.
  Currently a function, not a class.
\item \emph{Future improvements:} Make class, add time offset
  parameters.  Continued tweaking of this class and other motion
  models will be a main goal of the next month.
\item \emph{Status:} 30\% complete.
\end{itemize}



\subsection*{Other derived motion model classes}

\begin{itemize}
\item \emph{Team member:} All.
\item \emph{Current implementation:} Several versions of a SineModel
  exist, each with different parameters, though solely as functions
  (not as derived classes of \code{ParameterizedMotionModel}).
\item \emph{Future improvements:} Implementing and tweaking current
  and new motion model classes will be our primary effort over the
  next month.
\item \emph{Status:} 20\% complete.
\end{itemize}



\subsection*{\code{CameraFeedback} class and hardware}

\begin{itemize}
\item \emph{Team member:} Diana.
\item \emph{Current implementation:} No software, still configuring
  camera hardware.
\item \emph{Future improvements:} Write python wrapper class, get
  positional information from the Optitrack camera system to the program
\item \emph{Status:} 20\% complete, goal to have cameras working by October 8.
\end{itemize}



\section{Evaluation status}
% A list of the major steps in your planned evaluation of the system
% and the status of each. For some projects it may be easier to merge
% this list with the component/module status list above of your
% system. Include also who in the team is responsible for completing
% this module.

After running our program and finding a good gait, we will then
evaluate our system in two ways: the time taken to walk a certain
distance and the number of failures (i.e. falling down, getting
stuck). Both of these evaluation metrics depend on the result of
running our program for many iterations. Neither have been
completed. The evaluation for our particular project is an ongoing
process, but we anticipate a final evaluation taking place
early-to-mid November. The evaluation process will be completed by
Jason, Sarah, and Diana.

\section{Project schedule timeline}
% Tell us if you are meeting your project timeline or you are
% exceeding it or falling behind. If you are falling behind, explain
% how you will catch up or adjust the goals.

We are meeting our project timeline and are, in fact, slightly ahead
of schedule. We have read the relevant papers, gotten lab access, and
talked with the appropriate people to learn the background information
necessary for working with the robot. We learned how to make the robot
move and have implemented an initial parameter optimization algorithm,
which has worked better than we anticipated.



\section{Schedule}

Below is a rough schedule of fixed external deadlines and anticipated
team milestones.

\definecolor{doneColor}{rgb}{0.0,0.7,0.0}
\newcommand{\done}[1]{{\color{doneColor}\sout{#1}}}
%\newcommand{\done}[1]{#1}

\begin{center}
\begin{tabular}{|l||p{2.5in}|p{1.5in}|}
\hline
                        & {\bf Milestones} & {\bf Deadlines}
\\ \hline

Week 1 (9/13-9/19)    & \done{Read papers, get lab access, talk to relevant other researchers} & 9/17 Final proposals due
\\ \hline

Week 2 (9/20-9/26)    & \done{Continue reading, get robot to move} &
\\ \hline

Week 3 (9/27-10/03)   & \done{Implement parametrized gait and determine proposed coding schedule for more advanced algorithms in time for Code Review \#1.} &
\\ \hline

Week 4 (10/04-10/10)  & \done{Begin main algorithm dev/testing effort} & 10/5 Code Review \#1
\\ \hline

Week 5 (10/11-10/17)  & Algorithm dev/testing &
\\ \hline

Week 6 (10/18-10/24)  & Algorithm dev/testing &
\\ \hline

Week 7 (10/25-10/31)  & Algorithm dev/testing, quantify/solidify current results for Code Review \#2 &
\\ \hline

Week 8 (11/1-11/7)    & Finish collecting results, begin writing & 11/2 Code Review \#2
\\ \hline

Week 9 (11/8-11/14)   & Finish collecting results, writing &
\\ \hline

Week 10 (11/15-11/21) & Finish collecting results, writing, get final demo ready &
\\ \hline

Week 11 (11/22-11/28) & Finish collecting results, writing, get final demo ready  &
\\ \hline

Week 12 (11/29-11/30) & Final demo  & 11/30 Final presentation
\\ \hline
\end{tabular}
\end{center}



\section{Activity log (Optional)}

% an informal log your efforts. If you’ve been keeping notes on your
% progress, feel free to turn these in as well. These might include:
% (1) notes on what aspect of the project you worked on and when, (2)
% what was accomplished in each “session”, (3) design questions, (4)
% implementation questions, (5) descriptions of experiments and
% results, (6) problems that arose, (7) anything else that you think
% is relevant to the project.

\begin{itemize}
\item \emph{9/13-9/19:} Set up. Set up trac, worked on pre-proposal
  and presentation, got lab access, met with Jim Torresen, contacted
  Juan Cristobal Zagal, read papers, got the robot to walk using
  previously coded program. Questions -- How do we move the robot? How
  do we command the motors?
\item \emph{9/20-9/26:} Moving the robot. Installed Linux on the
  robot, figured out how to command the motors, coded a program to
  make the robot stand up, coded a simple motion pattern. Questions --
  For initial program, what parameters and model should be used? How
  do we limit the search space? What algorithm should we start with?
  How will the robot get information about how far it moved?
\item \emph{9/27-10/3:} Initial program. Chose several parameters and
  created a simple parameterized sine model, implemented a random hill
  climbing algorithm to optimize the parameters, ran some trial
  iterations, began setting up the cameras for feedback. Questions --
  Did the initial program work? What parameters are important? Are
  additional parameters necessary? What algorithm should we implement
  for the final program?
\item \emph{10/4-10/5:} Plan for final implementation. Discussed
  possible algorithms to use and new parameters to add. Worked on
  report for code review \#1. Questions -- Advantages for gradient
  descent? Should we start with a completely random parameter vector
  or should we start with one that results in motion and let the
  program optimize it?
\end{itemize}



\section{Proposal Modifications}
% Let us know if you are deviating from your original proposal.

We are not deviating from our original proposal.



\section{Appendix}
% Describe the code files you uploaded into CMS

A brief description of the code uploaded to the CMS follows:

\begin{itemize}
\item \code{optimize.py}: Employs random hill-climbing to choose and
  evaluate the parameters of the robots’ motion, either by changing
  one parameter completely randomly, or changing one parameter
  slightly. Evaluates each neighbor using manually input distance
  walked.
\item \code{Robot.py}: Implements the \code{Robot} class, described in
  \secref{implement}.
\item \code{SineModel.py}: Implements a sine based motion model,
  described in \secref{implement}.
\item \code{Motion.py}: Motion helper functions.
\end{itemize}

