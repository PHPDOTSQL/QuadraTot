\subsection{HyperNEAT Gait Generation \& Learning}
\seclabel{hyperNeatMethod}
    
%HyperNEAT is an indirect encoding for evolving artificial neural
%networks (ANNs) that is inspired by the way natural organisms
%develop~\cite{stanley2009hypercube}. It evolves Compositional Pattern
%Producing Networks (CPPNs)~\cite{stanley2007compositional}, each of
%which is a genome that encodes an ANN
%phenotype~\cite{stanley2009hypercube}.
%Each CPPN is itself a directed
%graph, where the nodes in the graph are mathematical functions, such as sine or
%Gaussian. The nature of these functions can facilitate the evolution
%of properties such as symmetry (e.g.\ a Gaussian function) and repetition (e.g.\ a sine
%function)~\cite{stanley2009hypercube, stanley2007compositional}. The
%signal on each link in the CPPN is multiplied by that link's weight,
%which can magnify or diminish its effect.  
%A CPPN is queried once for each link in the ANN phenotype to determine
%that link's weight. The inputs to the CPPN are the Cartesian
%coordinates of both the source (e.g.\ $x = 2$, $y = 4$) and target
%(e.g.\ $x = 3$, $y = 5$) nodes of a link and a constant bias
%value. The CPPN takes these five values as inputs and produces two output values. The first output value
%determines the weight of the link between the associated input
%(source) and hidden layer (target) nodes, and the second output value
%determines the weight of the link between the associated hidden
%(source) and output (target) layer nodes. All pairwise combinations of
%source and target nodes are iteratively passed as inputs to a CPPN to
%determine the weight of each ANN link.

%\acmFig{hyperneatExplanation.png}{.8}{HyperNEAT produces ANNs from
%  CPPNs. ANN weights are specified as a function of the geometric
%  coordinates of the source node and the target node for each
%  connection. The coordinates of these nodes and a constant bias are
%  iteratively passed to the CPPN to determine each connection
%  weight. The CPPN has two output values, which specify the weights
%  for each connection layer as shown.}

%HyperNEAT is capable of exploiting the geometry of a problem because
%the link values between nodes in the final ANN phenotype are a
%function of the geometric positions of those nodes~\cite{stanley2009hypercube, clune2009sensitivity,
%  clune2011performance}. In the case of quadruped locomotion, this
%property has previously been shown to help HyperNEAT produce gaits with front-back, left-right,
%and four-way symmetries~\cite{clune2009evolving,
%  clune2011performance}.
%  
%The evolution of the population of CPPNs occurs according to the
%principles of the NeuroEvolution of Augmenting Topologies (NEAT)
%algorithm~\cite{stanley2002evolving}, which was originally designed to
%evolve ANNs. NEAT can be fruitfully applied to CPPNs because of their
%structural similarity to ANNs. For example, mutations can add a node,
%and thus a function, to a CPPN graph, or change its link weights. The
%NEAT algorithm is unique in three main
%ways~\cite{stanley2002evolving}. Initially, it starts with small
%genomes that encode simple networks and slowly complexifies them via
%mutations that add nodes and links to the network, enabling the
%algorithm to evolve the topology of an ANN in addition to its
%weights. Secondly, NEAT has a fitness-sharing mechanism that preserves
%diversity in the system and gives time for new innovations to be tuned
%by evolution before competing them against more adapted
%rivals. Finally, NEAT tracks historical information to perform
%intelligent crossover while avoiding the need for expensive
%topological analysis. A full explanation of NEAT can be found
%in~\cite{stanley2002evolving}.
  
The ANN configuration follows previous studies that evolved 
gaits with HyperNEAT in simulation~\cite{clune2009evolving}.
%, but was adapted to accommodate the physical
%robot used in this paper.
%Specifically, the ANNs consists of three $3 \times 4$
%Cartesian grids of nodes forming input, hidden, and output
%layers. Adjacent layers were allowed to be completely connected, meaning that there
%could be $(3 \times 4)^2= 288$ links in each ANN (although evolution can set weights to 0, functionally eliminating the connection).
The inputs to the ANN
substrate were the current \emph{commanded} angles of each of the 9 joints of the robot and a sine and cosine wave (to facilitate the
production of periodic behaviors). The sine and cosine waves had a period of about half a second.

%The outputs of the substrate at each time step were nine numbers in
%the range $[-1,1]$, which were scaled according to the allowable
%ranges for each of the nine motors and then commanded
%the positions for each motor.  Occasionally HyperNEAT would produce networks that
%exhibited rapid oscillatory behaviors, switching from extreme negative to extreme positive numbers each time step.  This resulted in motor
%commands to alternate extremes every 25ms (given the command rate of
%40Hz), which tended to damage and overheat the motors.  To ameliorate
%this problem, we requested four times as many commanded
%positions from HyperNEAT ANN's and averaged over four commands at a time to
%obtain the actual gait $g(t)$.  This solution worked well and did not restrict the expressiveness of HyperNEAT.

%\editbox{are we mentioning that we gave HyperNEAT the center joint
%  here or earlier?}

%\acmFig{SpiderANN.jpg}{.7}{ANN configuration for HyperNEAT Runs. The
%  first two columns of each row of the input layer receive information
%  about a single leg (the current angle of its two joints). The final
%  column provides a sine and cosine wave to enable periodic movements
%  and the angle of the center joint. Evolution determines the function
%  of the hidden-layer nodes. The nodes in the output layer specify new
%  joint angles for each respective joint. The unlabeled nodes in the
%  input and output layers are ignored.}\label{hyperneatLayout}

As with the parameterized methods, three runs of HyperNEAT
were performed. The population size for HyperNEAT was 9 and runs
lasted 20 generations. These numbers are small, but were necessarily
constrained given how much time it took to conduct evolution directly
on a real robot.
%The remaining parameters were identical to Clune et
%al.~\cite{clune2011performance}.

% Preliminary HyperNEAT runs were promising and resulted in several
% interesting gaits.

% Unfortunately, the gaits generated by HyperNEAT also tended to
% stress the robot more than typical gaits had before, and the servos
% would often overheat and malfunction, requiring restarts.  We think
% these issues may be addressed by adding a small layer between the
% HyperNEAT strategy and the robot that disallows quickly shifting
% commanded positions, and we hope to be able to test these methods
% further once this filter is in place.
