%\subsection{Related Work}

Various machine learning techniques have proved to be effective at
generating gaits for legged robots. Kohl and Stone presented a policy
gradient reinforcement learning approach for generating a fast walk on
legged robots\cite{kohl}, which we implemented for comparison. Others
have evolved gaits for legged robots, producing competitive
%results~\cite{chernova2005evolutionary, hornby2005autonomous, zykov,
%  clune2009evolving, clune2011performance, clune2009hybrid,
%  clune2009sensitivity, tellez2006evolving, valsalam2008modular}. In
results~\cite{clune2009evolving, clune2009sensitivity, hornby2005autonomous}. In
fact, an evolved gait was used in the first commercially-available
version of Sony's AIBO robot~\cite{hornby2005autonomous}.

%Except for
%work with HyperNEAT~\cite{clune2009evolving, clune2009sensitivity}, the previous evolutionary
%approaches have helped evolution exploit the regularity of the problem
%by manually decomposing the task. Experimenters have
%to choose which legs should be coordinated, or otherwise facilitate
%the coordination of motion. Part of the motivation of this paper
%is to see if the space of regularities can be explored automatically,
%which has previously performed well~\cite{valsalam2008modular}, and to
%perform a direct comparison on the same robot to HyperNEAT.

In this paper we compare the performance of two different methods of
learning gaits: parameterized gaits optimized with six different
learning methods, and gaits generated by evolving neural networks with
the HyperNEAT generative encoding~\cite{stanley2009hypercube}. While
some of these methods, such as HyperNEAT, have been tested in
simulation~\cite{clune2009evolving}, we
investigate how they perform when evolving on a physical robot.
