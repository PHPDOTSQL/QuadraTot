\subsection{Platform details}






\edit{figure:robotDiagram make figure to go somewhere here, show drawing with inner/outer, l/r, and f/b motors.}




%\section{System Architecture and Implementation}
\seclabel{implement}

% Describe how you implemented your system and how you structured it. 
% This should give an overview of the system, not a detailed 
% documentation of the code. The documentation of the code is part of 
% the code you hand in. You might want to comment on high-level design 
% decisions that you made. Also explain how you obtained your
% data and any pre-processing you did to it.
\acmFig{robot_close.jpg}{1}{The robot}
\acmFig{topdown.png}{1}{A figure of the robot from a top-down perspective,
with motors labeled}

The quadruped robot has an on-board computer running Linux. The lower
level drivers are in C and the system is implemented in
Python. 

\acmFig{robot.JPG}{1}{The infrared LED mounted on top of the robot, which
provides feedback about distance traveled in conjunction with a Wii
remote}
\acmFig{wiiMote.JPG}{1}{A Wii remote tracks the location of the robot,
providing feedback about distance traveled}

A Wii remote tracks the location of the robot, providing feedback
about distance traveled, through an infrared LED mounted on top of 
the robot. A server is run on the robot and continuously tracks its
position using the CWiid library\cite{cwiid} to interface with the remote
via bluetooth. A client then connects via a socket to the tracking
server and requests position updates periodically. If the robot moves 
beyond the viewable range of the Wii remote, the system pauses and
directs the user to move the robot back within the viewable range 
before running again.

The robot is run using a given motion model, including, if desired, 
smooth interpolation over time for the beginning and end of the run. 
The servos are prevented from being commanded to a point outside their
normal range (0 - 1023) as well as beyond points where limbs would collide.
