\subsection{Platform details}






\edit{figure:robotDiagram make figure to go somewhere here, show drawing with inner/outer, l/r, and f/b motors.}




%\section{System Architecture and Implementation}
\seclabel{implement}

% Describe how you implemented your system and how you structured it. 
% This should give an overview of the system, not a detailed 
% documentation of the code. The documentation of the code is part of 
% the code you hand in. You might want to comment on high-level design 
% decisions that you made. Also explain how you obtained your
% data and any pre-processing you did to it.

\acmFig{topdown.png}{1}{A figure of the robot from a top-down perspective,
with servos labeled}

The quadruped robot was printed on the Objet
Connex 500 3-D Printing System. It weights 1884 g and measures approximately
38 centimeters from leg to leg in the crouch position depicted in
\figref{robot_close.jpg}. There are a total of 9 AX-12+ Dynamixel
Servomotors, displayed in \figref{topdown.png}: 4 outer leg joint
servos, 4 inner leg joint servos, and a final servo at the center ``hip'' of the
robot. The robot has an on-board computer running Linux. 
The lower level drivers are in C and the system is implemented in Python. 

\acmFigg{wiiMote.JPG}{robot.JPG}{1}{A Nintendo Wii remote tracks the location
of the robot, providing feedback about distance traveled, in pixels, through
an infrared LED mounted on top of the robot.}

We controlled the experiments from a computer that was connected via a
wireless Ethernet to the robot. A Nintendo Wii remote tracks the location 
of the robot, providing feedback
about distance traveled, in pixels, through an infrared LED mounted on top of 
the robot (\figref{wiiMote.JPG}). A server is run on the robot and continuously tracks its
position using the CWiid library\cite{cwiid} to interface with the remote
via bluetooth. A client then connects via a socket to the tracking
server and requests position updates periodically. If the robot walks 
beyond the viewable range of the Wii remote, a prompt informed the
user. The only human intervention required during the
experiment was to move the robot back inside the viewable area and resume
the run, and to handle any mechnanical failures that arose. This did not
interrupt the learning process or result in the loss of data.

The robot is run using a given motion model, including, if desired, 
smooth interpolation over time for the beginning and end of the run. 
The servos are prevented from being commanded to a point outside their
normal range (0 - 1023) as well as beyond points where limbs would collide
(150 - 770 for inner leg servos, 30 - 680 for outer leg servos, and 392 - 623
for the center, or ``hip'' servo.
