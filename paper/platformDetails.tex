\subsection{Platform details}
\seclabel{platformDetails}



%\seclabel{implement}

% Describe how you implemented your system and how you structured it. 
% This should give an overview of the system, not a detailed 
% documentation of the code. The documentation of the code is part of 
% the code you hand in. You might want to comment on high-level design 
% decisions that you made. Also explain how you obtained your
% data and any pre-processing you did to it.

\acmFig{topdown.png}{.7}{A figure of the robot from a top-down perspective,
with servos labeled}

The quadruped robot used in this study was assembled from parts
printed on the Objet Connex 500 3-D Printing System. It weights 1.88
kg with the on-board computer and measures approximately 38
centimeters from leg to opposite leg in the crouch position depicted
in \figref{robot_close.jpg}. The robot is actuated by 9 AX-12+
Dynamixel servos: one inner joint and one outer joint servo in each of
the four legs, and one servo at the center ``hip'' joint.  This final
unique servo allows the two halves of the robot to rotate with respect
to each other. \figref{topdown.png} shows the positions and numerical
designations of all nine servos.  Each servo could be commanded to a
position in the range [0,~1023], corresponding to a physical range
[-120$^{\circ}$,~+120$^{\circ}$].

All of the computation for gait learning, fitness evaluation, and
robot control was performed on the compact on-board CompuLab Fit-PC2,
running Ubuntu Linux.  All gait generation, learning, and fitness
evaluation code, except HyperNEAT, was written in Python and is
available on our website \cite{quadratotWeb}.  HyperNEAT is written in
C++.  We controlled the servos with the \code{pydynamixel} library
\cite{pydynamixel}.  The robot connected to a wireless network on
boot, which enabled us to control it via SSH.


To run a gait on the robot, one needed only to provide a \emph{gait
  function} --- any Python function that accepts a single input, time
starting at 0, and outputs a list of nine commanded positions, one for
each servo.  To safeguard against limb collision with the robot body,
the robot control code cropped the commands to a safe range.  This
range was [150,~770] for the inner leg servos, [30,~680] for the outer
leg servos, and [392,~623] for the center hip servo.


%The robot was set up to run when given any giat 

%is run using a given motion model, including, if desired, 
%smooth interpolation over time for the beginning and end of the run. 

%The servos are prevented from being commanded to a point outside their
%normal range (0 - 1023) as well as beyond points where limbs would collide

