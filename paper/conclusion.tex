\section{Conclusion and Future Work}
\seclabel{conclusion}

We have presented an array of approaches to optimizing a quadrupedal
gait for forward speed.  We implemented and tested six learning
strategies for parameterized gaits and compared them with gaits
encoded and evolved by HyperNEAT.

All methods resulted in an improvement over the robot's previous
hand-coded gait.  Using linear regression to build a model of gait
performance and predict promising directions for further exploration
seemed to work well, producing at best a gait 27.58 of body
lengths/minute.  The Nelder-Mead simplex method also worked fairly
well, likely due to its robustness to noise.  The other parameterized
methods did not outperform a random gait generation method.  Because
randomly choosing SineModel5 gaits performed so well, we concluded
that the motion representation for the robot may be more critical to
parameterized gait development than the learning algorithm itself.

The HyperNEAT gaits performed much better than all parameterized
methods, considering both the average and best gaits for each.  We
conjecture that this was because the HyperNEAT encoding allowed it to
take advantage of the symmetries of the problem, but otherwise left
the evolution to explore a much larger space of possibilities than the
more restrictive 5-dimensional parameterized space.  The typical
higher-performance HyperNEAT gait pictured in
\figref{neat_110115_211410_00000_002_filt_zoom} shows the fairly
complex pattern of motion that was obtained.  Whereas the
parameterized gaits were restricted to scaled and shifted sine waves
of a single frequency, this evolved gait contains some patterns with
multiple frequencies.  Despite this larger space of possibilities,
this gait also reused the same pattern for multiple motors.

