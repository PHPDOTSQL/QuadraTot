\section{Conclusion and Future Work}
\seclabel{conclusion}

\edit{write this, old section below}

% Briefly summarize the important results and conclusions presented in 
% the paper. What are the most important points illustrated by your 
% work? How will your results improve future research and applications 
% in the area?


We have presented an array of approaches to optimizing a quadrupedal
gait for forward speed.  We have implemented and tested different
learning strategies, including uniform and Gaussian random hill
climbing, policy gradient reinforcement learning, Nelder-Mead simplex,
several new predictive methods based on linear and support vector
regression, and an evolved neural network (HyperNEAT).  We have also
compared these approaches to random search as a baseline. Many of the
methods resulted in walks significantly faster than previously
hand-tuned gaits.

Because only three trials were tested with each algorithm and no
algorithm consistently outperformed the others, there was a large
standard error for each method, as shown in \figref{std_error}. Thus
it was unclear if any algorithm was superior to another in this
application. More trials with each algorithm would be necessary to
reach a definitive ranking. Because each algorithm discovered at least one gait of
over 10 body lengths/minute, including random search, we also
conjectured that the motion representation for the robot is more
integral to forward speed than the learning algorithm.  How to learn
the motion representation, in addition to its parameters, remains an
open problem.

