\section{Conclusion and Future Work}
\seclabel{conclusion}

We have presented an array of approaches to optimizing a fast quadrupedal
gait.  We implemented and tested six learning
strategies for parameterized gaits and compared them to gaits produced by neural networks
evolved with the HyperNEAT generative encoding.

All methods resulted in an improvement over the robot's previous
hand-coded gait.  Building a model of gait
performance with linear regression to predict promising directions for further exploration
worked well, producing a gait of 27.58 body
lengths/minute.  The Nelder-Mead simplex method performed nearly as well, likely due to its robustness to noise.  The other parameterized
methods did not outperform random search.  One reason the randomly-generated SineModel5 gaits performed so well could be because the gait representation was biased towards effective, regular gaits, making the highly exploratory random strategy more effective than more exploitative learning algorithms. 

HyperNEAT produced higher-performing gaits than all of the parameterized
methods.  This could be because HyperNEAT tends to generate coordinated gaits~\cite{clune2011performance, clune2009evolving}, allowing it to
take advantage of the symmetries of the problem, but can also explore a much larger space of possibilities than the
more restrictive 5-dimensional parameterized space.  HyperNEAT gaits tended to produce more complex sequences of motor commands, with different frequencies and degrees of coordination, whereas the
parameterized gaits were restricted to scaling single-frequency sine waves and could only produce certain types of motor regularities. 
