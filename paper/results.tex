\section{Results and Discussion}
\seclabel{results}

% Outline:
% - exploration of parameter space
% - parameterized optimization results, discussion
% - HyperNEAT results, discussion



\subsection{Exploration of Parameterized Gait Space}

Before optimizing the chosen family of parameterized gaits
(\secref{motionModel}) with learning methods, we performed an
experiment to explore the five dimensions of the SineModel5 parameter
space. Specifically, we selected a random parameter vector that
resulted in some motion, but not an exceptional gait. We then varied
each of the five parameters individually and measured performance,
repeating each measurement twice to get a rough estimate of the
measurement noise at each point.  The results of this exploration,
shown in \figref{explore_dim_1}, reveal that some dimensions ($\amp$,
$\tau$, $m_F$) are fairly smooth and exhibit global structure across
the allowable parameter range, while others ($m_O$, $m_R$) exhibit
more complex behavior.  In addition, it gives a rough indication that
measurement noise is often significant and is more likely to be large
for gaits that move more.  Of course, this is only a slice in each
dimension through a single point, and slices through a different point
could reveal different behavior.  The common point at the
intersection of all slices is shown as a red triangle in
each plot of \figref{explore_dim_1}.


%\acmFig{explore_dim_1}{1}{Fitness mean and standard deviation
%  vs. dimension 1.  The circle is a common point in
%  \figref{explore_dim_1} through \figref{explore_dim_5}}
%\acmFig{explore_dim_2}{1}{Fitness mean and standard deviation
%  vs. dimension 2.  The circle is a common point in
%  \figref{explore_dim_1} through \figref{explore_dim_5}}
%\acmFig{explore_dim_3}{1}{Fitness mean and standard deviation
%  vs. dimension 3.  The circle is a common point in
%  \figref{explore_dim_1} through \figref{explore_dim_5}}
%\acmFig{explore_dim_4}{1}{Fitness mean and standard deviation
%  vs. dimension 4.  The circle is a common point in
%  \figref{explore_dim_1} through \figref{explore_dim_5}}
%\acmFig{explore_dim_5}{1}{Fitness mean and standard deviation
%  vs. dimension 5.  The circle is a common point in
%  \figref{explore_dim_1} through \figref{explore_dim_5}}

\acmFiggggg{explore_dim_1}{explore_dim_2}{explore_dim_3}{explore_dim_4}{explore_dim_5}{1}{.53}{Fitness
  mean and standard deviation when each parameter dimension is varied
  independently.  The red triangle in each plot represents the same point in the 5-dimensional parameter space.}



\subsection{Learning Methods for Parameterized Gaits}

% Outline
% - beat hand gait
% - no learning strategy outperformed another by that much
% - real world noise yay

The results for the parameterized gaits are shown in
\figref{std_error} and \tabref{results}.  As a group, these strategies
produced results equal to or worse not significantly better than the randomly-generated
gaits, however, all methods managed to explore the space enough to
significantly improve on the previous hand coded gait in at least one
of the three runs.  No single strategy consistently outperformed the
others: for the first trial Linear Regression produced the fastest
gait at 27.58 body lengths/minute, for the second a random gait
actually won with 17.26 body lengths/minute, and for the third trial
the Nelder-Mead simplex method attained the fastest gait at 14.83 body
lengths/minute.



% octave:15> fit = [6.04 17.26 4.90;  11.37 9.44 2.69;  3.10 13.59 13.40;  0.68 14.69 3.60;  8.51 13.62 14.83;  27.58 12.51 1.95;  24.27  36.44  27.07];
% octave:16> mean(fit,2)
% ans =
% 
%     9.4000
%     7.8333
%    10.0300
%     6.3233
%    12.3200
%    14.0133
%    29.2600
% 
% octave:17> std(fit,0,2)
% ans =
% 
%     6.8308
%     4.5576
%     6.0023
%     7.3914
%     3.3546
%    12.8810
%     6.3737


%%%%%% OLD TABLE
% \begin{table*}
% \begin{center}
% \begin{tabular}{|r|c|c|c||c|}
% \hline
%                                          & A       & B      & C      &  Average \\
% \hline                                                               
% \hline                                                               
% Previous hand-coded gait                 & --      & --     & --     &  5.16 \\
% \hline                                                                 
% Random search                            & 6.04    & 17.26  & 4.90   &  9.40 \\
% \hline                                                                 
% Uniform Random Hill Climbing             & 11.37   & 9.44   & 2.69   &  7.83 \\
% \hline                                                                 
% Gaussian Random Hill Climbing            & 3.10    & 13.59  & 13.40  &  10.03 \\
% \hline                                                                 
% Policy Gradient Descent                  & 0.68    & 14.69  & 3.60   &  6.32 \\
% \hline                                                                 
% Nelder-Mead simplex                      & 8.51    & 13.62  & 14.83  &  12.32 \\
% \hline                                                                 
% Linear Regression                        & 27.58   & 12.51  & 1.95   &  14.01 \\
% \hline                                                                
% Evolutionary Neural Network (HyperNEAT)  & 24.27     & 36.44    & 27.07   & 29.26  \\
% \hline
% \end{tabular}
% \caption{The best gaits found for each starting vector and algorithm,
%   in body lengths per minute.}
% \tablabel{results}
% \end{center}
% \end{table*}

%%%%%%% NEW TABLE
\begin{table}
\begin{center}
\begin{tabular}{|r|c|c|c||c|}
\hline
                                         & Average & Std. Dev. \\
\hline                                    
\hline                                    
Previous hand-coded gait                 & 5.16   &   --     \\
\hline
Random search                            & 9.40   &   6.83   \\
\hline
Uniform Random Hill Climbing             & 7.83   &   4.56   \\
\hline
Gaussian Random Hill Climbing            & 10.03  &   6.00   \\
\hline
Policy Gradient Descent                  & 6.32   &   7.39   \\
\hline
Nelder-Mead simplex                      & 12.32  &   3.35   \\
\hline
Linear Regression                        & 14.01  &  12.88   \\
\hline
Evolved Neural Network              &        &          \\
(HyperNEAT)                              & 29.26  &   6.37   \\
\hline
\end{tabular}
\caption{The best gaits found for each starting vector and algorithm,
  in body lengths per minute.}
\tablabel{results}
\end{center}
\end{table}



\acmFig{std_error}{1}{Average results ($\pm $ standard error) for each
  of the parameterized learning methods.  Linear regression found the
  fastest overall gait and had the highest average, followed by
  Nelder-Mead simplex. Many methods were beaten by the simple random
  strategy.}




\subsection{HyperNEAT Gaits}

The results for the gaits produced and evolved by HyperNEAT are shown
in \figref{hnResults} and \tabref{results}.  Overall the evolved gaits
were the fastest by far, beating all the parameterized models when
comparing either the average or the best gaits of each.

\acmFig{hnResults}{1}{Average fitness ($\pm $ standard error) of
  generation champion (highest performing individual among the
  population of nine individuals) for each generation of the three
  HyperNEAT runs.}

%\acmFig{neat_110115_211410_00000_002_filt}{1}{Caption here...???}

\acmFig{neat_110115_211410_00000_002_filt_zoom}{1}{Example of one
  particular high-performance gait produced by HyperNEAT showing
  commands for each of nine motors.  Note the complexity of the motion
  pattern. Such patterns were not possible with the parameterized
  SineModel5, nor would they likely be contrived by a human designing
  a different low-dimensional parameterized motion model.}
