\section{Introduction and Background}

\acmFig{robot_close.jpg}{1}{The 3D printed quadruped robot used for
  this study.}


Gaits for walking robots are often designed with some explicit or
implicit goal in mind.  For some applications, the design criteria may
be obvious --- perhaps the robot needs to move as quickly or as
efficiently as possible --- but other times the objective is more complicated,
requiring simultaneous optimization of several desired traits, each
with its own relative importance.  The different combinations of desired traits and the relative weight placed on each can produce drastically different gaits.    For example, Honda's Asimo
\cite{_asmio} and Boston Dynamic's Big Dog \cite{_bigdog} both require
gaits that are relatively quick, power efficient, and robust to
changing terrain, but they vary widely in the importance placed
on each attribute.  Big Dog's gait is optimzed for much more difficult terrain than Asimo's, resulting in a gait of a completely different form.

Once the design goals are decided upon, gaits may be obtained by one of several methods.  Gaits may be designed by hand by an expert, as in \cite{_asmio, _bigdog}.  Alternately, they may be learned through repeated trial and evaluation.  Learned gaits offer several advantages over manually designed gaits.  Automatically learning gaits can save valuable engineering time and can allow customization of gaits to a particular robot and its unique actuators.  Most importantly, in some cases learned gaits can outperform manually designed gaits.

In this paper we compare the performance of three different methods of designing gaits: manual design, parametrized gait optimized using one of seven learnign methods, and gaits generated and evolved by HyperNEAT.  All experiments were done in hardware using the platform shown in \figref{robot_close.jpg}. To our knowledge this is the first time HyperNEAT has been used to directly evolve gaits in hardware.

In simulation HyperNEAT has been shown to produce the highest performing gaits when using an encoding that respects the geometric regularity of a problem \cite{jeffSomething}. To take advantage of this fact, we used an encoding for the physical robot that respects the symmetry of its form; this encoding is shown in \figref{hyperEncoding???}.

In addition, past research has suggested \cite{jeffSomething} one advantage the HyperNEAT encoding has over the NEAT encoding is in the bias toward regularity of its output. That is, biasing gaits toward regularity appears to be beneficial. Thus, to make our comparison as fair as possible, for the parametrized family of gaits, we used a construction that also biases toward regularity.

\edit{more past work in here???}



%Old: design by hand
%Newer: Learn gait
%  - parmametrized or more free form (HN, ???)
%  - in sim or hardware
%Us: first time HN in hardware, first time compared to other methods aiming for %symmetry on same hardware

Often
these optimizations are done manually by an expert who designs and
tweaks a gait specifically for a given objective.  Other groups have
used learning methods to generate gaits optimized for some metric.
Approaches differ in their starting assumptions, some essentially
tweaking the parameters of a hand-tuned model \cite{chernova}, others
exploring a reasonably compact parameter space \cite{kohl}, and still
others beginning with few assumptions besides periodicity
\cite{zykov}.





\edit{Write this}

%\editbox{Outline \\
% - Why gait learning is cool \\
% - Gait learning has been done before, in simulation and on hardware (cite, cite, cite) \\
% - HyperNEAT has been used to evolve gaits before in simulation~{clune2009evolving, clune2011performance, clune2009hybrid}, and it worked well in part because the gaits were regular, raising the question of whether this high performance was just because the gaits were regular? We test this hypothesis by comparing HyperNEAT to traditional machine learning methods biased towards regular gaits.  \\
% - Here, we do two cool things for the first time: \\
% - - evolve HN in hardware \\
% - - compare it to other learning methods on the same hardware.}




\edit{Old Quadratot intro follows}

% Motivate and abstractly describe the problem you are addressing and
% how you are addressing it. What is the problem? Why is it important?
% What is your basic approach? A short discussion of how it fits into
% % related work in the area is also desirable. Summarize the basic
% results and conclusions that you will present.

Applications of walking robots often call for the ability to walk as
quickly, efficiently, or with as little power as possible.  Often
these optimizations are done manually by an expert who designs and
tweaks a gait specifically for a given objective.  Other groups have
used learning methods to generate gaits optimized for some metric.
Approaches differ in their starting assumptions, some essentially
tweaking the parameters of a hand-tuned model \cite{chernova}, others
exploring a reasonably compact parameter space \cite{kohl}, and still
others beginning with few assumptions besides periodicity
\cite{zykov}.

We aimed to strike a middle ground between these approaches.  Our
motion generator did not rely on hand-tweaked gaits, but it did use
parameterized gaits which, by their nature, impose some assumptions on
the answers produced.  We then used machine learning to design gaits
for a quadruped robot with these models.  This paper presents a
comparison of the different learning methods implemented.  Most
methods created walks that are several times faster than the original
hand-tuned gait.  We invite readers with short attention spans to view
a video of some of our results online here:

%\url{http://www.youtube.com/watch?v=ODoiOj9DdGg}
\texttt{http://www.youtube.com/watch?v=ODoiOj9DdGg}

%The code so far consists of the robot class, optimization class,
%parameterized model, sine model, and camera feedback.
