\section{Introduction and Background}

\edit{Write this}

\editbox{Outline \\
 - Why gait learning is cool \\
 - Gait learning has been done before, in simulation and on hardware (cite, cite, cite) \\
 - HyperNEAT has been used to evolve gaits before in simulation~{clune2009evolving, clune2011performance, clune2009hybrid}, and it worked well in part because the gaits were regular, raising the question of whether this high performance was just because the gaits were regular? We test this hypothesis by comparing HyperNEAT to traditional machine learning methods biased towards regular gaits.  \\
 - Here, we do two cool things for the first time: \\
 - - evolve HN in hardware \\
 - - compare it to other learning methods on the same hardware.}




\edit{Old Quadratot intro follows}

% Motivate and abstractly describe the problem you are addressing and
% how you are addressing it. What is the problem? Why is it important?
% What is your basic approach? A short discussion of how it fits into
% % related work in the area is also desirable. Summarize the basic
% results and conclusions that you will present.

Applications of walking robots often call for the ability to walk as
quickly, efficiently, or with as little power as possible.  Often
these optimizations are done manually by an expert who designs and
tweaks a gait specifically for a given objective.  Other groups have
used learning methods to generate gaits optimized for some metric.
Approaches differ in their starting assumptions, some essentially
tweaking the parameters of a hand-tuned model \cite{chernova}, others
exploring a reasonably compact parameter space \cite{kohl}, and still
others beginning with few assumptions besides periodicity
\cite{zykov}.

We aimed to strike a middle ground between these approaches.  Our
motion generator did not rely on hand-tweaked gaits, but it did use
parameterized gaits which, by their nature, impose some assumptions on
the answers produced.  We then used machine learning to design gaits
for a quadruped robot with these models.  This paper presents a
comparison of the different learning methods implemented.  Most
methods created walks that are several times faster than the original
hand-tuned gait.  We invite readers with short attention spans to view
a video of some of our results online here:

%\url{http://www.youtube.com/watch?v=ODoiOj9DdGg}
\texttt{http://www.youtube.com/watch?v=ODoiOj9DdGg}

%The code so far consists of the robot class, optimization class,
%parameterized model, sine model, and camera feedback.
