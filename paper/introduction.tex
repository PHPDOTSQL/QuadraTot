\section{Introduction and Background}

\acmFig{robot_close.jpg}{1}{The 3D printed quadruped robot used for
  this study.}


Gaits for walking robots are often designed with some explicit or
implicit goal in mind.  For some applications, the design criteria may
be obvious: perhaps the robot needs to move as quickly or as
efficiently as possible, or with minimal power usage or strain on
actuators.  In many cases, however, the objective is more complicated,
requiring simultaneous optimization of several desired traits, each
with its own relative importance.  For example, Honda's Asimo
\cite{_asmio} and Boston Dynamic's Big Dog \cite{_bigdog} both require
gaits that are relatively quick, power efficient, and robust to
changing terrain, but with they vary widely in the importance placed
on each attribute.  A primary concern for Big Dog's gait is that it is
robust to steep terrain, whereas

In other cases, the objective may be more complicated; perhaps it is
desired that the gait be fairly [robust]

\_asimo and \_bigDog very different (and some hexapod?)

Many times gait will be designed and tweaked by hand \cite{_asimo}

here we describe a study, a comparison of few methods. started with
sped, but then not-breaking as well.


When designing gaits for walking robots, often it is desirable that
this gait enable the robot to walk as

Old: design by hand
Newer: Learn gait
  - parmametrized or more free form (HN, ???)
  - in sim or hardware
Us: first time HN in hardware, first time compared to other methods aiming for symmetry on same hardware

Often
these optimizations are done manually by an expert who designs and
tweaks a gait specifically for a given objective.  Other groups have
used learning methods to generate gaits optimized for some metric.
Approaches differ in their starting assumptions, some essentially
tweaking the parameters of a hand-tuned model \cite{chernova}, others
exploring a reasonably compact parameter space \cite{kohl}, and still
others beginning with few assumptions besides periodicity
\cite{zykov}.





\edit{Write this}

%\editbox{Outline \\
% - Why gait learning is cool \\
% - Gait learning has been done before, in simulation and on hardware (cite, cite, cite) \\
% - HyperNEAT has been used to evolve gaits before in simulation~{clune2009evolving, clune2011performance, clune2009hybrid}, and it worked well in part because the gaits were regular, raising the question of whether this high performance was just because the gaits were regular? We test this hypothesis by comparing HyperNEAT to traditional machine learning methods biased towards regular gaits.  \\
% - Here, we do two cool things for the first time: \\
% - - evolve HN in hardware \\
% - - compare it to other learning methods on the same hardware.}




\edit{Old Quadratot intro follows}

% Motivate and abstractly describe the problem you are addressing and
% how you are addressing it. What is the problem? Why is it important?
% What is your basic approach? A short discussion of how it fits into
% % related work in the area is also desirable. Summarize the basic
% results and conclusions that you will present.

Applications of walking robots often call for the ability to walk as
quickly, efficiently, or with as little power as possible.  Often
these optimizations are done manually by an expert who designs and
tweaks a gait specifically for a given objective.  Other groups have
used learning methods to generate gaits optimized for some metric.
Approaches differ in their starting assumptions, some essentially
tweaking the parameters of a hand-tuned model \cite{chernova}, others
exploring a reasonably compact parameter space \cite{kohl}, and still
others beginning with few assumptions besides periodicity
\cite{zykov}.

We aimed to strike a middle ground between these approaches.  Our
motion generator did not rely on hand-tweaked gaits, but it did use
parameterized gaits which, by their nature, impose some assumptions on
the answers produced.  We then used machine learning to design gaits
for a quadruped robot with these models.  This paper presents a
comparison of the different learning methods implemented.  Most
methods created walks that are several times faster than the original
hand-tuned gait.  We invite readers with short attention spans to view
a video of some of our results online here:

%\url{http://www.youtube.com/watch?v=ODoiOj9DdGg}
\texttt{http://www.youtube.com/watch?v=ODoiOj9DdGg}

%The code so far consists of the robot class, optimization class,
%parameterized model, sine model, and camera feedback.
