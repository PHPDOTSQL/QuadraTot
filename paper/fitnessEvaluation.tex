\subsection{Fitness evaluation details}
\seclabel{fitnessEvaluation}

\editbox{write this.  Old quadratot Experimental Evaluation section below.}

%\section{Experimental Evaluation}

%\subsection{Methodology}
% What are the criteria you are using to evaluate your method? What
% specific hypotheses does your experiment test? Describe the 
% experimental methodology that you used. What are the dependent and 
% independent variables? For projects in machine learning, what is 
% the training/test data that was used, and why is it realistic or
% interesting? Exactly what performance data did you collect and how 
% are you presenting and analyzing it? Comparisons to competing methods 
% that address the same problem or to variations of your own algorithm 
% are particularly useful.

The metric for evaluation of the designed gait is speed. We stop each 
run after plateauing results (no improvement for one third of the 
policies seen so far). The standard length of a run designates that it
should be stopped after there is no improvement for one half of the policies
seen so far, but since all runs took place on the actual robot, without use
of a simulator, certain time limitations were imposed on the learning process.

We controlled our experiments from a computer that was connected via a
wireless ethernet to the robot. The robot collected data about
distance walked automatically on its own. If it walked outside of the
Wii remote's viewable area, it informed the user, so the only human
intervention required during an experiment was to move the robot back
inside the viewable area and to resume the run, which did not
interrupt the learning process or result in the loss of data.

We evaluated the efficacy of a set of parameters by sending these
parameters to the robot and instructing it to walk for a certain
length of time. The robot always began from the same position and
returned to the starting position at the end of the run in order to
measure true displacement without giving credit for ending in a leaned
position. More efficient parameters resulted in a faster gait, which
translated into a longer distance walked and a better score. After completing
an evaluation, the robot sent the resulting distance walked back to the
host computer and prepared itself for a new set of parameters to evaluate.

Each algorithm was run on 3 different initial parameter vectors on the
physical robot. We decided to evaluate all methods starting at the
same three vector in order to allow for the fair comparison of each
algorithm.  We evaluate each method based on the amount of improvement
seen from the initial parameter vectors, and on the fastest speeds
achieved during runs.

The resulting gaits from our algorithms quickly outperformed the original
hand-coded walk designed for this robot. The fastest walk, for example, was
4 times faster.
