\subsection{Fitness Evaluation Details}
\seclabel{fitnessEvaluation}

% Points to make:
% - linear interpolation
% - fitness reported in pixels
% - parametrized models
%   - started all on three thetas
%   - plateued at 1/3 of lifetime
% - HyperNEAT
%   - broke things a lot... so guided evolution (or in HN section?)
%   - eventually included ``shimmy''
% - finite sized window led to unfortunate bias
%
% Given earlier section, now have way of evaluating any gait g(t).
%
% Want to measure average speed, so we measure distance displaced over
% a fixed lenght run. Because we're measureing a single point on
% robot, need to start and end in same position (crouch)
% So we linearly interpolate begining (1s) and end (2s).
%
% Fitness is measured and reported directly as euclidian distance of pixels moved: equation
%
% For the parametrized models, the starting location may be important. To be as fair as possible, we picked three random points and starred all models there.
%   generally ran until 1/3 plateau
%
% For HN, couldn't do a,b,c thing, so just did three runs of length <foo>
% Things started breaking, so included shimmy to test brokenness
% penalized runs that broke servos
%
% Finite sized window kinda hurt stuff.


\acmFigg{wiiMote_crop.jpg}{robot_led_crop.jpg}{1}{A Nintendo Wii
  remote provided the location of the robot by tracking the infrared
  LED mounted on the robot's antenna.  The LED position was measured
  in pixels and transmitted from the Wii remote to the robot via
  bluetooth.}

To track the position of the robot and thus determine gait fitness, we
mounted a Nintendo Wii remote on the ceiling and an infrared LED on
top of the robot (\figref{wiiMote_crop.jpg}).  The Wii remote contains
an IR camera that can track and report the position of up to four IR
sources in its image frame.  The resolution of the camera was 1024 by
768 pixels with view angles of about 40$^\circ$ by 30$^\circ$, which
produced a resolution of 1.7mm per pixel when mounted at a height of
2.63m.  At this height, the viewable window on the floor was
approximately 175 x 120 cm.

A separate tracking server, written in Python, ran on the robot and
interfaced with the Wii remote via bluetooth using the CWiid
library\cite{cwiid}.  Our fitness testing code communicated with this
server via socket connection and requested position updates at the
beginning and end of each run.



As mentioned earlier, the metric for evaluating gaits was their
average speed.  In reality this was measured as the distance the robot
moved over a 12 second run.  The Euclidean distance between the
starting position and the ending position, in pixels, was reported
directly as the fitness (save for a tweak for HyperNEAT; see below).

Since only a single point on the robot --- the IR LED --- was measured
for the purposes of computing fitness, it was important that the
position of the IR LED accurately reflect the position of the robot as
a whole.  To enforce this constraint, the robot was always measured
while in the \emph{ready} position (the position shown in
\figref{robot_close.jpg}).  This was done to prevent assigning extra
fitness to, for example, gaits that ended with the robot leaning into
the direction of travel.  While this would mean that a part of the
robot has traveled farther, this extra distance would not generalize
to a longer run.  Thus, we chose not to reward this behavior.


In order to measure start and end position in the same pose, 

ensure we could
start and end each run in the same position,


\begin{tabular}{ccccc}
Ready position & a & b & c & d\\
\hline
t = 0, measure  & $t \in [0,1]$ & $t \in [1,10]$ & $t \in [10,12]$ & t = 12
\end{tabular}




\edit{Put this in there somewhere:}

To be as fair as possible when obtaining the results presented in
\secref{results}, we picked three random points --- $\vec{\theta}_A$,
$\vec{\theta}_B$, and $\vec{\theta}_C$ --- and started all seven
methods at these points...

\editbox{proofread. especially middle paragraph}

%\section{Experimental Evaluation}

The metric for evaluation of the designed gait was speed. To evaluate
a set of parameters, the robot was sent the parameters and instructed
to walk for a certain length of time. For each evaluation, the robot
always started and ended in the same position in order to measure true
displacement and not reward gaits the ended in a lean, since the LED
would have moved a different distance the robot. More efficient
parameters resulted in a faster gait, which translated into a longer
distance walked, measured in pixels, and a better score.

The size of the Wii remote window is approximately 175 x 120 cm. A
concern about this is that if the robot walks further than usual,
it walks outside the viewable range and thus does not count the run.
This possibility could bias the final results reported on the robot.
% probably could be better written

The only human intervention required during most learning was to
occasionally move the robot back into the viewable area of the w


Each of the parameter ??? methods was run on 3 different initial
parameter vectors, in order to fairly compare the algorithms. We allowed 
each run for the parameter ??? methods to
continue until the results plateaued (no improvement for one third of
the policies seen so far). Three runs of HyperNEAT were completed,
each with a different initial seed and run for 20 generations.
