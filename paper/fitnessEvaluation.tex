\subsection{Fitness evaluation details}
\seclabel{fitnessEvaluation}

\editbox{proofread.}

%\section{Experimental Evaluation}

We controlled the experiments from a computer that was connected via a wireless Ethernet to the robot. An infrared LED was mounted to the antenna on top of the robot and a Wii remote was attached to the ceiling. The Wii remote tracked the location of the LED. The robot tracked its position via Bluetooth by using the CWiid library\cite{cwiid} to communicate with the Wii remote. If the robot walked outside of the Wii remote's viewable area, a prompt informed the experimenter. The only human intervention required during an experiment was to move the robot back inside the viewable area and resume the run. This did not interrupt the learning process or result in the loss of data.

The metric for evaluation of the designed gait was speed. To evaluate a set of parameters, the robot was sent the parameters and instructed to walk for a certain length of time. For each evaluation, the robot always started and ended in the same position in order to measure true displacement and not reward gaits the ended in a lean, since the LED would have moved a different distance the robot. More efficient parameters resulted in a faster gait, which translated into a longer distance walked and a better score.

Each of the parameter ??? methods was run on 3 different initial parameter vectors, in order to allow for the fair comparison of the algorithms. We allowed each run for the parameter ??? methods to continue until the results plateaued (no improvement for one third of the policies seen so far). Three runs of HyperNEAT were completed, each with a different initial seed and run for 20 generations.
