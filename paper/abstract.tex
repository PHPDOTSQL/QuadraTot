Creating gaits for legged robots is an important task to enable robots
to access rugged terrain, yet designing such gaits by hand is a
challenging and time-consuming process. In this paper we investigate
various algorithms for automating the creation of quadruped
gaits. Because many robots do not have accurate simulators, we test
gait learning algorithms entirely on a physical robot.  We compare the
performance of two classes of learning gaits: locally searching
parameterized motion models and evolving artificial neural networks
with the HyperNEAT generative encoding. Specifically, we test six
different parameterized learning strategies: uniform and Gaussian
random hill climbing, policy gradient reinforcement learning,
Nelder-Mead simplex, a random baseline, and a new method that builds a
model of the fitness landscape with linear regression to guide further
exploration.  While all parameter search methods outperformed a
manually designed gait, only the linear regression and Nelder-Mead
simplex strategies outperform a random baseline strategy. Gaits
evolved by HyperNEAT perform considerably better than all
parameterized local search methods and produced gaits nearly 9 times faster than a hand-designed gait.  The best evolved HyperNEAT gaits exhibit
complex motion patterns that contain multiple frequencies, yet are
regular in that the leg movements are coordinated.

%Overall, this
%research illuminates various approaches to evolving quadruped gaits
%directly in hardware.

%Learning approaches differ in their starting assumptions, some
%tweaking the parameters of a hand-tuned model, others exploring a
%reasonably compact parameter space, and still others beginning with
%few assumptions besides periodicity.
