Applications of walking robots often call for the ability to walk as
quickly, efficiently, or with as little power as possible.  Gaits to
achieve these objectives may be manually designed by an expert or
learned by repeated trial and error.

Learning approaches differ in their starting assumptions, some
essentially tweaking the parameters of a hand-tuned model, others
exploring a reasonably compact parameter space, and still others
beginning with few assumptions besides periodicity.

This study extensively compares three methods of designing gaits:
manually design, local search of parameterized motion models, and
evolution of artificial neural networks using the HyperNEAT encoding.

All tests....

While these three methods have been used individually for designing
gaits, to the authors' knowledge they have never been compared 

LR works, NM works


% In this study we make an ex
% We aimed to strike a middle ground between these approaches.  Our 
% motion generator did not rely on hand-tweaked gaits, but it did use 
% parameterized gaits which, by their nature, impose some assumptions on 
% the answers produced.  We then used machine learning to design gaits 
% for a quadruped robot with these models.  This paper presents a 
% comparison of the different learning methods implemented.  Most 
% methods created walks that are several times faster than the original 
% hand-tuned gait.  We invite readers with short attention spans to view 
% a video of some of our results online here: 


We have presented an array of approaches to optimizing a quadrupedal
gait for forward speed.  We implemented and tested six learning
strategies for parameterized gaits and compared them with gaits
encoded and evolved by HyperNEAT.

All methods resulted in an improvement over the robot's previous
hand-coded gait.  Using linear regression to build a model of gait
performance and predict promising directions for further exploration
seemed to work well, producing at best a gait 27.58 of body
lengths/minute.  The Nelder-Mead simplex method also worked fairly
well, likely due to its robustness to noise.  The other parameterized
methods did not outperform a random gait generation method.  Because
randomly choosing SineModel5 gaits performed so well, we concluded
that the motion representation for the robot may be more critical to
parameterized gait development than the learning algorithm itself.

The HyperNEAT gaits performed much better than all parameterized
methods, considering both the average and best gaits for each.  We
conjecture that this was because the HyperNEAT encoding allowed it to
take advantage of the symmetries of the problem, but otherwise left
the evolution to explore a much larger space of possibilities than the
more restrictive 5-dimensional parameterized space.  The typical
higher-performance HyperNEAT gait pictured in
\figref{neat_110115_211410_00000_002_filt_zoom} shows the fairly
complex pattern of motion that was obtained.  Whereas the
parameterized gaits were restricted to scaled and shifted sine waves
of a single frequency, this evolved gait contains some patterns with
multiple frequencies.  Despite this larger space of possibilities,
this gait also reused the same pattern for multiple motors.





% A brief summary of the report. This is not an introduction -
% it should be complete. Include 1-2 sentences for each of the items
% below, up to and including conclusions.

This paper presents an array of approaches to optimizing a quadrupedal
gait for forward speed.  We implement, test, and compare different
learning strategies including uniform and Gaussian random hill
climbing, policy gradient reinforcement learning,
Nelder-Mead simplex, new predictive methods based on linear
and support vector regression, and an evolved neural network
(HyperNEAT).  We compare results to a baseline random
search method.  Many of the methods resulted in walks significantly
faster than previously hand-tuned gaits.

%Because the fastest learned walk
%was not significantly faster than the fastest randomly generated walk,
%we conjecture that the motion representation for the robot is more
%integral to forward speed than the learning algorithm.

\edit{Write the abstract}
