\section{Problem definition}
% Precisely define the problem you are addressing (i.e. formally specify
% the inputs and outputs). Elaborate on why this is an interesting and
% important problem.
%
%\edit{write this.  Quadratot report section below}
%
%We are testing several different learning methods to design a
%parametrized gait for a quadruped robot from the Cornell Computational
%Synthesis Lab.
%
%The output each of the learning algorithms is a function of time,
%$g(t)$, that outputs a vector of commanded motor positions.  This
%function is generated using a parametrized motion model, described in
%\secref{implement}.
%
%The robot executes these commands and measures its change in location
%using the tracking system described in \secref{implement}.  The input
%to the learning algorithms is this measured displacement, which the
%algorithms attempt to maximize. This displacement is measured for each
%gait over a constant length run, usually 12 seconds.
%
%A comparison and evaluation of the many different methods available
%for optimizing the gait of legged robots will be useful for future
%work on this challenging multidimensional control problem.


The gait learning problem aims to find a \emph{gait} that maximizes
some performance metric.

Mathematically, we define a gait as a function that specifies
a vector of commanded motor positions for a robot over time.
%Using this definition, we require one function per motor.
We can write gaits without feedback --- also called feedforward gaits
--- as

\be
\vec{x} = g(t)
\ee

%\noindent for commanded position $x_i$ for motor $i$.  The
\noindent for commanded position vector $\vec{x}$.  The
function depends only on time.  Using the same framework, a gait
including measured motor position, force, or other values could be
written as

\be
\vec{x} = g(t, \vec{z})
\ee

\noindent for some measurement vector $\vec{z}$.  Feedforward gaits
are deterministic (unless they include random elements), so they
produce the same command pattern each time they are run.  While the
commanded positions will be the same from trial to trial, the actual
robot motion and measured fitness will vary due to the noisiness of
trials in the real world.

The output of a gait that includes feedback, on the other hand,
depends on the actual measurements, and noise in these measurements
may result in different $\vec{x}$ values each time they are run.

For the system evaluated in this paper, we chose to use only
feedforward gaits.  An interesting extension would be to allow gaits
to depend on the measured servo positions, loads, voltage drops, or
other quantities.

The ultimate goal was to design gaits that were as fast as possible,
so to optimize for speed, our performance metric was displacement over
a period of motion lasting approximately 10 seconds.  For details of
exactly how this displacement was measured, see
\secref{fitnessEvaluation}.
