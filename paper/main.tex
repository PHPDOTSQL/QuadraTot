% This is "sig-alternate.tex" V1.9 April 2009
% This file should be compiled with V2.4 of "sig-alternate.cls" April 2009
%
% This example file demonstrates the use of the 'sig-alternate.cls'
% V2.4 LaTeX2e document class file. It is for those submitting
% articles to ACM Conference Proceedings WHO DO NOT WISH TO
% STRICTLY ADHERE TO THE SIGS (PUBS-BOARD-ENDORSED) STYLE.
% The 'sig-alternate.cls' file will produce a similar-looking,
% albeit, 'tighter' paper resulting in, invariably, fewer pages.
%
% ----------------------------------------------------------------------------------------------------------------
% This .tex file (and associated .cls V2.4) produces:
%       1) The Permission Statement
%       2) The Conference (location) Info information
%       3) The Copyright Line with ACM data
%       4) NO page numbers
%
% as against the acm_proc_article-sp.cls file which
% DOES NOT produce 1) thru' 3) above.
%
% Using 'sig-alternate.cls' you have control, however, from within
% the source .tex file, over both the CopyrightYear
% (defaulted to 200X) and the ACM Copyright Data
% (defaulted to X-XXXXX-XX-X/XX/XX).
% e.g.
% \CopyrightYear{2007} will cause 2007 to appear in the copyright line.
% \crdata{0-12345-67-8/90/12} will cause 0-12345-67-8/90/12 to appear in the copyright line.
%
% ---------------------------------------------------------------------------------------------------------------
% This .tex source is an example which *does* use
% the .bib file (from which the .bbl file % is produced).
% REMEMBER HOWEVER: After having produced the .bbl file,
% and prior to final submission, you *NEED* to 'insert'
% your .bbl file into your source .tex file so as to provide
% ONE 'self-contained' source file.
%
% ================= IF YOU HAVE QUESTIONS =======================
% Questions regarding the SIGS styles, SIGS policies and
% procedures, Conferences etc. should be sent to
% Adrienne Griscti (griscti@acm.org)
%
% Technical questions _only_ to
% Gerald Murray (murray@hq.acm.org)
% ===============================================================
%
% For tracking purposes - this is V1.9 - April 2009

\documentclass{sig-alternate}


% [JBY] might not want some of these?
\usepackage{graphicx}
\usepackage{fancyhdr}
\usepackage{color}
%\usepackage{ulem}
\usepackage{ownstyles}
\usepackage{hyperref}

\graphicspath{{../figures/}}




\begin{document}
%
% --- Author Metadata here ---
\conferenceinfo{GECCO}{2011 Dublin, Ireland}
%\CopyrightYear{2007} % Allows default copyright year (20XX) to be over-ridden - IF NEED BE.
%\crdata{0-12345-67-8/90/01}  % Allows default copyright data (0-89791-88-6/97/05) to be over-ridden - IF NEED BE.
% --- End of Author Metadata ---

%\title{A comparison of gait learning methods on a quadruped robot}
%\title{Seven gait learning methods on a quadruped robot}
%\title{Generating gaits for physical quadruped robots: evolved neural networks vs. local parameterized search. \\Round one: neuroevolution.}
\title{Generating gaits for physical quadruped robots: evolved neural networks vs. local parameterized search.}
%
% You need the command \numberofauthors to handle the 'placement
% and alignment' of the authors beneath the title.
%
% For aesthetic reasons, we recommend 'three authors at a time'
% i.e. three 'name/affiliation blocks' be placed beneath the title.
%
% NOTE: You are NOT restricted in how many 'rows' of
% "name/affiliations" may appear. We just ask that you restrict
% the number of 'columns' to three.
%
% Because of the available 'opening page real-estate'
% we ask you to refrain from putting more than six authors
% (two rows with three columns) beneath the article title.
% More than six makes the first-page appear very cluttered indeed.
%
% Use the \alignauthor commands to handle the names
% and affiliations for an 'aesthetic maximum' of six authors.
% Add names, affiliations, addresses for
% the seventh etc. author(s) as the argument for the
% \additionalauthors command.
% These 'additional authors' will be output/set for you
% without further effort on your part as the last section in
% the body of your article BEFORE References or any Appendices.

\numberofauthors{3} %  in this sample file, there are a *total*
% of EIGHT authors. SIX appear on the 'first-page' (for formatting
% reasons) and the remaining two appear in the \additionalauthors section.
%

\author{
% You can go ahead and credit any number of authors here,
% e.g. one 'row of three' or two rows (consisting of one row of three
% and a second row of one, two or three).
%
% The command \alignauthor (no curly braces needed) should
% precede each author name, affiliation/snail-mail address and
% e-mail address. Additionally, tag each line of
% affiliation/address with \affaddr, and tag the
% e-mail address with \email.
%
%
% % 1st. author
% \alignauthor
% Jason Yosinski\\
%        \affaddr{Cornell University}\\
%        \affaddr{239 Upson Hall}\\
%        \affaddr{Ithaca, NY  14853, USA}\\
%        \email{jy495@cornell.edu}
% % 2nd. author
% \alignauthor
% Diana Hidalgo\\
%        \affaddr{Cornell University}\\
%        \affaddr{239 Upson Hall}\\
%        \affaddr{Ithaca, NY  14853, USA}\\
%        \email{djh283@cornell.edu}
% % 3rd. author
% \alignauthor
% Sarah ``Phamily'' Nguyen\\
%        \affaddr{Cornell University}\\
%        \affaddr{239 Upson Hall}\\
%        \affaddr{Ithaca, NY  14853, USA}\\
%        \email{smn64@cornell.edu}
% \and  % use '\and' if you need 'another row' of author names
% % 4th. author
% \alignauthor Jeff ``Gene Pool Lifeguard'' Clune\\
%        \affaddr{Cornell University}\\
%        \affaddr{239 Upson Hall}\\
%        \affaddr{Ithaca, NY  14853, USA}\\
%        \email{jeffclune@cornell.edu}
% % 5th. author
% \alignauthor Hod Lipson\\
%        \affaddr{Cornell University}\\
%        \affaddr{242 Upson Hall}\\
%        \affaddr{Ithaca, NY  14853, USA}\\
%        \email{hod.lipson@cornell.edu}
%
%
% 1st. author
\alignauthor
Anonymous Author \\
       \affaddr{Some organization}\\
       \affaddr{1234 address}\\
       \affaddr{Somewhere}\\
       \email{anony@mous.com}
% 2nd. author
\alignauthor
Anonymous Author \\
       \affaddr{Some organization}\\
       \affaddr{1234 address}\\
       \affaddr{Somewhere}\\
       \email{anony@mous.com}
% 3rd. author
\alignauthor
Anonymous Author \\
       \affaddr{Some organization}\\
       \affaddr{1234 address}\\
       \affaddr{Somewhere}\\
       \email{anony@mous.com}
%\and  % use '\and' if you need 'another row' of author names
%% 4th. author
%\alignauthor
%Anonymous Author \\
%       \affaddr{Some organization}\\
%       \affaddr{1234 address}\\
%       \affaddr{Somewhere}\\
%       \email{anony@mous.com}
%% 5th. author
%\alignauthor
%Anonymous Author \\
%       \affaddr{Some organization}\\
%       \affaddr{1234 address}\\
%       \affaddr{Somewhere}\\
%       \email{anony@mous.com}
}

% There's nothing stopping you putting the seventh, eighth, etc.
% author on the opening page (as the 'third row') but we ask,
% for aesthetic reasons that you place these 'additional authors'
% in the \additional authors block, viz.

%\additionalauthors{Additional authors: John Smith (The Th{\o}rv{\"a}ld Group,
%email: {\texttt{jsmith@affiliation.org}}) and Julius P.~Kumquat
%(The Kumquat Consortium, email: {\texttt{jpkumquat@consortium.net}}).}

\date{9 February 2011}

% Just remember to make sure that the TOTAL number of authors
% is the number that will appear on the first page PLUS the
% number that will appear in the \additionalauthors section.

\maketitle

\begin{abstract}
% A brief summary of the report. This is not an introduction -
% it should be complete. Include 1-2 sentences for each of the items
% below, up to and including conclusions.

This paper presents an array of approaches to optimizing a quadrupedal
gait for forward speed.  We implement, test, and compare different
learning strategies including uniform and Gaussian random hill
climbing, policy gradient reinforcement learning,
Nelder-Mead simplex, new predictive methods based on linear
and support vector regression, and an evolved neural network
(HyperNEAT).  We compare results to a baseline random
search method.  Many of the methods resulted in walks significantly
faster than previously hand-tuned gaits.

%Because the fastest learned walk
%was not significantly faster than the fastest randomly generated walk,
%we conjecture that the motion representation for the robot is more
%integral to forward speed than the learning algorithm.

\edit{Write the abstract}

\end{abstract}

% A category with the (minimum) three required fields
\category{I.2.6}{Artificial Intelligence}{Learning}



% \terms is one of 16 general terms here:
% http://homepages.inf.ed.ac.uk/wadler/fool/authors/sig-alternate-v1.htm
\terms{Algorithms, Experimentation, Performance}

\keywords{Evolving Quadruped Robotic Gaits, HyperNEAT}


%%%%%%%%%%%%%%%%%%%%%%

% Section 1
\section{Introduction}

\figp{quadratotWhiteBooties}{.75}{Example figure...}

Various learning algorithms have been tested to be effective for
legged robots. Their studies tested the viability of applying RL
cite{yosinski2011evolvingmethod to legged robot gaits learning. Competitive performance has
been verified on methods such as HyperNEAT \cite{yosinski2011evolving-robot-gaits} and
GA \cite{chernova2004an-evolutionary-approach-to-gait} and other methods \cite{hornby2005autonomous-evolution-of-dynamic} \cite{zykov2004evolving-dynamic-gaits} \cite{tellez2006evolving-the-walking-behaviour} \cite{valsalam2008modular-neuroevolution-for-multilegged}. Despite the excellent performance of these algorithms, a major
part of achieving good performance relies in tuning the parameters for
the evolutionary algorithms \cite{kormushev2011bipedal-walking-energy}. Here we present a
different way for learning gaits using RL (Hill climbing). In our
experiment, it turned out this method achieved competitive performance
with Problem Definition:The gait-learning problem is defined to find a
gait that maximizes some specific metric. Mathematically, we define a
gait as a function that specifies a vector of commanded motor
positions for a robot over time. Gaits without feedback --- also
called open-loop gaits --- can be defined as

\be\vec{x} = g(t)\ee

According to the definition, open-loop gaits are deterministic. One
particular gait should behave exactly the same when it is run from
trial to trial. However, the actual robot motion and fitness measured
will vary due to the noisiness of the real world physics.In our
trials, one gait pattern was generated and sent to the robot and
executed in an open loop manner, as defined. In particular, in this
paper, we chose to compare open-loop gaits generated specifically by
the RL\_POWER and HyperNEAT. The particular metric used will be
described later.

\subsection{Related Work}

\editbox{If we talk about past gait-learning work in here, is it a bit redundant?}

\edit{write this.  Old quadratot section below}



% Briefly explain who else worked on related problems in the past and what
% methods they used. Explain if you are using similar methods, or if your
% approach is different and if so - how (either is ok).


Various maching learning techniques have proven to be useful in
finding control policies for a wide variety of robots. Kohl and
Stone\cite{kohl} presented a policy gradient reinforcement learning
approach for generating a fast walk on legged robots. We experimented
with this method to create a walk for our robot (Policy Gradient
Descent). Chernova and Velosa\cite{chernova} took an evolutionary
approach to this problem which we did not implement.  Zykov, Bongard,
and Lipson\cite{zykov} describe the evolution of dynamic gaits on a
physical robot requiring no prior assumptions about the locomotion
pattern beyond the fact that it should be rhythmic.



\subsection{Outline of Sections}

The remainder of the paper is organized as follows. In
\secref{problemDefinition} we define more rigorously the problem of
gait learning. \secref{experimentalSetup} describes our experimental
setup, including the hardware we used and our methods for evaluating
fitness of a gait. \secref{gaitGenLearn} discusses the different gait
generation and learning methods we tested, and \secref{results}
presents and discusses performance results.  Finally,
\secref{conclusion} concludes with possible future extensions to this
work.


% Section 2
\section{Problem definition}
\seclabel{problemDefinition}

% Precisely define the problem you are addressing (i.e. formally specify
% the inputs and outputs). Elaborate on why this is an interesting and
% important problem.
%
%\edit{write this.  Quadratot report section below}
%
%We are testing several different learning methods to design a
%parameterized gait for a quadruped robot from the Cornell Computational
%Synthesis Lab.
%
%The output each of the learning algorithms is a function of time,
%$g(t)$, that outputs a vector of commanded motor positions.  This
%function is generated using a parameterized motion model, described in
%\secref{implement}.
%
%The robot executes these commands and measures its change in location
%using the tracking system described in \secref{implement}.  The input
%to the learning algorithms is this measured displacement, which the
%algorithms attempt to maximize. This displacement is measured for each
%gait over a constant length run, usually 12 seconds.
%
%A comparison and evaluation of the many different methods available
%for optimizing the gait of legged robots will be useful for future
%work on this challenging multidimensional control problem.


The gait learning problem aims to find a \emph{gait} that maximizes
some performance metric.

Mathematically, we define a gait as a function that specifies
a vector of commanded motor positions for a robot over time.
%Using this definition, we require one function per motor.
We can write gaits without feedback --- also called open-loop or
feedforward gaits --- as

\be
\vec{x} = g(t)
\ee

%\noindent for commanded position $x_i$ for motor $i$.  The
\noindent for commanded position vector $\vec{x}$.  The
function depends only on time.

%Using the same framework, a gait
%including measured motor position, force, or other values could be
%written as
%
%\be
%\vec{x} = g(t, \vec{z})
%\ee
%
%\noindent for some measurement vector $\vec{z}$.

It follows that feedforward gaits are deterministic, producing the
same command pattern each time they are run.  While the commanded
positions will be the same from trial to trial, the actual robot
motion and measured fitness will vary due to the noisiness of trials
in the real world.

%The output of a gait that includes feedback, on the other hand,
%depends on the actual measurements, and noise in these measurements
%may result in different $\vec{x}$ values each time they are run.

For the system evaluated in this paper, we chose to compare only
feedforward gaits generated by both the parameterized methods and
HyperNEAT.  An interesting extension would be to allow gaits to depend
on the measured servo positions, loads, voltage drops, or other
quantities.

The ultimate goal was to design gaits that were as fast as possible,
so to optimize for speed, our performance metric was displacement over
a period of motion lasting approximately 12 seconds.  For details of
exactly how this displacement was measured, see
\secref{fitnessEvaluation}.


% Section 3
\section{Experimental Setup}
	\seclabel{experimentalSetup}
	
	An important aspect of this study was to evaluate gait generation and
	learning algorithms in the noisy, unforgiving real world, instead of
	using a simulator.  In this section we give details of the hardware
	setup we used: \secref{platformDetails} discusses the robotic
	platform, and \secref{fitnessEvaluation} describes the tracking system
	we used to measure each gait's fitness.
	
	\subsection{Platform details}
\seclabel{platformDetails}



%\seclabel{implement}

% Describe how you implemented your system and how you structured it. 
% This should give an overview of the system, not a detailed 
% documentation of the code. The documentation of the code is part of 
% the code you hand in. You might want to comment on high-level design 
% decisions that you made. Also explain how you obtained your
% data and any pre-processing you did to it.

\acmFig{topdown.png}{1}{A figure of the robot from a top-down perspective,
with servos labeled}

The quadruped robot used in this study was assembled from parts
printed on the Objet Connex 500 3-D Printing System. It weights 1.88
kg with the on-board computer and measures approximately 38
centimeters from leg to opposite leg in the crouch position depicted
in \figref{robot_close.jpg}. The robot is actuated by 9 AX-12+
Dynamixel servos: one inner joint and one outer joint servo in each of
the four legs, and one servo at the center ``hip'' joint.  This final
unique servo allows the two halves of the robot to rotate with respect
to each other. \figref{topdown.png} shows the positions and numerical
designations of all nine servoes.

All of the computation for gait learning, fitness evaluation, and robot control
was performed on the compact on-board CompuLab Fit-PC2, running Ubuntu
Linux.  All gait generation, learning, and fitness evaluation code,
except HyperNEAT, is in Python and is available on our website
\cite{quadraWeb}.  HyperNEAT is written in C++.  We controlled the servos with the \code{pydynamixel} library \cite{pydynamixel}.  The robot
connects to a wireless network on boot, which enabled us to control it via SSH.

%\acmFigg{wiiMote.jpg}{robot_led.jpg}{1}{A Nintendo Wii remote tracks the
%  location of the robot, providing feedback about distance traveled,
%  in pixels, through an infrared LED mounted on top of the robot.}

\acmFigg{wiiMote_crop.jpg}{robot_led_crop.jpg}{1}{A Nintendo Wii remote tracks the
  location of the robot, providing feedback about distance traveled,
  in pixels, through an infrared LED mounted on top of the robot.}


To track the position of the robot and thus determine gait fitness, we
mounted a Nintendo Wii remote on the ceiling and an infrared LED on
top of the robot (\figref{wiiMote_crop.jpg}).  The Wii remote contains an
IR camera that can track and report the position of any IR sources in
its image frame.  The resolution of the camera was 1024 by 768 pixels,
which produced a resolution of 1.7mm per pixel when mounted at a
height of \edit{???}m.

A separate tracking server, written in Python, ran on the robot and
interfaced with the Wii remote via bluetooth using the CWiid
library\cite{cwiid}.  Our fitness testing code connected to this
server via socket and requested position updates at the beginning and
end of each run (see \secref{fitnessEvaluation}).

%to interface with the remote via bluetooth. A client then connects via
%a socket to the tracking server and requests position updates
%periodically. If the robot walks beyond the viewable range of the Wii
%remote, a prompt informed the user. The only human intervention
%required during the experiment was to move the robot back inside the
%viewable area and resume the run, and to handle any mechnanical
%failures that arose. This did not interrupt the learning process or
%result in the loss of data.

To run a gait on the robot, a \emph{gait
  function} is provided, any Python function that accepts a single input --- time
starting at 0 --- and outputs a list of nine commanded positions, one
for each servo.  To safeguard against limb collision with the robot
body, the robot control code cropped the commands to a safe range.
This range was [150, 770] for the inner leg servos, [30, 680] for the
outer leg servos, and [392, 623] for the center hip servo.


%The robot was set up to run when given any giat 

%is run using a given motion model, including, if desired, 
%smooth interpolation over time for the beginning and end of the run. 

%The servos are prevented from being commanded to a point outside their
%normal range (0 - 1023) as well as beyond points where limbs would collide


	\subsection{Fitness evaluation details}
\seclabel{fitnessEvaluation}

\editbox{proofread.}

%\section{Experimental Evaluation}

We controlled the experiments from a computer that was connected via a
wireless Ethernet to the robot. An infrared LED was mounted to the
antenna on top of the robot and a Wii remote was attached to the
ceiling. The Wii remote tracked the location of the LED. The robot
tracked its position via Bluetooth by using the CWiid
library\cite{cwiid} to communicate with the Wii remote. If the robot
walked outside of the Wii remote's viewable area, a prompt informed
the experimenter. The only human intervention required during an
experiment was to move the robot back inside the viewable area and
resume the run. This did not interrupt the learning process or result
in the loss of data.

The metric for evaluation of the designed gait was speed. To evaluate
a set of parameters, the robot was sent the parameters and instructed
to walk for a certain length of time. For each evaluation, the robot
always started and ended in the same position in order to measure true
displacement and not reward gaits the ended in a lean, since the LED
would have moved a different distance the robot. More efficient
parameters resulted in a faster gait, which translated into a longer
distance walked and a better score.

Each of the parameter ??? methods was run on 3 different initial
parameter vectors, in order to allow for the fair comparison of the
algorithms. We allowed each run for the parameter ??? methods to
continue until the results plateaued (no improvement for one third of
the policies seen so far). Three runs of HyperNEAT were completed,
each with a different initial seed and run for 20 generations.




	
% Section 4
\section{Gait Generation and Learning}
	\seclabel{gaitGenLearn}
	\subsection{Parameterized Gaits}
\seclabel{motionModel}

We have chosen to implement two classes of algorithms to automatically create the motion-generating function $g(t)$ (described in
\secref{problemDefinition}).  The first method uses a family of
parameterized gaits, described in this section and the next. The
second method evolves a non-parametric gait controlled by neural networks, and is detailed in
\secref{hyperNeatMethod}.

By a \emph{parameterized gait}, we mean a gait produced by a
parameterized function $g(t; \vec{\theta})$. Fixing the parameter
$\vec{\theta}$ yields a deterministic motion function over time.  We
tried several parametrizations on the robot and, upon obtaining
reasonable early success, settled on one particular parametrization, which we call  \emph{SineModel5}. Its root pattern is a sine wave and it has five parameters (\tabref{params}).

%\newcommand{\amp}{\ensuremath{\mathrm{amp}}}
\newcommand{\amp}{\ensuremath{\alpha}}

\begin{table}[h!]
\begin{center}
\begin{tabular}{|c|c|c|}
\hline
Parameter    & Description               & Range \\
\hline
\hline
\amp         & Amplitude                 & [0, 400] \\
\hline
$\tau$       & Period                    & [.5, 8] \\
\hline
$m_O$        & Outer-motor multiplier    & [-2, 2] \\
\hline
$m_F$        & Front-motor multiplier    & [-1, 1] \\
\hline
$m_R$        & Right-motor multiplier    & [-1, 1] \\
\hline
\end{tabular}
\caption{The \emph{SineModel5} motion model parameters.}
\tablabel{parameters}
\label{tab:params}
\end{center}
\end{table}

Intuitively, SineModel5 starts with 8 identical sine waves of
amplitude $\amp$ and period $\tau$, multiplies the waves for all outer
motors by $m_O$, multiplies the waves for all front motors by $m_F$,
and multiplies the waves for all right motors by $m_R$.  To obtain the
actual motor position commands, these waves are offset by
appropriate fixed constants so that the base position (when the sine
waves are at 0) is approximately a crouch (the position shown in
\figref{robot_whitebg.jpg}).  In this motion model, the ninth
(center) motor is kept at a neutral position.  Thus, the commanded
position for all motors, as a vector function of time, is:

\[
\vec{g}(t) =
\left[ {\begin{array}{c@{ }l@{ }l}
\amp \cdot \sin(2\pi t / \tau) & \ \cdot \           m_F            & + C_I \\ % 0
\amp \cdot \sin(2\pi t / \tau) & \ \cdot \ m_O \cdot m_F            & + C_O \\ % 1
\amp \cdot \sin(2\pi t / \tau) & \                                  & + C_I \\ % 2
\amp \cdot \sin(2\pi t / \tau) & \ \cdot \ m_O                      & + C_O \\ % 3
\amp \cdot \sin(2\pi t / \tau) & \ \cdot \                     m_R  & + C_I \\ % 4
\amp \cdot \sin(2\pi t / \tau) & \ \cdot \ m_O           \cdot m_R  & + C_O \\ % 5
\amp \cdot \sin(2\pi t / \tau) & \ \cdot \           m_F \cdot m_R  & + C_I \\ % 6
\amp \cdot \sin(2\pi t / \tau) & \ \cdot \ m_O \cdot m_F \cdot m_R  & + C_O \\ % 7
0                              & \                                  & + C_C \\ % 8
\end{array} } \right]
\]

\noindent Here the fixed constants that determine the crouched
position are $C_I = 800$ for inner motors, $C_O = 40$ for outer
motors, and $C_C = 512$ for the center hip motor.  The nine functions (in order)
correspond to the nine motors listed in
\figref{topdown.png}.



\subsection{Learning Methods for Parameterized Gaits}
\seclabel{learningMethods}

Using the SineModel5 parameterized motion model described in the
previous section, along with the allowable ranges for each of the five
parameters (shown in \tabref{parameters}), the task becomes how to
choose the combination of five parameters that results in the fastest
motion, per the evaluation methods in \secref{fitnessEvaluation}.

If we choose a value for the five dimensional parameter
$\vec{\theta}$, then a given physical trial gives us one measurement
of the fitness $f(\vec{\theta})$ of that parameter vector.  Two
things make learning difficult.  First, each evaluation of
$f(\vec{\theta})$ is expensive, taking 15-20 seconds on
average.  Second, the fitness returned by such evaluations has proved
to be very noisy, with the standard deviation of the noise often being
roughly equivalent to the size of the measurement.

We test the ability of different \emph{learning algorithms} to choose the next value of $\vec{\theta}$ to try, given a list of the
$\vec{\theta}$ values already evaluated and their fitness measurements $f(\vec{\theta})$.

We evaluated a total of six different
learning algorithms for the parameterized motion models.  All methods
needed a way of picking a $\vec{\theta}$ for the first trial of each run.  This could be done by selecting a random starting vector within the
allowable domain.  However, to most directly compare learning methods
by their ability to improve on the initially chosen $\vec{\theta}$, we
decided to evaluate the different methods by starting them at the same three
initial $\vec{\theta}$ vectors in the three runs. 

%All employed a simple random sampling method for
%requirement (1); that is, all methods picked their initial
%$\vec{\theta}$ value via uniform random sampling within the allowed
%parameter ranges. \edit{make sure this is true. Isn't really, should say that we started with A, B, C}.  Thus, the
%differences in the algorithms was how they selected new $\vec{\theta}$
%values to try from their past experience.

The six learning algorithms for the parameterized motion models are as follows:



%\section{Method}
%\seclabel{method}

% Describe in reasonable detail the algorithm you are using to address
% this problem. A pseudo-code description of the algorithm you are
% using is frequently useful. If it makes sense for your project,
% trace through a concrete example, showing how your algorithm
% processes this example. The example should be complex enough to
% illustrate all of the important aspects of the problem but simple
% enough to be easily understood. If possible, an intuitively
% meaningful example is better than one with meaningless symbols.

%We use several parameterized motion models that command motors to
%positions based on a sine wave, creating a periodic pattern.  While we
%investigated several models, for the bulk of our experiments, we used
%a model whose five parameters are: amplitude, wavelength, scale inner
%vs outer motors, scale left vs right motors, scale back vs front
%motors. Each strategy below attempts to choose the best possible
%parameters for this motion model.  

%We implemented and tested 8 different learning strategies.  All
%strategies except for the HyperNEAT method\cite{clune} were
%constrained to pick parameters from within predetermined ranges.

%\edit{JMC:Bullets=narrower column=wasted space. subsubsection instead?}


\emph{Random}: This method randomly generates parameter vectors in the allowable range for every trial.  This strategy was used only
  as baseline for comparison.

\emph{Uniform random hill climbing}: This method always starts
  with the current best gait, and then chooses next $\vec{\theta}$ by
  randomly choosing one parameter to adjust and replacing it with a
  new value chosen with uniform probability in the allowable range for
  that parameter. This new point is evaluated, and if it results in a
  longer distance walked than the previous best gait, it is saved as
  the new best gait. The process is then repeated, always starting
  with the best gait.

\emph{Gaussian random hill climbing}: This method works
  similarly to Uniform random hill climbing, except the next
  $\vec{\theta}$ is generated by adding random Gaussian noise to the
  current best gait.  This results in all parameters being changed at
  once, but the resulting vector is always fairly close to the
  previous best gait.  We used independently selected noise in each
  dimension, scaled such that the standard deviation of the noise was
  5\% of the range of that dimension.

\emph{N-dimensional policy gradient ascent}: We implemented Kohl
  and Stone's \cite{kohl} method for local gradient ascent for gait
  learning with noisy fitness evaluations. This strategy explicitly
  estimates the gradient of the objective function. It does this by
  first generating $n$ parameter vectors near the initial vector by
  perturbing each dimension of each vector randomly by either
  $-\epsilon$, $0$, or $\epsilon$. Then each vector is run on the
  robot, and for each dimension we group the results into three
  groups: $-\epsilon$, $0$, and $\epsilon$.  The gradient along this
  dimension is then estimated as the average score for the $\epsilon$
  group minus the average score for the $-\epsilon$ group. Finally,
  the method creates the next $\vec{\theta}$ by changing all
  parameters by a fixed-size step in the direction of the gradient.
  For this study we used values of $\epsilon$ equal to 5\% of the
  allowable range in each dimension (ranges listed in
  \tabref{parameters}), and a step size scaled such that if all dimensions
  were in the range [0,~1], the norm of the step size would be 0.1.

%\item \emph{Nelder-Mead simplex method}: The Nelder-Mead simplex
%  method \cite{nm} creates an initial simplex with $d+1$ vertices,
%  where $d$ is the dimension of the parameter space. The initial
%  parameter vector is stored as the first vertex and the other five
%  vertices are created by adding to one dimension at a time one tenth
%  of the allowable range for that parameter. It then tests the fitness
%  of each vertex, and, in general, it reflects the worst point over
%  the centroid in an attempt to improve it.  However, to prevent
%  cycles and becoming stuck in local minima, several other rules are
%  used.  If the reflected point is better than the second worst point
%  and worse than the best point, then the reflected point replaces the
%  worst. If the reflected point is better than the best point, the
%  simplex is expanded in the direction of the reflected point. The
%  better of the reflected and the expanded point replaces the worst
%  point. If the reflected point is worse than the second worst point,
%  then the simplex is contracted away from the reflected point. If the
%  contracted point is better than the reflected point, the contracted
%  point replaces the worst point. If the contracted point is worse
%  than the reflected point, the entire simplex is shrunk \cite{nm}.

\emph{Nelder-Mead simplex method}: The Nelder-Mead simplex
  method \cite{nm} creates an initial simplex with $d+1$ vertices,
  where $d$ is the dimension of the parameter space. It then tests the
  fitness of each vertex, and, in general, it reflects the worst point
  over the centroid in an attempt to improve it.  Several additional
  rules are used to prevent cycles and local minima \cite{nm}.

\emph{Linear regression}: To initialize, this method chooses and
  evaluates five random parameter vectors. It then fits a linear model
  from parameter vector to fitness. In a loop, the method chooses and
  evaluates a new parameter vector generated by taking a fixed-size
  step in the direction of the gradient for each parameter, and fits a
  new linear model to all vectors evaluated so far, choosing the model
  to minimize the sum of squared errors. The step size is the same as in
  \emph{N-dimensional policy gradient ascent}.


Each of the six learning methods was run on 3 different initial
parameter vectors ($\vec{\theta}_A$,
$\vec{\theta}_B$, and $\vec{\theta}_C$) in order to fairly compare the algorithms. Three runs were performed per learning method. Runs continued until the performance plateaued, which we defined as when there was no improvement during the last third of
a run.

%\item \emph{SVM regression}: Similarly to linear regression, this
%  model starts with several random vectors, but this time they are
%  chosen in a small neighborhood about some initial random vector.
%  These vectors (generally 8) are evaluated, and a support vector
%  regression model is fit to the observed fitnesses.  To choose the
%  next vector for evaluation, we randomly generate 100 vectors in the
%  neighborhood of the best observed gait, and select for evaluation
%  the vector with the best predicted performance in the learned model.
%  We suspected that if we always chose the best predicted point out of
%  100, we may end up progressing along a narrow subspace, prohibiting
%  learning of the true local fitness function.  Put another way, we
%  would always choose exploitation of knowledge vs. exploration of the
%  space.  To address this concern, we added a small amount of noise to
%  the new chosen $\vec{\theta}$ before continuing with the next phase
%  of evaluation and modeling.
%
%  Such a method naturally has many tunable parameters, and we
%  endeavored to select these parameters by tuning the method in
%  simulation.  To estimate the performance of the algorithm, we ran it
%  against a simulation with a known optimum.  The simulated function
%  was in the same five dimensional parameter space, and simply
%  returned a fitness determined as the height of a Gaussian with a
%  random mean.  The width of the Gaussian in each dimension was 20\%
%  of the range of each dimension, and the maximum value at the peak
%  was 100.  \figref{svm_sim_results} shows the learning results on
%  this simulated model using the ultimately selected SVM parameters.
%  Interestingly, a non-zero value of \code{bumpBy} resulted in better
%  learning than noise free (exploration free) learning.
%
%  \edit{JMC: This next paragraph sounds like it belongs in Results?}
%  Ultimately, however, the version of SVM tuned for simulation still
%  did not show competitive performance on the real robot.  We tried
%  tuning some parameters on the real robot, but after some amount of
%  tuning, the method still exhibited too little exploration and easily
%  became stuck in local minima.






	\subsection{HyperNEAT Gait Generation and Learning}
\seclabel{hyperNeatMethod}
    
HyperNEAT is an indirect encoding for evolving artificial neural
networks (ANNs) that is inspired by the way natural organisms
develop~\citep{stanley2009hypercube}. It evolves Compositional Pattern
Producing Networks (CPPNs)~\citep{stanley2007compositional}, each of
which is a genome that encodes an ANN
phenotype~\citep{stanley2009hypercube}. Each CPPN is itself a directed
graph, where the nodes in the graph are mathematical functions, such as sine or
Gaussian. The nature of these functions can facilitate the evolution
of properties such as symmetry (e.g.\ a Gaussian function) and repetition (e.g.\ a sine
function)~\citep{stanley2009hypercube, stanley2007compositional}. The
signal on each link in the CPPN is multiplied by that link's weight,
which can magnify or diminish its effect.
  
A CPPN is queried once for each link in the ANN phenotype to determine
that link's weight~(\figref{hyperneatExplanation.png}). The inputs to the CPPN are the Cartesian
coordinates of both the source (e.g.\ $x = 2$, $y = 4$) and target
(e.g.\ $x = 3$, $y = 5$) nodes of a link and a constant bias
value. The CPPN takes these five values as inputs and produces two output values. The first output value
determines the weight of the link between the associated input
(source) and hidden layer (target) nodes, and the second output value
determines the weight of the link between the associated hidden
(source) and output (target) layer nodes. All pairwise combinations of
source and target nodes are iteratively passed as inputs to a CPPN to
determine the weight of each ANN link.

\acmFig{hyperneatExplanation.png}{.8}{HyperNEAT produces ANNs from
  CPPNs. ANN weights are specified as a function of the geometric
  coordinates of the source node and the target node for each
  connection. The coordinates of these nodes and a constant bias are
  iteratively passed to the CPPN to determine each connection
  weight. The CPPN has two output values, which specify the weights
  for each connection layer as shown.  Figure from
  \cite{clune2011performance}.}

HyperNEAT is capable of exploiting the geometry of a problem because
the link values between nodes in the final ANN phenotype are a
function of the geometric positions of those nodes~\citep{stanley2009hypercube, clune2009sensitivity,
  clune2011performance}. In the case of quadruped locomotion, this
property has previously been shown to help HyperNEAT produce gaits with front-back, left-right,
and four-way symmetries~\citep{clune2009evolving,
  clune2011performance}.
  
The evolution of the population of CPPNs occurs according to the
principles of the NeuroEvolution of Augmenting Topologies (NEAT)
algorithm~\citep{stanley2002evolving}, which was originally designed to
evolve ANNs. NEAT can be fruitfully applied to CPPNs because of their
structural similarity to ANNs. For example, mutations can add a node,
and thus a function, to a CPPN graph, or change its link weights. The
NEAT algorithm is unique in three main
ways~\citep{stanley2002evolving}. Initially, it starts with small
genomes that encode simple networks and slowly complexifies them via
mutations that add nodes and links to the network, enabling the
algorithm to evolve the topology of an ANN in addition to its
weights. Secondly, NEAT has a fitness-sharing mechanism that preserves
diversity in the system and gives time for new innovations to be tuned
by evolution before competing them against more adapted
rivals. Finally, NEAT tracks historical information to perform
intelligent crossover while avoiding the need for expensive
topological analysis. A full explanation of NEAT can be found
in~\citep{stanley2002evolving}.
  
The ANN configuration follows previous studies that evolved quadruped
gaits with HyperNEAT in simulation~\citep{clune2011performance,
  clune2009evolving}, but was adapted to accommodate the physical robot used in this paper. Specifically, the ANN has a fixed topology (i.e. the number of nodes does not evolve) that consists of three $3 \times 4$
Cartesian grids of nodes forming input, hidden, and output
layers (\figref{SpiderANN.jpg}). Adjacent layers were allowed to be completely connected, meaning that there
could be $(3 \times 4)^2= 288$ links in each ANN (although evolution can set weights to 0, functionally eliminating the connection). The inputs to the
substrate were the current \emph{commanded} angles of each of the 9 joints of the robot and a sine and cosine wave (to facilitate the
production of periodic behaviors).  The sine and cosine waves had a period of about half a second.

While measured angles could have been used as input instead of
commanded angles, this may have given HyperNEAT gaits an unfair
advantage by making them closed-loop. Thus, we restricted inputs to
the commanded angles only, which allowed us to produce completely
deterministic gaits using HyperNEAT.

The outputs of the substrate at each time step were nine numbers in
the range $[-1,1]$, which were scaled according to the allowable
ranges for each of the nine motors and then commanded
the positions for each motor.  Occasionally HyperNEAT would produce networks that
exhibited rapid oscillatory behaviors, switching from extreme negative to extreme positive numbers each time step.  This resulted in motor
commands to alternate extremes every 25ms (given the command rate of
40Hz), which tended to damage and overheat the motors.  To ameliorate
this problem, we requested four times as many commanded
positions from HyperNEAT ANN's and averaged over four commands at a time to
obtain the actual gait $g(t)$.  This solution worked well and did not restrict the expressiveness of HyperNEAT.
%\editbox{are we mentioning that we gave HyperNEAT the center joint
%  here or earlier?}

\acmFig{SpiderANN.jpg}{.7}{ANN configuration for HyperNEAT Runs. The
  first two columns of each row of the input layer receive information
  about a single leg (the current angle of its two joints). The final
  column provides a sine and cosine wave to enable periodic movements
  and the angle of the center joint. Evolution determines the function
  of the hidden-layer nodes. The nodes in the output layer specify new
  joint angles for each respective joint. The unlabeled nodes in the
  input and output layers are ignored. Figure adapted from
  \cite{clune2011performance}.}\label{hyperneatLayout}

As with the parameterized methods, three runs of HyperNEAT
were performed. Runs lasted 20 generations and the population size was 9 (which allowed a bare minimum of diversity within and between NEAT species).  These numbers are small, but were necessarily
constrained given how much time it took to conduct evolution directly
on a real robot. The remaining parameters were identical to Clune et
al.~(2011).
  

% Preliminary HyperNEAT runs were promising and resulted in several
% interesting gaits.

% Unfortunately, the gaits generated by HyperNEAT also tended to
% stress the robot more than typical gaits had before, and the servos
% would often overheat and malfunction, requiring restarts.  We think
% these issues may be addressed by adding a small layer between the
% HyperNEAT strategy and the robot that disallows quickly shifting
% commanded positions, and we hope to be able to test these methods
% further once this filter is in place.


% Section 5
\section{Results and Discussion}
\seclabel{results}

% Outline:
% - exploration of parameter space
% - parameterized optimization results, discussion
% - HyperNEAT results, discussion



\subsection{Exploration of Parameterized Gait Space}

Before optimizing the chosen family of parameterized gaits
(\secref{motionModel}) with learning methods, we performed an
experiment to explore the five dimensions of the SineModel5 parameter
space. Specifically, we selected a random parameter vector that
resulted in some motion, but not an exceptional gait. We then varied
each of the five parameters individually and measured performance,
repeating each measurement twice to get a rough estimate of the
measurement noise at each point.  The results of this exploration,
shown in \figref{explore_dim_1}, reveal that some dimensions ($\amp$,
$\tau$, $m_F$) are fairly smooth and exhibit global structure across
the allowable parameter range, while others ($m_O$, $m_R$) exhibit
more complex behavior.  In addition, it gives a rough indication that
measurement noise is often significant and is more likely to be large
for gaits that move more.  Of course, this is only a slice in each
dimension through a single point, and slices through a different point
could reveal different behavior.  The common point at the
intersection of all slices is shown as a red triangle in
each plot of \figref{explore_dim_1}.


%\acmFig{explore_dim_1}{1}{Fitness mean and standard deviation
%  vs. dimension 1.  The circle is a common point in
%  \figref{explore_dim_1} through \figref{explore_dim_5}}
%\acmFig{explore_dim_2}{1}{Fitness mean and standard deviation
%  vs. dimension 2.  The circle is a common point in
%  \figref{explore_dim_1} through \figref{explore_dim_5}}
%\acmFig{explore_dim_3}{1}{Fitness mean and standard deviation
%  vs. dimension 3.  The circle is a common point in
%  \figref{explore_dim_1} through \figref{explore_dim_5}}
%\acmFig{explore_dim_4}{1}{Fitness mean and standard deviation
%  vs. dimension 4.  The circle is a common point in
%  \figref{explore_dim_1} through \figref{explore_dim_5}}
%\acmFig{explore_dim_5}{1}{Fitness mean and standard deviation
%  vs. dimension 5.  The circle is a common point in
%  \figref{explore_dim_1} through \figref{explore_dim_5}}

\acmFiggggg{explore_dim_1}{explore_dim_2}{explore_dim_3}{explore_dim_4}{explore_dim_5}{1}{.53}{Fitness
  mean and standard deviation when each parameter dimension is varied
  independently.  The red triangle in each plot represents the same point in the 5-dimensional parameter space.}



\subsection{Learning Methods for Parameterized Gaits}

% Outline
% - beat hand gait
% - no learning strategy outperformed another by that much
% - real world noise yay

The results for the parameterized gaits are shown in
\figref{std_error} and \tabref{results}.  As a group, these strategies
produced results equal to or worse not significantly better than the randomly-generated
gaits, however, all methods managed to explore the space enough to
significantly improve on the previous hand coded gait in at least one
of the three runs.  No single strategy consistently outperformed the
others: for the first trial Linear Regression produced the fastest
gait at 27.58 body lengths/minute, for the second a random gait
actually won with 17.26 body lengths/minute, and for the third trial
the Nelder-Mead simplex method attained the fastest gait at 14.83 body
lengths/minute.



% octave:15> fit = [6.04 17.26 4.90;  11.37 9.44 2.69;  3.10 13.59 13.40;  0.68 14.69 3.60;  8.51 13.62 14.83;  27.58 12.51 1.95;  24.27  36.44  27.07];
% octave:16> mean(fit,2)
% ans =
% 
%     9.4000
%     7.8333
%    10.0300
%     6.3233
%    12.3200
%    14.0133
%    29.2600
% 
% octave:17> std(fit,0,2)
% ans =
% 
%     6.8308
%     4.5576
%     6.0023
%     7.3914
%     3.3546
%    12.8810
%     6.3737


%%%%%% OLD TABLE
% \begin{table*}
% \begin{center}
% \begin{tabular}{|r|c|c|c||c|}
% \hline
%                                          & A       & B      & C      &  Average \\
% \hline                                                               
% \hline                                                               
% Previous hand-coded gait                 & --      & --     & --     &  5.16 \\
% \hline                                                                 
% Random search                            & 6.04    & 17.26  & 4.90   &  9.40 \\
% \hline                                                                 
% Uniform Random Hill Climbing             & 11.37   & 9.44   & 2.69   &  7.83 \\
% \hline                                                                 
% Gaussian Random Hill Climbing            & 3.10    & 13.59  & 13.40  &  10.03 \\
% \hline                                                                 
% Policy Gradient Descent                  & 0.68    & 14.69  & 3.60   &  6.32 \\
% \hline                                                                 
% Nelder-Mead simplex                      & 8.51    & 13.62  & 14.83  &  12.32 \\
% \hline                                                                 
% Linear Regression                        & 27.58   & 12.51  & 1.95   &  14.01 \\
% \hline                                                                
% Evolutionary Neural Network (HyperNEAT)  & 24.27     & 36.44    & 27.07   & 29.26  \\
% \hline
% \end{tabular}
% \caption{The best gaits found for each starting vector and algorithm,
%   in body lengths per minute.}
% \tablabel{results}
% \end{center}
% \end{table*}

%%%%%%% NEW TABLE
\begin{table}
\begin{center}
\begin{tabular}{|r|c|c|c||c|}
\hline
                                         & Average & Std. Dev. \\
\hline                                    
\hline                                    
Previous hand-coded gait                 & 5.16   &   --     \\
\hline
Random search                            & 9.40   &   6.83   \\
\hline
Uniform Random Hill Climbing             & 7.83   &   4.56   \\
\hline
Gaussian Random Hill Climbing            & 10.03  &   6.00   \\
\hline
Policy Gradient Descent                  & 6.32   &   7.39   \\
\hline
Nelder-Mead simplex                      & 12.32  &   3.35   \\
\hline
Linear Regression                        & 14.01  &  12.88   \\
\hline
Evolved Neural Network              &        &          \\
(HyperNEAT)                              & 29.26  &   6.37   \\
\hline
\end{tabular}
\caption{The best gaits found for each starting vector and algorithm,
  in body lengths per minute.}
\tablabel{results}
\end{center}
\end{table}



\acmFig{std_error}{1}{Average results ($\pm $ SE) for each
  of the parameterized learning methods.  Linear regression found the
  fastest overall gait and had the highest average, followed by
  Nelder-Mead simplex. Many methods were beaten by the random
  strategy.}




\subsection{HyperNEAT Gaits}

The results for the gaits evolved by HyperNEAT are shown
in \figref{hnResults} and \tabref{results}.  Overall the HyperNEAT gaits
were the fastest by far, beating all the parameterized models when
comparing either average or best gaits.  We believe
that this is because HyperNEAT was allowed to explore a
much richer space of motions, but did so while still utilizing
symmetries when advantageous.
\figref{neat_110115_211410_00000_002_filt_zoom} shows a typical
HyperNEAT gait that had high fitness.  The pattern of motion is both
complex (containing multiple frequencies and repeating patterns across time)
and regular, in that patterns of multiple motors are coordinated. 

The evaluation of the gaits produced by HyperNEAT was more noisy than for the parameterized gaits, which made learning difficult. For example, we tested an example HyperNEAT generation-champion gait 11 times and found that its mean performance was 26 body lengths per minute ($\pm$ 13 SD), but it had a max of 38 and a min of 3.  Many effective HyperNEAT gaits were not preserved across generations because if performance in one trial was poor, the genome was unlikely to be selected for. We believe the slope of the HyperNEAT learning curve would be steeper if the noise in the evaluations could be reduced.

\acmFig{hnResults}{1}{Average fitness ($\pm $ SE) of
  the highest performing individual in the population for each generation of HyperNEAT runs. The fitness of many high-performing HyperNEAT gaits were halved if the gait overly stressed the motors (see text), meaning that HyperNEAT's true performance without this penalty would be much higher.}

%\acmFig{neat_110115_211410_00000_002_filt}{1}{Caption here...???}

\acmFig{neat_110115_211410_00000_002_filt_zoom}{1}{Example of one
  particular high-performance gait produced by HyperNEAT showing
  commands for each of nine motors.  Note the complexity of the motion
  pattern. Such patterns were not possible with the parameterized
  SineModel5, nor would they likely be contrived by a human designing
  a different low-dimensional parameterized motion model.}


% Section 6
\section{Conclusion and Future Work}
\seclabel{conclusion}

We have presented an array of approaches for optimizing a quadrupedal
gaits for speed.  We implemented and tested six learning
strategies for parameterized gaits and compared them to gaits produced by neural networks
evolved with the HyperNEAT generative encoding.

All methods resulted in an improvement over the robot's previous
hand-coded gait.  Building a model of gait
performance with linear regression to predict promising directions for further exploration
worked well, producing a gait of 27.5 body
lengths/minute.  The Nelder-Mead simplex method performed nearly as well, likely due to its robustness to noise.  The other parameterized
methods did not outperform random search.  One reason the randomly-generated SineModel5 gaits performed so well could be because the gait representation was biased towards effective, regular gaits, making the highly exploratory random strategy more effective than more exploitative learning algorithms. 

HyperNEAT produced higher-performing gaits than all of the parameterized
methods. Its best-performing gait traveled 45.7 body lengths per minute, which is nearly 9 times the speed of the hand-coded gait.  This could be because HyperNEAT tends to generate coordinated gaits~\citep{clune2011performance, clune2009evolving}, allowing it to
take advantage of the symmetries of the problem. HyperNEAT can also explore a much larger space of possibilities than the
more restrictive 5-dimensional parameterized space.  HyperNEAT gaits tended to produce more complex sequences of motor commands, with different frequencies and degrees of coordination, whereas the
parameterized gaits were restricted to scaling single-frequency sine waves and could only produce certain types of motor regularities. 

%\section{Future work}
\seclabel{futureWork}

\edit{write this, old section below}

% What are the major shortcomings of your current method? For each 
% shortcoming, propose additions or enhancements that would help overcome it.

% Outline:
% - allow feedback gaits (depending on measured servo positions,
%   loads, voltage drops, or other quantities)
% - better sensor that doesn't bias results downward.

There are several directions in which we could continue our
work. First, the error margins for our runs were large, so reducing
them by running more trials would lead to more conclusive data. It
would also be insightful to run our algorithms on different motion
models.  We suspect that our choice of motion model influenced the
results greatly, as even random choices in the space produced gaits
that moved a significant fraction of the time (> 5\%).  It would be
interesting to see how learning methods would perform using a model
that included a much higher percentage of unproductive gaits. We also
intend to experiment further with SVM regression and evolutionary
algorithms/HyperNEAT.  Some parameter vectors resulted in the robot
turning, as opposed to it walking long distances. This could be an
interesting learning goal in future projects. To these ends, we
propose the following additions and enhancements:

\begin{itemize}
\item More runs and/or longer runs
\item Different motion representations
\item Better tuning of SVM regression
\item Evolutionary algorithms/HyperNEAT
\item Learning how to turn
\end{itemize}


% Section 7
\section{Acknowledgments}

This work was supported in part by NSF CDI Grant ECCS 0941561, NSF
Creative-IT grant 0757478, and NSF Postdoctoral Research Fellowship
DBI-1003220.

%The content of this paper is solely the responsibility of the authors
%and does not necessarily represent the official views of the
%sponsoring organizations.''

% List any people not on the team who helped you with your project: Your
% TA, other people you consulted with or had useful discussions (say in 
% a word or two what they did), people who proofread your report, and 
% any external code you used (libraries etc).
%\begin{itemize}
%\item Hod Lipson, Cornell Computational Synthesis Lab: adviser.
%\item Jim T\o rreson, University of Oslo: adviser.
%\item Juan Zagal, University of Chile: designed and printed robot and provided code for hand tuned gait.
%\item Jeff Clune, CCSL Red Couch: collaborated on HyperNEAT implementation/testing.
%\item Cooper Bills, Cornell University: assisted with Wii tracker development.
%\item Anshumali Srivastava, Cornell University: Teaching Assistant.
%\end{itemize}




%%%%%%%%%%%%%%%%%%%%%%


%\section{References}

\bibliographystyle{abbrv}
\bibliography{references}

%\bibliography{references.bib}
%\thebibliography

%Generated by bibtex from your ~.bib file.  Run latex,
%then bibtex, then latex twice (to resolve references)
%to create the ~.bbl file.  Insert that ~.bbl file into
%the .tex source file and comment out
%the command \texttt{{\char'134}thebibliography}.

% This next section command marks the start of
% Appendix B, and does not continue the present hierarchy

\end{document}
