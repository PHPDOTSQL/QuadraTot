\section{Experimental Setup}

The quadruped robot used was assembled from parts purchased online and
parts printed by the Objet Connex 500 3-D Printing System. The robot
actuation system consists of 5 AX-18+ Dynamixel servos and 4 AX-12+
Dynamixel servos: one inner joint with one AX-18+ servo and one outer
joint with AX-12+servo in each of the four legs, and one AX-18+ servo
at the center. To avoid the formerly reported AX-18+ servos are used
in this robot because of their stronger actuation power than that of
AX-12+. Each servo could be set to a position in the range [0, 1023]
by using PyDynamixel library, corresponding to a physical range [-120$^\circ$,
  +120$^\circ$]. Also, to prevent collisions with the robot body, the control
module filter out the commands to a safe range. This range was [-85$^\circ$,
  +60$^\circ$] for the inner leg servos, [-113$^\circ$, +39$^\circ$] for the outer leg
servos, and [-28$^\circ$, +28$^\circ$] for the center hip servo. In the studies of
this paper, tethered cables powered both the computer and the
servos. It measures approximately 39.5 centimeters from leg to
opposite leg in the crouch position showed in \figref{quadratotWhiteBooties}. 

Our performance metric was the displacement over the evaluation period of 12 seconds
for each. Same as \cite{yosinski2011evolving-robot-gaits}, the displacement was
measured using a Wii remote that was placed on the ceiling. Different
from the original model described in \cite{yosinski2011evolving-robot-gaits}, the quadruped robot was
equipped with a three-infrared-LED cluster on top rather than just
one. The reason for this setup is that when fierce gaits were
executed, the Wii remote loses tracking of the robot position due to
the limited visible angle of a single LED.  These three LEDs were
place tightly together to act as one signal emitter. Each LED was
tilted outwards in order to maximize the visible range. A
separate tracking server ran on the robot PC interacted with the Wii
remote via bluetooth by using the CWiid library.  A corresponding
client communicates with this server via socket connection. Our
fitness evaluation code talks with the Wii remote by using this client
and updates the position from beginning to end during each run. The
metric for evaluating gaits was the Euclidian distance the robot
traveled during a 12-second run on flat surface. Previous work done in
\cite{yosinski2011evolving-robot-gaits} and \cite{clune2009evolving-coordinated-quadruped} suggested that extremely fierce gaits are not viable in
general. These gaits tend to either overburden the servos or flip the
robot. Thus, RL PoWER were tested after carefully cropping out the
extremely fierce gaits each time. As described earlier, each leg has
two joints, inner joint j1 and outer j2. The position of one leg is
determined by the sum of the values of j1 and j2. Extremely fierce
gaits usually have relatively small values of $g$. As shown in \figref{smartCrop}, this cropping method works
by mapping the wild gaits onto the boundary of a quasi-triangular area in
the two-dimensional space of j1 and j2.

\figp{smartCrop}{1}{The red line is the cropping border, $g = j1 +  j2$. As shown in \figref{sixLegPositions}}

\figp{sixLegPositions}{1}{The left two positions are allowed; middle
  two positions are on the red border line from \figref{smartCrop}; right two positions are deemed as "extremely fierce", thus are disallowed.}
