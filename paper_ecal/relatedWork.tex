\subsection{Related Work}

Various machine learning techniques have proved to be effective at
generating gaits for legged robots. Kohl and Stone presented a policy
gradient reinforcement learning approach for generating a fast walk on
legged robots \citep{kohl}, which we implemented for
comparison. Others have evolved gaits for legged robots, producing
competitive results \citep{chernova2005evolutionary,
  hornby2005autonomous, zykov, clune2009evolving,
  clune2011performance, clune2009hybrid, clune2009sensitivity,
  tellez2006evolving, valsalam2008modular}. In fact, an evolved gait
was used in the first commercially-available version of Sony's AIBO
robot~\citep{hornby2005autonomous}. Except for work with HyperNEAT
\citep{clune2009evolving, clune2011performance, clune2009hybrid,
  clune2009sensitivity}, the previous evolutionary approaches have
helped evolution exploit the regularity of the problem by manually
decomposing the task. Experimenters have to choose which legs should
be coordinated, or otherwise facilitate the coordination of
motion. Part of the motivation of this paper is to compare the
regularities produced by HyperNEAT to those generated by a more
systematic exploration of regularities via a parameterized model.
