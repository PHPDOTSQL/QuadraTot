\section{Introduction and Background}

\acmFig{robot_whitebg.jpg}{.75}{The quadruped robot for which gaits
  were evolved. The translucent parts were produced by a 3D
  printer. Videos of the gaits can be viewed at http://bit.ly/hLHBYq}

%Gaits for walking robots are often designed with some explicit or
%implicit goal in mind.  For some applications, the design criteria may
%be obvious --- perhaps the robot needs to move as quickly or as
%efficiently as possible --- but other times the objective is more complicated,
%requiring simultaneous optimization of several desired traits, each
%with its own relative importance.  The different combinations of desired traits and the relative weight placed on each can produce drastically different gaits.    For example, Honda's Asimo
%\citep{chestnutt2006footstep} and Boston Dynamic's Big Dog \citep{raibert2008bigdog} both require
%gaits that are relatively quick, power efficient, and robust to
%changing terrain, but they vary widely in the importance placed
%on each attribute.  Big Dog's gait is optimized for much more difficult terrain than Asimo's, resulting in a gait of a completely different form.

%Once the design goals are decided upon, gaits may be obtained by one

Legged robots have the potential to access many types of terrain
unsuitable for wheeled robots, but doing so requires the creation of a
gait specifying how the robot walks.  Such gaits may be designed either manually by an expert or
via computer learning algorithms.  It is advantageous to automatically
learn gaits because doing so can save valuable engineering time and allows
gaits to be customized to the idiosyncrasies of different robots.
Additionally, learned gaits have outperformed engineered gaits in some cases~\citep{hornby2005autonomous, valsalam2008modular}.

In this paper we compare the performance of two different methods of
learning gaits: parameterized gaits optimized with six different
learning methods, and gaits generated by evolving neural networks with
the HyperNEAT generative encoding~\citep{stanley2009hypercube}. While
some of these methods, such as HyperNEAT, have been tested in
simulation~\citep{clune2009evolving, clune2011performance}, we
investigate how they perform when evolving on a physical robot
(\figref{robot_whitebg.jpg}).  

Previous work has shown that quadruped gaits perform better when they
are \emph{regular} (i.e.\ when the legs are
coordinated)~\citep{clune2009evolving,
  clune2011performance,valsalam2008modular}. For example, HyperNEAT
produced fast, natural gaits in part because its bias towards regular
gaits created coordinated movements that outperformed gaits evolved by
an encoding not biased towards regularity~\citep{clune2009evolving,
  clune2011performance}. One of the motivations of this paper is to
investigate whether any learning method biased towards regularity
would perform well at producing quadruped gaits, or whether
HyperNEAT's high performance is due to additional factors, such as its
abstraction of biological development (described below). We test this
hypothesis by comparing HyperNEAT to six local search algorithms with
a parametrization biased toward regularity.

An additional motivation is to test whether gaits evolved in simulation transfer to reality well, especially those evolved with cutting-edge evolutionary algorithms. Because HyperNEAT gaits performed well in
simulation, it is interesting to test whether HyperNEAT can produce fast
gaits for a physical robot, including handling the noisy, unforgiving nature of the real world. Such tests help us better understand the real world implications of results reported only in simulation. It is additionally interesting to test how
more traditional gait optimization techniques compete with
evolutionary algorithms when evolving in hardware. A final motivation of this research is simply to evolve effective gaits for a physical robot.



% Put back in for reference... some good text here I think

% Applications of walking robots often call for the ability to walk as 
% quickly, efficiently, or with as little power as possible.  Often 
% these optimizations are done manually by an expert who designs and 
% tweaks a gait specifically for a given objective.  Other groups have 
% used learning methods to generate gaits optimized for some metric. 
% Approaches differ in their starting assumptions, some essentially 
% tweaking the parameters of a hand-tuned model \citep{chernova}, others 
% exploring a reasonably compact parameter space \citep{kohl}, and still 
% others beginning with few assumptions besides periodicity 
% \citep{zykov}. 
%  
% We aimed to strike a middle ground between these approaches.  Our 
% motion generator did not rely on hand-tweaked gaits, but it did use 
% parameterized gaits which, by their nature, impose some assumptions on 
% the answers produced.  We then used machine learning to design gaits 
% for a quadruped robot with these models.  This paper presents a 
% comparison of the different learning methods implemented.  Most 
% methods created walks that are several times faster than the original 
% hand-tuned gait.  We invite readers with short attention spans to view 
% a video of some of our results online here: 
%  
% 
% %\url{http://www.youtube.com/watch?v=ODoiOj9DdGg} 
% \texttt{http://www.youtube.com/watch?v=ODoiOj9DdGg} 

