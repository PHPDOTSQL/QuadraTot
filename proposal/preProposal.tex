\section{Team}

The human members of our team are Diana Hidalgo, Sarah Nguyen, and
Jason Yosinski.  The sole robotic member of our team is a 9 degree of
freedom starfish shaped robot from Hod Lipson's lab.



\section{Problem statement}

\edit{todo}

2.Problem statement and motivation. A statement of the problem, issue,
or task that you’re interested in studying, and why it is interesting
or important.



\section{I/O specification}

\edit{todo}

3.I/O Specification. A clear, concise description of what the final
system will do in terms of I/O behavior: What will it take in, and
what it will produce.



\section{Background reading (optional)}

\edit{todo}

4.Background Reading (OPTIONAL in pre-proposal stage). A list of
relevant readings that you’ll use to gain some background in your
selected topic.



\section{General approach (optional)}

\edit{todo}

5.General Approach (OPTIONAL in pre-proposal stage). A high-level
description of the general approach you’ll use (e.g. heuristic search,
learning, rules, belief networks). This section should have a
subsection entitled “Where’s the AI” where you explicitly articulate
the AI component of this project.



\section{System architecture and work plan (optional)}

\edit{todo}

6.System Architecture and work plan (OPTIONAL in pre-proposal
stage). Explain the main components of the system, how they can be
independently developed and independently tested, and who will do
what.



\section{Data sources (optional)}

\edit{todo}

7.Data sources (OPTIONAL in pre-proposal stage). If your project
involves analyzing real data (e.g. stock market data, sport data, user
behavior), identify potential sources for this data.



\section{Evaluation plans (optional)}

\edit{todo}

8.Evaluation Plans (OPTIONAL in pre-proposal stage). An explicit,
coherent plan for quantitatively and/or qualitatively evaluating the
system. In particular, identify (a) metrics that will help you measure
performance, e.g. time or space to solve a problem, quality of a
solution, or competition or comparison against a standard approach (b)
a simple “toy” problem that you can use for early testing of your
system – something that should be trivial to solve, (b) a “hard” test
case, that will serve as you ultimate test.



\section{Schedule}

Below is a rough schedule of fixed external deadlines and anticipated
team milestones.

\begin{center}
\begin{tabular}{|l||p{2.5in}|p{1.5in}|}
\hline
                        & {\bf Milestones} & {\bf Deadlines}
\\ \hline

Week 1 (9/13-9/19)    & Read papers, get lab access, talk to relevant other researchers & 9/17 Final proposals due
\\ \hline

Week 2 (9/20-9/26)    & Continue reading, get robot to move &
\\ \hline

Week 3 (9/27-10/03)   & Implement parametrized gait and determine proposed coding schedule for more advanced algorithms in time for Code Review \#1. &
\\ \hline

Week 4 (10/04-10/10)  & Begin main algorithm dev/testing effort & 10/5 Code Review \#1
\\ \hline

Week 5 (10/11-10/17)  & Algorithm dev/testing &
\\ \hline

Week 6 (10/18-10/24)  & Algorithm dev/testing &
\\ \hline

Week 7 (10/25-10/31)  & Algorithm dev/testing, quantify/solidify current results for Code Review \#2 &
\\ \hline

Week 8 (11/1-11/7)    & Finish collecting results, begin writing & 11/2 Code RReview \#2
\\ \hline

Week 9 (11/8-11/14)   & Finish collecting results, writing &
\\ \hline

Week 10 (11/15-11/21) & Finish collecting results, writing, get final demo ready &
\\ \hline

Week 11 (11/22-11/28) & Finish collecting results, writing, get final demo ready  &
\\ \hline

Week 12 (11/29-11/30) & Final demo  & 11/30 Final presentation
\\ \hline
\end{tabular}
\end{center}
