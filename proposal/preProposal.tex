\section{Team}

The human members of our team are Diana Hidalgo, Sarah Nguyen, and
Jason Yosinski.  The sole robotic member of our team is a 9 degree of
freedom starfish shaped robot from Hod Lipson's lab.



\section{Problem statement}

There has been much interest in and research surrounding legged
robots.  Such robots exhibit both challenges and opportunities not
present in more traditional wheeled robots.  One of these challenges
is getting the robot to move as quickly and efficiently as possible.
Past research has shown that machine learning methods of optimizing a
walking robot's gait can outperform even the best hand coded gaits
(Chernova 2004).

To this end, we will use machine learning to design a gait for a
quadraped robot, available from Hod Lipson's lab.  We anticipate using
speed and efficiency as metrics for evaluation of the designed gait;
these metrics are presented further in Section~\ref{sec:evaluation}.
The robot learning will be accomplished via one of several machine
learning algorithms, discussed in Section~\ref{sec:approach}.

%2.Problem statement and motivation. A statement of the problem, issue,
%or task that you’re interested in studying, and why it is interesting
%or important.



\section{I/O specification}


Input:

\begin{itemize}
\item Motor encoder position (optional, likely)
\item Motor force feedback (optional, likely)
\item Robot position estimate via external pose estimation system (optional, less likely)
\end{itemize}

 
\noindent Output:

\begin{itemize}
\item Motor position commands over time
\end{itemize}


\section{Background reading}

We have selected three papers to read for background information on
walking robots. The papers are \emph{An Evolutionary Approach to Gait
  Learning for Four-Legged Robots} by Sonia Chenova and Manuela
Veloso, \emph{Policy Gradient Reinforcement Learning for Fast
  Quadrepedal Locomotion} by Nate Kohl and Peter Stone, and
\emph{Evolving Dynamic Gaits on a Physical Robot} by Viktor Zykov,
Josh Bongard, and Hod Lipson. Hopefully these papers will give us
insights on what kinds of algorithms and techniques to employ in our
learning problem.




\section{General approach}
\label{sec:approach}

\subsection{Where's the AI?}

Our general approach is still under development.  We are currently
evaluating the following AI methods for possible inclusion:

\begin{itemize}
\item Optimization (gradient descent, Levenberg-Marquardt, ...)
\item Evolutionary algorithm / Genetic algorithm
\item Reinforcement learning
\item Supervised learning?
\end{itemize}

In addition to these more general methods, the following issues will
also need to be investigated:

\begin{itemize}
\item Reduced dimensionality of parameter space (periodic, symmetric, ...)
\item Parametrized gait vs. non-parametrized gait (?)
\item Geometric constraints
\end{itemize}



\section{System architecture and work plan}

We will be working with a robot that has an on-board computer running
either Windows or Linux. The lower level drivers are in C and we hope
to implement the system in Python.

The first week we will work together on making the robot move. Until we get the platrom ready to run our code, it will be difficult to separate the tasks. However, once we begin implementing our chosen algorithms, this division will become more clear.



% \section{Data sources (optional)}
% 
% \edit{todo}
% 
% 7.Data sources (OPTIONAL in pre-proposal stage). If your project
% involves analyzing real data (e.g. stock market data, sport data, user
% behavior), identify potential sources for this data.
% 


\section{Evaluation plans}
\label{sec:evaluation}

We will need to evaluate our system for two separate reasons.  First,
our learning methods depend on knowing the current performance of the
robot.  In order to optimize this performance, the learning algorithms
themselves will need access to the metrics of interest.  Of course,
the other reason for evaluation is to assess the performance of the
system as a whole.

For this latter purpose, we propose the use of three main metrics:

\begin{itemize}
\item Speed taken to walk a certain distance
\item Number of failures (falling down, getting stuck) over a certain number of attempts
\item Efficiency and power consumption, estimated through a model of the motor
\end{itemize}

\noindent We anticipate comparing our results along these dimensions
to a hard-coded walking configuration, as well as possibly to other
quadraped robots in the lab:




\section{Schedule}

Below is a rough schedule of fixed external deadlines and anticipated
team milestones.

\begin{center}
\begin{tabular}{|l||p{2.5in}|p{1.5in}|}
\hline
                        & {\bf Milestones} & {\bf Deadlines}
\\ \hline

Week 1 (9/13-9/19)    & Read papers, get lab access, talk to relevant other researchers & 9/17 Final proposals due
\\ \hline

Week 2 (9/20-9/26)    & Continue reading, get robot to move &
\\ \hline

Week 3 (9/27-10/03)   & Implement parametrized gait and determine proposed coding schedule for more advanced algorithms in time for Code Review \#1. &
\\ \hline

Week 4 (10/04-10/10)  & Begin main algorithm dev/testing effort & 10/5 Code Review \#1
\\ \hline

Week 5 (10/11-10/17)  & Algorithm dev/testing &
\\ \hline

Week 6 (10/18-10/24)  & Algorithm dev/testing &
\\ \hline

Week 7 (10/25-10/31)  & Algorithm dev/testing, quantify/solidify current results for Code Review \#2 &
\\ \hline

Week 8 (11/1-11/7)    & Finish collecting results, begin writing & 11/2 Code RReview \#2
\\ \hline

Week 9 (11/8-11/14)   & Finish collecting results, writing &
\\ \hline

Week 10 (11/15-11/21) & Finish collecting results, writing, get final demo ready &
\\ \hline

Week 11 (11/22-11/28) & Finish collecting results, writing, get final demo ready  &
\\ \hline

Week 12 (11/29-11/30) & Final demo  & 11/30 Final presentation
\\ \hline
\end{tabular}
\end{center}
